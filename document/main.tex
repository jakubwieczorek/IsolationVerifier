\documentclass[a4paper, keeplastbox]{jacow}

\usepackage{pdfpages,multirow,ragged2e}
\geometry{footskip=1cm}
\ifboolexpr{bool{xetex} or bool{luatex}} 
 {}                                      
 {\usepackage[utf8]{inputenc}}           

\usepackage[USenglish]{babel}
\usepackage{authblk}
\usepackage{float}
\usepackage[nottoc]{tocbibind}
\usepackage{listings}
\usepackage{tikz}
\usepackage{pgfplots}
\sisetup{detect-weight,exponent-product=\cdot,output-decimal-marker={,},per-mode=symbol,binary-units=true,range-phrase={-},range-units=single}
\usetikzlibrary{arrows,calc,decorations.markings,math,arrows.meta}

\ifboolexpr{bool{jacowbiblatex}}%
 {
  \addbibresource{jacow-test.bib}
  \addbibresource{biblatex-examples.bib}
 }{}
\listfiles

\makeatletter
\let\@fnsymbol\@arabic
\makeatother

\def\figurename{Fig.}
\def\listingname{Lst.}

\begin{document}

\title{Leakage detection system for isolation cooling pipes report}

\author{Jakub Wieczorek \thanks{jakub.lukasz.wieczorek@cern.ch EP/CMX Warsaw University of Technology (PL) The Institute of Electronic Systems}}

\maketitle

\section{Introduction}
The \verb|CMS| \verb|Preshower| detector is cooled down by \verb|C6F14| liquid. The pipes that are used to transfer it are partially isolated using \verb|Armaflex| (\figurename{} \ref{rys:pipe}) in order to cool down the ending device not the surrounding environment. \verb|Armaflex| is subjected to ageing (esp. given the presence of radiation). When isolation is indeed damaged due to for example time, then air gets between isolation material and pipe and because of the temperature difference in the vicinity of pipe and the pipe itself, water condensates on the pipe's surface and gets inside \verb|Armaflex| through small holes where isolation was damaged, so the possibility of presence of water or ice in the vicinity of the pipe or in the \verb|Armaflex| body cannot be excluded. That situation causes energy dissipation and the cooling system no longer cools down only the detector, but also condensed water and ice in the isolation and on the pipe. Therefore a non-intrusive method of monitoring should be developed in order to detect such situation. As it is impossible to cut the isolation in order to install a proper dew point or water sensor (that would also have to be radiation tolerant) it was decided to test the possibility to monitor the existence of water or ice as a function of the resistance measurement between two points in the insulating pipe material. Another approach would be studying a possible different temperature measurement in case the existence of water or ice but this would require the insertion of a radiation tolerant temperature sensor and would put back to the problems of installing a relative humidity dew point sensor in the insulation body. The main point of the project is to construct a non-intrusive reliable system which will detect if \verb|Armaflex| isolation is damaged or not by more primitive and simple technologies than digital sensors.

\section{Measurement principles/Description}
\subsection{Description}

The idea is to use two needles stuck into isolating material with accurate position (distance, depth) and finally measure and study resistance between that two needles as a function of the water presence.
\begin{figure}[H]
	\begin{center}
		\scalebox{.2}{\includegraphics{./pictures/pipe.jpg}}
	\end{center}
	\caption{Isolating pipe}
	\label{rys:pipe}
\end{figure}

At the beginning proposed solution of exploiting needles in order to measure humidity has to be explored thoroughly. Set of measurements were collected in time from complete dry isolation to fully wet and full of water. Experiments were performed for 30 minutes using \verb|Hewlett| \verb|Packard| \verb|4263B| \verb|LCR Meter| (\figurename{} \ref{rys:hewlett}), which was connected to the PC via \verb|GPIB-USB-HS| control device and \verb|MATLAB| where communication with \verb|LCR Meter| was implemented.

\begin{figure}[H]
	\begin{center}
		\scalebox{.25}{\includegraphics{./pictures/hewlett.jpg}}
	\end{center}
	\caption{Hewlett Packard 4263B LCR Meter}
	\label{rys:hewlett}
\end{figure}

The \verb|LCR| \verb|meter| allows to perform measurements of different quantities like capacitance or impedance. For the project purpose the device was configured to measure resistance. Both the device and the measurement configuration were performed in \verb|MATLAB| using \verb|Basic| language and commands.

A pair of needles were mounted on a calipers as shown on \figurename{} \ref{rys:needles} in order to exactly set and monitor the needles distance and verify that both are in the same depth. All the experiments were performed with isolation placed on the floor horizontally as well as the calipers. 

\begin{figure}[H]
	\begin{center}
		\scalebox{.25}{\includegraphics[angle=90,origin=c]{./pictures/needles.jpg}}
	\end{center}
	\caption{Needles mounted on the calipers with connected LCR Meter}
	\label{rys:needles}
\end{figure}

\section{Experiments}
A couple of experiments were performed with different needles configuration, that is a place of sticking into isolation and different amount of water. 
Needles were placed in three ways. First they pierced fully isolation in the middle, consequently they were stuck to the top and bottom edge also horizontally, what means that the needles did not crossed inner hole of the isolation. The last experiment was performed with the needles stuck in the middle, however only to the depth of 1cm -- before they reached an inner hole. 

\begin{figure}[H]
	\begin{center}
		\scalebox{.7}{% This file was created by matlab2tikz.
%
%The latest updates can be retrieved from
%  http://www.mathworks.com/matlabcentral/fileexchange/22022-matlab2tikz-matlab2tikz
%where you can also make suggestions and rate matlab2tikz.
%
\definecolor{mycolor1}{rgb}{0.00000,0.44700,0.74100}%
%
\begin{tikzpicture}

\begin{axis}[%
width=4.521in,
height=3.566in,
at={(0.758in,0.481in)},
scale only axis,
xmin=0,
xmax=30,
xlabel style={font=\color{white!15!black}},
xlabel={t},
ymin=-10000,
ymax=230626000,
ylabel style={font=\color{white!15!black}},
ylabel={R},
axis background/.style={fill=white},
xmajorgrids,
ymajorgrids,
legend style={legend cell align=left, align=left, draw=white!15!black}
]
\addplot [color=mycolor1]
  table[row sep=crcr]{%
0	85135700\\
0.01667	104307000\\
0.03334	99810400\\
0.05001	86722800\\
0.06668	101016000\\
0.08335	83570500\\
0.10002	122970000\\
0.11669	109422000\\
0.13336	97676100\\
0.15003	100366000\\
0.1667	103080000\\
0.18337	118850000\\
0.20004	78201900\\
0.21671	119926000\\
0.23338	87707900\\
0.25005	89795800\\
0.26672	102566000\\
0.28339	93943500\\
0.30006	105891000\\
0.31673	112400000\\
0.3334	98382400\\
0.35007	114188000\\
0.36674	98119600\\
0.38341	89768400\\
0.40008	101453000\\
0.41675	115252000\\
0.43342	117577000\\
0.45009	125183000\\
0.46676	128292000\\
0.48343	127022000\\
0.5001	138459000\\
0.51677	120607000\\
0.53344	136423000\\
0.55011	119849000\\
0.56678	131306000\\
0.58345	116699000\\
0.60012	101256000\\
0.61679	145725000\\
0.63346	167937000\\
0.65013	176639000\\
0.6668	179339000\\
0.68347	148686000\\
0.70014	176535000\\
0.71681	173546000\\
0.73348	167267000\\
0.75015	176257000\\
0.76682	156604000\\
0.78349	207126000\\
0.80016	185202000\\
0.81683	183992000\\
0.8335	195781000\\
0.85017	194707000\\
0.86684	162772000\\
0.88351	184969000\\
0.90018	181625000\\
0.91685	152426000\\
0.93352	116416000\\
0.95019	118296000\\
0.96686	114586000\\
0.98353	137691000\\
1.0002	81371500\\
1.01687	106485000\\
1.03354	90836700\\
1.05021	99746200\\
1.06688	111222000\\
1.08355	115719000\\
1.10022	89918600\\
1.11689	111136000\\
1.13356	105068000\\
1.15023	97698200\\
1.1669	87995700\\
1.18357	119660000\\
1.20024	105913000\\
1.21691	104971000\\
1.23358	121543000\\
1.25025	105704000\\
1.26692	128863000\\
1.28359	103056000\\
1.30026	125302000\\
1.31693	125137000\\
1.3336	120300000\\
1.35027	116906000\\
1.36694	128346000\\
1.38361	98011600\\
1.40028	125105000\\
1.41695	106380000\\
1.43362	103783000\\
1.45029	102021000\\
1.46696	79959500\\
1.48363	77993200\\
1.5003	82354100\\
1.51697	84112600\\
1.53364	104307000\\
1.55031	111456000\\
1.56698	101604000\\
1.58365	99773200\\
1.60032	113584000\\
1.61699	150180000\\
1.63366	171272000\\
1.65033	179897000\\
1.667	189469000\\
1.68367	202043000\\
1.70034	208154000\\
1.71701	209808000\\
1.73368	172048000\\
1.75035	173684000\\
1.76702	158608000\\
1.78369	104024000\\
1.80036	110000000\\
1.81703	93330100\\
1.8337	100760000\\
1.85037	126506000\\
1.86704	111685000\\
1.88371	130877000\\
1.90038	143213000\\
1.91705	180749000\\
1.93372	191488000\\
1.95039	213452000\\
1.96706	185600000\\
1.98373	165925000\\
2.0004	145394000\\
2.01707	132541000\\
2.03374	109995000\\
2.05041	77515800\\
2.06708	103125000\\
2.08375	92461900\\
2.10042	139286000\\
2.11709	146120000\\
2.13376	183182000\\
2.15043	179224000\\
2.1671	195003000\\
2.18377	192092000\\
2.20044	208061000\\
2.21711	151813000\\
2.23378	167692000\\
2.25045	111630000\\
2.26712	92900400\\
2.28379	125798000\\
2.30046	94794600\\
2.31713	122392000\\
2.3338	129534000\\
2.35047	189246000\\
2.36714	190605000\\
2.38381	203351000\\
2.40048	222102000\\
2.41715	161122000\\
2.43382	166436000\\
2.45049	149723000\\
2.46716	134712000\\
2.48383	109773000\\
2.5005	93811600\\
2.51717	89069600\\
2.53384	106366000\\
2.55051	134706000\\
2.56718	152103000\\
2.58385	161300000\\
2.60052	195157000\\
2.61719	190517000\\
2.63386	195709000\\
2.65053	212511000\\
2.6672	219786000\\
2.68387	179639000\\
2.70054	228516000\\
2.71721	150174000\\
2.73388	153765000\\
2.75055	161288000\\
2.76722	129091000\\
2.78389	102196000\\
2.80056	85082700\\
2.81723	88823400\\
2.8339	107417000\\
2.85057	92823900\\
2.86724	119632000\\
2.88391	106394000\\
2.90058	156696000\\
2.91725	124773000\\
2.93392	135893000\\
2.95059	175694000\\
2.96726	164825000\\
2.98393	169443000\\
3.0006	185847000\\
3.01727	209552000\\
3.03394	208686000\\
3.05061	207248000\\
3.06728	198879000\\
3.08395	158892000\\
3.10062	165685000\\
3.11729	150875000\\
3.13396	136812000\\
3.15063	101819000\\
3.1673	97366300\\
3.18397	113527000\\
3.20064	94101600\\
3.21731	117273000\\
3.23398	116615000\\
3.25065	115647000\\
3.26732	141775000\\
3.28399	131856000\\
3.30066	164519000\\
3.31733	155588000\\
3.334	175571000\\
3.35067	199210000\\
3.36734	208716000\\
3.38401	206096000\\
3.40068	197268000\\
3.41735	165123000\\
3.43402	192227000\\
3.45069	178133000\\
3.46736	136148000\\
3.48403	111003000\\
3.5007	128201000\\
3.51737	104270000\\
3.53404	112259000\\
3.55071	82774100\\
3.56738	100846000\\
3.58405	96578800\\
3.60072	81642600\\
3.61739	93017000\\
3.63406	102237000\\
3.65073	102696000\\
3.6674	93715900\\
3.68407	95253700\\
3.70074	112497000\\
3.71741	95505800\\
3.73408	91190500\\
3.75075	114098000\\
3.76742	119388000\\
3.78409	104026000\\
3.80076	82784100\\
3.81743	107091000\\
3.8341	112353000\\
3.85077	125329000\\
3.86744	103353000\\
3.88411	126002000\\
3.90078	152083000\\
3.91745	143245000\\
3.93412	130590000\\
3.95079	145925000\\
3.96746	137573000\\
3.98413	156481000\\
4.0008	144402000\\
4.01747	121808000\\
4.03414	141344000\\
4.05081	107661000\\
4.06748	110739000\\
4.08415	107232000\\
4.10082	105032000\\
4.11749	89465900\\
4.13416	99820100\\
4.15083	127207000\\
4.1675	108643000\\
4.18417	111724000\\
4.20084	134608000\\
4.21751	138224000\\
4.23418	148628000\\
4.25085	171302000\\
4.26752	183481000\\
4.28419	201055000\\
4.30086	211778000\\
4.31753	205093000\\
4.3342	196130000\\
4.35087	188580000\\
4.36754	222530000\\
4.38421	182320000\\
4.40088	197648000\\
4.41755	196676000\\
4.43422	209313000\\
4.45089	181798000\\
4.46756	199046000\\
4.48423	203810000\\
4.5009	195081000\\
4.51757	174065000\\
4.53424	190023000\\
4.55091	189911000\\
4.56758	187976000\\
4.58425	176627000\\
4.60092	130946000\\
4.61759	153470000\\
4.63426	144282000\\
4.65093	127314000\\
4.6676	138426000\\
4.68427	166144000\\
4.70094	145393000\\
4.71761	140508000\\
4.73428	126951000\\
4.75095	117687000\\
4.76762	142346000\\
4.78429	131513000\\
4.80096	144302000\\
4.81763	140893000\\
4.8343	169576000\\
4.85097	183679000\\
4.86764	149388000\\
4.88431	152218000\\
4.90098	163784000\\
4.91765	178053000\\
4.93432	175630000\\
4.95099	166812000\\
4.96766	194686000\\
4.98433	193740000\\
5.001	218269000\\
5.01767	215899000\\
5.03434	195400000\\
5.05101	196288000\\
5.06768	207470000\\
5.08435	183832000\\
5.10102	212940000\\
5.11769	153297000\\
5.13436	159961000\\
5.15103	165275000\\
5.1677	148722000\\
5.18437	142360000\\
5.20104	115014000\\
5.21771	120961000\\
5.23438	126578000\\
5.25105	118414000\\
5.26772	94085300\\
5.28439	129178000\\
5.30106	92346300\\
5.31773	98266800\\
5.3344	82753000\\
5.35107	113020000\\
5.36774	78762700\\
5.38441	108718000\\
5.40108	90813000\\
5.41775	89210600\\
5.43442	105668000\\
5.45109	86215800\\
5.46776	86079300\\
5.48443	98172100\\
5.5011	112893000\\
5.51777	80656600\\
5.53444	97403300\\
5.55111	98351600\\
5.56778	78553400\\
5.58445	89225200\\
5.60112	113249000\\
5.61779	105978000\\
5.63446	104827000\\
5.65113	104201000\\
5.6678	107442000\\
5.68447	144571000\\
5.70114	149340000\\
5.71781	129286000\\
5.73448	119952000\\
5.75115	153729000\\
5.76782	170753000\\
5.78449	186770000\\
5.80116	184325000\\
5.81783	194543000\\
5.8345	212072000\\
5.85117	203922000\\
5.86784	189912000\\
5.88451	201507000\\
5.90118	197675000\\
5.91785	157471000\\
5.93452	141405000\\
5.95119	145069000\\
5.96786	96774000\\
5.98453	80035200\\
6.0012	119465000\\
6.01787	92352800\\
6.03454	92664500\\
6.05121	80390300\\
6.06788	80003500\\
6.08455	111180000\\
6.10122	98223000\\
6.11789	84415000\\
6.13456	112120000\\
6.15123	87234300\\
6.1679	117972000\\
6.18457	64942100\\
6.20124	105005000\\
6.21791	98152000\\
6.23458	100233000\\
6.25125	107474000\\
6.26792	97760900\\
6.28459	90308200\\
6.30126	103164000\\
6.31793	110628000\\
6.3346	117862000\\
6.35127	134095000\\
6.36794	120214000\\
6.38461	117715000\\
6.40128	103357000\\
6.41795	95217500\\
6.43462	131864000\\
6.45129	114987000\\
6.46796	128018000\\
6.48463	97567100\\
6.5013	121577000\\
6.51797	129414000\\
6.53464	139236000\\
6.55131	156653000\\
6.56798	130578000\\
6.58465	131985000\\
6.60132	162346000\\
6.61799	153878000\\
6.63466	152300000\\
6.65133	180764000\\
6.668	176031000\\
6.68467	190465000\\
6.70134	204961000\\
6.71801	181115000\\
6.73468	192139000\\
6.75135	186655000\\
6.76802	217218000\\
6.78469	190115000\\
6.80136	222834000\\
6.81803	190196000\\
6.8347	179510000\\
6.85137	183455000\\
6.86804	199250000\\
6.88471	195693000\\
6.90138	189718000\\
6.91805	159727000\\
6.93472	150519000\\
6.95139	137239000\\
6.96806	100822000\\
6.98473	112903000\\
7.0014	88680400\\
7.01807	96658300\\
7.03474	113892000\\
7.05141	86479500\\
7.06808	67346500\\
7.08475	76502200\\
7.10142	96080800\\
7.11809	110041000\\
7.13476	128098000\\
7.15143	120064000\\
7.1681	139147000\\
7.18477	119726000\\
7.20144	132447000\\
7.21811	122814000\\
7.23478	143793000\\
7.25145	147086000\\
7.26812	129023000\\
7.28479	130584000\\
7.30146	156300000\\
7.31813	152513000\\
7.3348	151925000\\
7.35147	143784000\\
7.36814	150666000\\
7.38481	168826000\\
7.40148	166644000\\
7.41815	142735000\\
7.43482	158454000\\
7.45149	167568000\\
7.46816	152073000\\
7.48483	151612000\\
7.5015	120724000\\
7.51817	120344000\\
7.53484	100466000\\
7.55151	107097000\\
7.56818	122240000\\
7.58485	120282000\\
7.60152	108189000\\
7.61819	80823900\\
7.63486	120518000\\
7.65153	131446000\\
7.6682	134338000\\
7.68487	158883000\\
7.70154	167254000\\
7.71821	152649000\\
7.73488	179302000\\
7.75155	201605000\\
7.76822	209354000\\
7.78489	190420000\\
7.80156	194617000\\
7.81823	198696000\\
7.8349	146872000\\
7.85157	131849000\\
7.86824	94225200\\
7.88491	112658000\\
7.90158	100023000\\
7.91825	97539000\\
7.93492	74483800\\
7.95159	102127000\\
7.96826	87868100\\
7.98493	118434000\\
8.0016	123480000\\
8.01827	142657000\\
8.03494	131524000\\
8.05161	181060000\\
8.06828	174337000\\
8.08495	181522000\\
8.10162	188142000\\
8.11829	188105000\\
8.13496	207913000\\
8.15163	208808000\\
8.1683	187729000\\
8.18497	200753000\\
8.20164	183379000\\
8.21831	190841000\\
8.23498	137691000\\
8.25165	129760000\\
8.26832	118330000\\
8.28499	127845000\\
8.30166	96082000\\
8.31833	112578000\\
8.335	87372100\\
8.35167	105745000\\
8.36834	110255000\\
8.38501	93819900\\
8.40168	109978000\\
8.41835	88630400\\
8.43502	101654000\\
8.45169	108408000\\
8.46836	86301700\\
8.48503	127505000\\
8.5017	135194000\\
8.51837	154371000\\
8.53504	155344000\\
8.55171	177717000\\
8.56838	173507000\\
8.58505	186824000\\
8.60172	200906000\\
8.61839	215488000\\
8.63506	198463000\\
8.65173	186102000\\
8.6684	215416000\\
8.68507	160940000\\
8.70174	163883000\\
8.71841	140870000\\
8.73508	138905000\\
8.75175	132117000\\
8.76842	100605000\\
8.78509	102638000\\
8.80176	86355500\\
8.81843	86711800\\
8.8351	104641000\\
8.85177	76180200\\
8.86844	81961000\\
8.88511	90582500\\
8.90178	119477000\\
8.91845	118051000\\
8.93512	95652500\\
8.95179	104848000\\
8.96846	134033000\\
8.98513	133167000\\
9.0018	153191000\\
9.01847	148179000\\
9.03514	198677000\\
9.05181	167490000\\
9.06848	198107000\\
9.08515	190625000\\
9.10182	214430000\\
9.11849	179807000\\
9.13516	190168000\\
9.15183	189237000\\
9.1685	175124000\\
9.18517	186031000\\
9.20184	162906000\\
9.21851	178467000\\
9.23518	147493000\\
9.25185	121643000\\
9.26852	84850200\\
9.28519	101674000\\
9.30186	117309000\\
9.31853	113132000\\
9.3352	95438400\\
9.35187	99055700\\
9.36854	87207800\\
9.38521	105108000\\
9.40188	114908000\\
9.41855	109327000\\
9.43522	124095000\\
9.45189	143533000\\
9.46856	148534000\\
9.48523	167227000\\
9.5019	162003000\\
9.51857	210481000\\
9.53524	207939000\\
9.55191	186487000\\
9.56858	174837000\\
9.58525	167386000\\
9.60192	198624000\\
9.61859	152938000\\
9.63526	168185000\\
9.65193	144052000\\
9.6686	146165000\\
9.68527	124157000\\
9.70194	121374000\\
9.71861	123472000\\
9.73528	91017600\\
9.75195	113784000\\
9.76862	98506400\\
9.78529	105001000\\
9.80196	109108000\\
9.81863	107194000\\
9.8353	117517000\\
9.85197	128424000\\
9.86864	130520000\\
9.88531	180744000\\
9.90198	150982000\\
9.91865	175872000\\
9.93532	184955000\\
9.95199	208648000\\
9.96866	221619000\\
9.98533	189679000\\
10.002	209407000\\
10.01867	198281000\\
10.03534	176315000\\
10.05201	176755000\\
10.06868	174580000\\
10.08535	176531000\\
10.10202	178299000\\
10.11869	177566000\\
10.13536	171051000\\
10.15203	162619000\\
10.1687	129589000\\
10.18537	172805000\\
10.20204	147039000\\
10.21871	135714000\\
10.23538	116511000\\
10.25205	116046000\\
10.26872	90124500\\
10.28539	89321500\\
10.30206	110659000\\
10.31873	93185400\\
10.3354	100792000\\
10.35207	109883000\\
10.36874	95298200\\
10.38541	103007000\\
10.40208	108439000\\
10.41875	101395000\\
10.43542	144919000\\
10.45209	103877000\\
10.46876	133738000\\
10.48543	113401000\\
10.5021	165778000\\
10.51877	182543000\\
10.53544	170451000\\
10.55211	166147000\\
10.56878	173871000\\
10.58545	163415000\\
10.60212	172881000\\
10.61879	180449000\\
10.63546	176671000\\
10.65213	191961000\\
10.6688	198370000\\
10.68547	188278000\\
10.70214	199004000\\
10.71881	173611000\\
10.73548	200793000\\
10.75215	197055000\\
10.76882	172668000\\
10.78549	181919000\\
10.80216	133878000\\
10.81883	123769000\\
10.8355	106209000\\
10.85217	88424200\\
10.86884	94392300\\
10.88551	82812100\\
10.90218	86681400\\
10.91885	110927000\\
10.93552	100395000\\
10.95219	137903000\\
10.96886	104411000\\
10.98553	119876000\\
11.0022	133186000\\
11.01887	140260000\\
11.03554	148669000\\
11.05221	141944000\\
11.06888	161753000\\
11.08555	163410000\\
11.10222	201952000\\
11.11889	183020000\\
11.13556	190765000\\
11.15223	213062000\\
11.1689	210705000\\
11.18557	177804000\\
11.20224	164539000\\
11.21891	148282000\\
11.23558	133467000\\
11.25225	93893000\\
11.26892	114092000\\
11.28559	106234000\\
11.30226	90092100\\
11.31893	108741000\\
11.3356	107633000\\
11.35227	123687000\\
11.36894	127917000\\
11.38561	156036000\\
11.40228	166766000\\
11.41895	191235000\\
11.43562	180178000\\
11.45229	189169000\\
11.46896	180604000\\
11.48563	187323000\\
11.5023	209878000\\
11.51897	182977000\\
11.53564	172529000\\
11.55231	144262000\\
11.56898	110906000\\
11.58565	122347000\\
11.60232	105668000\\
11.61899	97603200\\
11.63566	99977200\\
11.65233	117186000\\
11.669	118048000\\
11.68567	115759000\\
11.70234	156935000\\
11.71901	156524000\\
11.73568	192278000\\
11.75235	189121000\\
11.76902	200146000\\
11.78569	184981000\\
11.80236	188935000\\
11.81903	139383000\\
11.8357	129570000\\
11.85237	113902000\\
11.86904	99260800\\
11.88571	84410400\\
11.90238	117820000\\
11.91905	123570000\\
11.93572	136280000\\
11.95239	151869000\\
11.96906	163847000\\
11.98573	196747000\\
12.0024	197694000\\
12.01907	209005000\\
12.03574	163380000\\
12.05241	170814000\\
12.06908	155154000\\
12.08575	122702000\\
12.10242	101116000\\
12.11909	95939500\\
12.13576	86007000\\
12.15243	120665000\\
12.1691	138210000\\
12.18577	155359000\\
12.20244	176831000\\
12.21911	210025000\\
12.23578	193271000\\
12.25245	177030000\\
12.26912	177529000\\
12.28579	184663000\\
12.30246	184123000\\
12.31913	127520000\\
12.3358	108326000\\
12.35247	108911000\\
12.36914	107210000\\
12.38581	103127000\\
12.40248	106553000\\
12.41915	80578900\\
12.43582	93115300\\
12.45249	122806000\\
12.46916	128937000\\
12.48583	125743000\\
12.5025	124488000\\
12.51917	165753000\\
12.53584	180060000\\
12.55251	154827000\\
12.56918	167806000\\
12.58585	178021000\\
12.60252	188335000\\
12.61919	196759000\\
12.63586	167523000\\
12.65253	202625000\\
12.6692	184174000\\
12.68587	214240000\\
12.70254	202571000\\
12.71921	203985000\\
12.73588	197976000\\
12.75255	194771000\\
12.76922	204772000\\
12.78589	182520000\\
12.80256	182130000\\
12.81923	186577000\\
12.8359	172302000\\
12.85257	119006000\\
12.86924	149039000\\
12.88591	110821000\\
12.90258	94443800\\
12.91925	100839000\\
12.93592	104813000\\
12.95259	111226000\\
12.96926	97045300\\
12.98593	84140800\\
13.0026	84677500\\
13.01927	112654000\\
13.03594	129416000\\
13.05261	109561000\\
13.06928	119592000\\
13.08595	122892000\\
13.10262	116073000\\
13.11929	148710000\\
13.13596	174982000\\
13.15263	157658000\\
13.1693	139507000\\
13.18597	156274000\\
13.20264	195173000\\
13.21931	183309000\\
13.23598	179135000\\
13.25265	209815000\\
13.26932	183550000\\
13.28599	198048000\\
13.30266	218519000\\
13.31933	203046000\\
13.336	202625000\\
13.35267	159191000\\
13.36934	152653000\\
13.38601	144528000\\
13.40268	130405000\\
13.41935	129855000\\
13.43602	107763000\\
13.45269	84247300\\
13.46936	98358000\\
13.48603	95701200\\
13.5027	72496000\\
13.51937	83395900\\
13.53604	89479900\\
13.55271	98152000\\
13.56938	87767500\\
13.58605	79197200\\
13.60272	91519600\\
13.61939	103921000\\
13.63606	110686000\\
13.65273	134747000\\
13.6694	100776000\\
13.68607	130041000\\
13.70274	110044000\\
13.71941	134947000\\
13.73608	150456000\\
13.75275	182306000\\
13.76942	173566000\\
13.78609	189211000\\
13.80276	198870000\\
13.81943	204701000\\
13.8361	197233000\\
13.85277	198833000\\
13.86944	192098000\\
13.88611	193650000\\
13.90278	161486000\\
13.91945	159697000\\
13.93612	170527000\\
13.95279	151780000\\
13.96946	147414000\\
13.98613	145806000\\
14.0028	133232000\\
14.01947	153721000\\
14.03614	137826000\\
14.05281	155764000\\
14.06948	145623000\\
14.08615	139377000\\
14.10282	126566000\\
14.11949	100780000\\
14.13616	129074000\\
14.15283	119298000\\
14.1695	99219600\\
14.18617	101498000\\
14.20284	103526000\\
14.21951	105379000\\
14.23618	75764900\\
14.25285	112414000\\
14.26952	102667000\\
14.28619	114679000\\
14.30286	105904000\\
14.31953	91959700\\
14.3362	95942000\\
14.35287	112708000\\
14.36954	117128000\\
14.38621	100572000\\
14.40288	109565000\\
14.41955	119289000\\
14.43622	120922000\\
14.45289	143729000\\
14.46956	130936000\\
14.48623	152932000\\
14.5029	159853000\\
14.51957	184023000\\
14.53624	185677000\\
14.55291	197525000\\
14.56958	170391000\\
14.58625	202006000\\
14.60292	216899000\\
14.61959	195141000\\
14.63626	213522000\\
14.65293	197512000\\
14.6696	190683000\\
14.68627	191124000\\
14.70294	201018000\\
14.71961	190243000\\
14.73628	174034000\\
14.75295	173804000\\
14.76962	199189000\\
14.78629	170692000\\
14.80296	158082000\\
14.81963	201730000\\
14.8363	139118000\\
14.85297	140111000\\
14.86964	156400000\\
14.88631	145642000\\
14.90298	142043000\\
14.91965	99626100\\
14.93632	131647000\\
14.95299	100079000\\
14.96966	79778600\\
14.98633	102468000\\
15.003	96583700\\
15.01967	94221200\\
15.03634	94343900\\
15.05301	122731000\\
15.06968	116125000\\
15.08635	112427000\\
15.10302	116193000\\
15.11969	144853000\\
15.13636	149266000\\
15.15303	203167000\\
15.1697	180849000\\
15.18637	185654000\\
15.20304	218156000\\
15.21971	182783000\\
15.23638	193971000\\
15.25305	183594000\\
15.26972	145831000\\
15.28639	162854000\\
15.30306	145567000\\
15.31973	125059000\\
15.3364	121984000\\
15.35307	107330000\\
15.36974	101578000\\
15.38641	90033800\\
15.40308	105311000\\
15.41975	109736000\\
15.43642	105846000\\
15.45309	103018000\\
15.46976	112547000\\
15.48643	154203000\\
15.5031	139151000\\
15.51977	169328000\\
15.53644	158250000\\
15.55311	163586000\\
15.56978	179366000\\
15.58645	200980000\\
15.60312	198339000\\
15.61979	192935000\\
15.63646	217925000\\
15.65313	211503000\\
15.6698	190590000\\
15.68647	161027000\\
15.70314	204185000\\
15.71981	198538000\\
15.73648	177835000\\
15.75315	187211000\\
15.76982	191075000\\
15.78649	173090000\\
15.80316	174115000\\
15.81983	153130000\\
15.8365	176061000\\
15.85317	156972000\\
15.86984	182889000\\
15.88651	166689000\\
15.90318	157329000\\
15.91985	146468000\\
15.93652	135447000\\
15.95319	142220000\\
15.96986	107778000\\
15.98653	109240000\\
16.0032	105785000\\
16.01987	104236000\\
16.03654	109890000\\
16.05321	116824000\\
16.06988	98701200\\
16.08655	92511700\\
16.10322	106738000\\
16.11989	129284000\\
16.13656	124629000\\
16.15323	124169000\\
16.1699	114897000\\
16.18657	123084000\\
16.20324	121195000\\
16.21991	135493000\\
16.23658	142132000\\
16.25325	128881000\\
16.26992	144893000\\
16.28659	182364000\\
16.30326	163946000\\
16.31993	194146000\\
16.3366	208186000\\
16.35327	192347000\\
16.36994	213199000\\
16.38661	221421000\\
16.40328	194396000\\
16.41995	186979000\\
16.43662	186744000\\
16.45329	182086000\\
16.46996	197036000\\
16.48663	209381000\\
16.5033	190114000\\
16.51997	182059000\\
16.53664	185686000\\
16.55331	175182000\\
16.56998	154774000\\
16.58665	163531000\\
16.60332	225212000\\
16.61999	208835000\\
16.63666	197387000\\
16.65333	192432000\\
16.67	199400000\\
16.68667	230626000\\
16.70334	203190000\\
16.72001	172909000\\
16.73668	197329000\\
16.75335	184642000\\
16.77002	179494000\\
16.78669	215873000\\
16.80336	190695000\\
16.82003	170749000\\
16.8367	150188000\\
16.85337	171633000\\
16.87004	138381000\\
16.88671	151019000\\
16.90338	162563000\\
16.92005	130303000\\
16.93672	116579000\\
16.95339	107782000\\
16.97006	120520000\\
16.98673	122356000\\
17.0034	107896000\\
17.02007	89085600\\
17.03674	128146000\\
17.05341	102857000\\
17.07008	107939000\\
17.08675	102493000\\
17.10342	100181000\\
17.12009	88451700\\
17.13676	113199000\\
17.15343	111573000\\
17.1701	110049000\\
17.18677	100322000\\
17.20344	113297000\\
17.22011	131386000\\
17.23678	100660000\\
17.25345	135322000\\
17.27012	94835700\\
17.28679	121586000\\
17.30346	108679000\\
17.32013	103506000\\
17.3368	118336000\\
17.35347	88096900\\
17.37014	100481000\\
17.38681	70620500\\
17.40348	88568400\\
17.42015	100560000\\
17.43682	96061300\\
17.45349	94409200\\
17.47016	115737000\\
17.48683	111948000\\
17.5035	131540000\\
17.52017	128005000\\
17.53684	133138000\\
17.55351	123471000\\
17.57018	141680000\\
17.58685	141523000\\
17.60352	167090000\\
17.62019	144585000\\
17.63686	149473000\\
17.65353	164968000\\
17.6702	124872000\\
17.68687	139324000\\
17.70354	155767000\\
17.72021	150543000\\
17.73688	148918000\\
17.75355	144159000\\
17.77022	169109000\\
17.78689	177918000\\
17.80356	168222000\\
17.82023	147540000\\
17.8369	137885000\\
17.85357	158861000\\
17.87024	152483000\\
17.88691	151024000\\
17.90358	155805000\\
17.92025	159888000\\
17.93692	158199000\\
17.95359	144231000\\
17.97026	123851000\\
17.98693	112104000\\
18.0036	106049000\\
18.02027	117266000\\
18.03694	117249000\\
18.05361	101683000\\
18.07028	84405400\\
18.08695	107567000\\
18.10362	91922500\\
18.12029	76500900\\
18.13696	115398000\\
18.15363	92659400\\
18.1703	100435000\\
18.18697	105853000\\
18.20364	122662000\\
18.22031	109630000\\
18.23698	134196000\\
18.25365	120977000\\
18.27032	110528000\\
18.28699	120760000\\
18.30366	103348000\\
18.32033	133146000\\
18.337	126991000\\
18.35367	114208000\\
18.37034	99735500\\
18.38701	136440000\\
18.40368	119461000\\
18.42035	117944000\\
18.43702	138034000\\
18.45369	138084000\\
18.47036	108572000\\
18.48703	118810000\\
18.5037	112602000\\
18.52037	119917000\\
18.53704	86627000\\
18.55371	107318000\\
18.57038	112445000\\
18.58705	107537000\\
18.60372	113522000\\
18.62039	122374000\\
18.63706	106084000\\
18.65373	103740000\\
18.6704	102418000\\
18.68707	97639800\\
18.70374	106362000\\
18.72041	115353000\\
18.73708	113138000\\
18.75375	93956900\\
18.77042	85782800\\
18.78709	117190000\\
18.80376	103370000\\
18.82043	90987500\\
18.8371	83727900\\
18.85377	83280700\\
18.87044	100231000\\
18.88711	109086000\\
18.90378	89929500\\
18.92045	92125800\\
18.93712	105777000\\
18.95379	101768000\\
18.97046	107182000\\
18.98713	115685000\\
19.0038	118986000\\
19.02047	157623000\\
19.03714	129024000\\
19.05381	161558000\\
19.07048	149269000\\
19.08715	165174000\\
19.10382	167364000\\
19.12049	187483000\\
19.13716	179807000\\
19.15383	181976000\\
19.1705	209440000\\
19.18717	223181000\\
19.20384	185169000\\
19.22051	208802000\\
19.23718	197785000\\
19.25385	174781000\\
19.27052	154874000\\
19.28719	173519000\\
19.30386	147530000\\
19.32053	160215000\\
19.3372	164047000\\
19.35387	151066000\\
19.37054	132512000\\
19.38721	117586000\\
19.40388	123456000\\
19.42055	108501000\\
19.43722	118473000\\
19.45389	125862000\\
19.47056	108544000\\
19.48723	105821000\\
19.5039	114671000\\
19.52057	97718200\\
19.53724	106059000\\
19.55391	104075000\\
19.57058	83873900\\
19.58725	94114600\\
19.60392	94641900\\
19.62059	90229100\\
19.63726	88202500\\
19.65393	106216000\\
19.6706	98263200\\
19.68727	108431000\\
19.70394	115048000\\
19.72061	139553000\\
19.73728	138364000\\
19.75395	147733000\\
19.77062	141830000\\
19.78729	174847000\\
19.80396	170595000\\
19.82063	193885000\\
19.8373	175332000\\
19.85397	184734000\\
19.87064	197585000\\
19.88731	184392000\\
19.90398	179553000\\
19.92065	165494000\\
19.93732	167163000\\
19.95399	187171000\\
19.97066	163627000\\
19.98733	165034000\\
20.004	147934000\\
20.02067	138972000\\
20.03734	128153000\\
20.05401	106270000\\
20.07068	123326000\\
20.08735	121911000\\
20.10402	150213000\\
20.12069	130037000\\
20.13736	128045000\\
20.15403	129764000\\
20.1707	107002000\\
20.18737	102357000\\
20.20404	119819000\\
20.22071	104923000\\
20.23738	125825000\\
20.25405	106892000\\
20.27072	132685000\\
20.28739	127393000\\
20.30406	123682000\\
20.32073	114596000\\
20.3374	155014000\\
20.35407	119362000\\
20.37074	156223000\\
20.38741	175551000\\
20.40408	159275000\\
20.42075	140497000\\
20.43742	157553000\\
20.45409	195789000\\
20.47076	191682000\\
20.48743	183051000\\
20.5041	226340000\\
20.52077	189253000\\
20.53744	205477000\\
20.55411	200048000\\
20.57078	191376000\\
20.58745	191498000\\
20.60412	176233000\\
20.62079	186344000\\
20.63746	158377000\\
20.65413	162162000\\
20.6708	149495000\\
20.68747	147162000\\
20.70414	126623000\\
20.72081	137632000\\
20.73748	119470000\\
20.75415	110891000\\
20.77082	99299300\\
20.78749	101608000\\
20.80416	89889900\\
20.82083	91857200\\
20.8375	91866800\\
20.85417	80156200\\
20.87084	96988100\\
20.88751	118638000\\
20.90418	104033000\\
20.92085	132877000\\
20.93752	119754000\\
20.95419	150188000\\
20.97086	188980000\\
20.98753	199420000\\
21.0042	194955000\\
21.02087	197527000\\
21.03754	209598000\\
21.05421	218563000\\
21.07088	214690000\\
21.08755	206820000\\
21.10422	184641000\\
21.12089	202385000\\
21.13756	185383000\\
21.15423	186664000\\
21.1709	186906000\\
21.18757	190021000\\
21.20424	182243000\\
21.22091	196076000\\
21.23758	162296000\\
21.25425	148254000\\
21.27092	170938000\\
21.28759	162034000\\
21.30426	148503000\\
21.32093	181415000\\
21.3376	188092000\\
21.35427	176448000\\
21.37094	157874000\\
21.38761	162916000\\
21.40428	157491000\\
21.42095	160262000\\
21.43762	170791000\\
21.45429	167220000\\
21.47096	151114000\\
21.48763	148877000\\
21.5043	143562000\\
21.52097	117641000\\
21.53764	137337000\\
21.55431	128901000\\
21.57098	129571000\\
21.58765	124813000\\
21.60432	91213200\\
21.62099	105500000\\
21.63766	117143000\\
21.65433	89177300\\
21.671	97808000\\
21.68767	102175000\\
21.70434	97073600\\
21.72101	99198500\\
21.73768	123278000\\
21.75435	118229000\\
21.77102	81410900\\
21.78769	108981000\\
21.80436	100526000\\
21.82103	112551000\\
21.8377	114201000\\
21.85437	106030000\\
21.87104	124185000\\
21.88771	138345000\\
21.90438	126328000\\
21.92105	133028000\\
21.93772	128490000\\
21.95439	159378000\\
21.97106	137603000\\
21.98773	132704000\\
22.0044	163393000\\
22.02107	163798000\\
22.03774	164063000\\
22.05441	158619000\\
22.07108	176517000\\
22.08775	169861000\\
22.10442	189978000\\
22.12109	190106000\\
22.13776	178918000\\
22.15443	198350000\\
22.1711	187841000\\
22.18777	191410000\\
22.20444	176275000\\
22.22111	176444000\\
22.23778	163848000\\
22.25445	172018000\\
22.27112	197127000\\
22.28779	153719000\\
22.30446	145641000\\
22.32113	154517000\\
22.3378	123063000\\
22.35447	113446000\\
22.37114	92214100\\
22.38781	88957500\\
22.40448	93765800\\
22.42115	87899900\\
22.43782	67364000\\
22.45449	125549000\\
22.47116	108234000\\
22.48783	109453000\\
22.5045	144473000\\
22.52117	124552000\\
22.53784	130620000\\
22.55451	104446000\\
22.57118	109502000\\
22.58785	124268000\\
22.60452	132632000\\
22.62119	124296000\\
22.63786	113124000\\
22.65453	106627000\\
22.6712	134221000\\
22.68787	122919000\\
22.70454	133190000\\
22.72121	124174000\\
22.73788	137319000\\
22.75455	121958000\\
22.77122	134330000\\
22.78789	139127000\\
22.80456	132660000\\
22.82123	150947000\\
22.8379	150251000\\
22.85457	147381000\\
22.87124	163162000\\
22.88791	163298000\\
22.90458	174700000\\
22.92125	144917000\\
22.93792	164873000\\
22.95459	159334000\\
22.97126	152201000\\
22.98793	143442000\\
23.0046	118733000\\
23.02127	134753000\\
23.03794	164009000\\
23.05461	99631700\\
23.07128	129518000\\
23.08795	133718000\\
23.10462	143884000\\
23.12129	113109000\\
23.13796	130780000\\
23.15463	136176000\\
23.1713	148176000\\
23.18797	144999000\\
23.20464	124436000\\
23.22131	148828000\\
23.23798	121584000\\
23.25465	119772000\\
23.27132	159344000\\
23.28799	145161000\\
23.30466	134457000\\
23.32133	139985000\\
23.338	135085000\\
23.35467	141794000\\
23.37134	146539000\\
23.38801	163860000\\
23.40468	141432000\\
23.42135	139573000\\
23.43802	184942000\\
23.45469	177910000\\
23.47136	170871000\\
23.48803	177774000\\
23.5047	152339000\\
23.52137	169234000\\
23.53804	155746000\\
23.55471	181164000\\
23.57138	189820000\\
23.58805	175270000\\
23.60472	196393000\\
23.62139	152516000\\
23.63806	153734000\\
23.65473	178391000\\
23.6714	152541000\\
23.68807	150634000\\
23.70474	141666000\\
23.72141	123890000\\
23.73808	129876000\\
23.75475	104902000\\
23.77142	127529000\\
23.78809	80000000\\
23.80476	117195000\\
23.82143	56660100\\
23.8381	109362000\\
23.85477	136312000\\
23.87144	108137000\\
23.88811	87507800\\
23.90478	115295000\\
23.92145	114787000\\
23.93812	107143000\\
23.95479	109970000\\
23.97146	125194000\\
23.98813	135029000\\
24.0048	128547000\\
24.02147	110300000\\
24.03814	162577000\\
24.05481	120821000\\
24.07148	154649000\\
24.08815	180650000\\
24.10482	184562000\\
24.12149	182724000\\
24.13816	199969000\\
24.15483	210673000\\
24.1715	189690000\\
24.18817	197085000\\
24.20484	209574000\\
24.22151	192066000\\
24.23818	191045000\\
24.25485	177105000\\
24.27152	190158000\\
24.28819	173173000\\
24.30486	147618000\\
24.32153	148782000\\
24.3382	106873000\\
24.35487	132089000\\
24.37154	113032000\\
24.38821	95452900\\
24.40488	101024000\\
24.42155	105998000\\
24.43822	84616200\\
24.45489	93499500\\
24.47156	100310000\\
24.48823	97285700\\
24.5049	121938000\\
24.52157	136889000\\
24.53824	135347000\\
24.55491	138024000\\
24.57158	133346000\\
24.58825	126314000\\
24.60492	118738000\\
24.62159	145350000\\
24.63826	138964000\\
24.65493	146338000\\
24.6716	137007000\\
24.68827	157962000\\
24.70494	143939000\\
24.72161	139214000\\
24.73828	145691000\\
24.75495	123247000\\
24.77162	142538000\\
24.78829	142745000\\
24.80496	139923000\\
24.82163	149961000\\
24.8383	144237000\\
24.85497	145001000\\
24.87164	110394000\\
24.88831	120961000\\
24.90498	122055000\\
24.92165	108419000\\
24.93832	141019000\\
24.95499	122953000\\
24.97166	123270000\\
24.98833	147275000\\
25.005	158016000\\
25.02167	169180000\\
25.03834	177372000\\
25.05501	171210000\\
25.07168	185286000\\
25.08835	207641000\\
25.10502	206931000\\
25.12169	193449000\\
25.13836	171193000\\
25.15503	214463000\\
25.1717	188159000\\
25.18837	184363000\\
25.20504	180453000\\
25.22171	216210000\\
25.23838	193677000\\
25.25505	203174000\\
25.27172	173038000\\
25.28839	163593000\\
25.30506	175857000\\
25.32173	193627000\\
25.3384	175409000\\
25.35507	176235000\\
25.37174	178405000\\
25.38841	187466000\\
25.40508	169034000\\
25.42175	174250000\\
25.43842	181419000\\
25.45509	181561000\\
25.47176	171376000\\
25.48843	183533000\\
25.5051	179878000\\
25.52177	165767000\\
25.53844	197812000\\
25.55511	191214000\\
25.57178	184008000\\
25.58845	200666000\\
25.60512	176225000\\
25.62179	195632000\\
25.63846	194772000\\
25.65513	221438000\\
25.6718	186143000\\
25.68847	198541000\\
25.70514	202239000\\
25.72181	189287000\\
25.73848	192881000\\
25.75515	228413000\\
25.77182	211006000\\
25.78849	189839000\\
25.80516	188840000\\
25.82183	153503000\\
25.8385	182078000\\
25.85517	186263000\\
25.87184	185382000\\
25.88851	151022000\\
25.90518	154344000\\
25.92185	134563000\\
25.93852	130116000\\
25.95519	134760000\\
25.97186	135641000\\
25.98853	111643000\\
26.0052	106516000\\
26.02187	109498000\\
26.03854	118109000\\
26.05521	99168800\\
26.07188	118940000\\
26.08855	97252000\\
26.10522	107340000\\
26.12189	92142500\\
26.13856	106425000\\
26.15523	84063700\\
26.1719	102582000\\
26.18857	93005900\\
26.20524	91216100\\
26.22191	100351000\\
26.23858	99363300\\
26.25525	112242000\\
26.27192	114101000\\
26.28859	119468000\\
26.30526	109444000\\
26.32193	134461000\\
26.3386	147037000\\
26.35527	162435000\\
26.37194	178209000\\
26.38861	165673000\\
26.40528	165814000\\
26.42195	177430000\\
26.43862	196344000\\
26.45529	215403000\\
26.47196	179840000\\
26.48863	204343000\\
26.5053	201396000\\
26.52197	179888000\\
26.53864	170401000\\
26.55531	212074000\\
26.57198	166337000\\
26.58865	173128000\\
26.60532	145465000\\
26.62199	155493000\\
26.63866	141444000\\
26.65533	105053000\\
26.672	123358000\\
26.68867	93887300\\
26.70534	76464500\\
26.72201	108455000\\
26.73868	100820000\\
26.75535	115699000\\
26.77202	70083800\\
26.78869	94485600\\
26.80536	126043000\\
26.82203	120830000\\
26.8387	139354000\\
26.85537	133953000\\
26.87204	155409000\\
26.88871	197152000\\
26.90538	180938000\\
26.92205	173485000\\
26.93872	220939000\\
26.95539	224062000\\
26.97206	182601000\\
26.98873	161031000\\
27.0054	143078000\\
27.02207	141484000\\
27.03874	134780000\\
27.05541	117888000\\
27.07208	103149000\\
27.08875	94930600\\
27.10542	68559100\\
27.12209	102700000\\
27.13876	118822000\\
27.15543	103898000\\
27.1721	133872000\\
27.18877	130018000\\
27.20544	158480000\\
27.22211	163160000\\
27.23878	174984000\\
27.25545	193621000\\
27.27212	209660000\\
27.28879	196564000\\
27.30546	174561000\\
27.32213	184336000\\
27.3388	154906000\\
27.35547	154759000\\
27.37214	144884000\\
27.38881	131125000\\
27.40548	98401300\\
27.42215	101907000\\
27.43882	96236200\\
27.45549	116465000\\
27.47216	92193500\\
27.48883	97769100\\
27.5055	137788000\\
27.52217	162780000\\
27.53884	151923000\\
27.55551	186108000\\
27.57218	202107000\\
27.58885	192808000\\
27.60552	200838000\\
27.62219	185876000\\
27.63886	201264000\\
27.65553	168922000\\
27.6722	162338000\\
27.68887	117025000\\
27.70554	100437000\\
27.72221	102697000\\
27.73888	116933000\\
27.75555	123714000\\
27.77222	136661000\\
27.78889	182018000\\
27.80556	176981000\\
27.82223	187227000\\
27.8389	188935000\\
27.85557	187448000\\
27.87224	171281000\\
27.88891	130431000\\
27.90558	97381200\\
27.92225	113154000\\
27.93892	85012800\\
27.95559	104152000\\
27.97226	97478800\\
27.98893	137565000\\
28.0056	160507000\\
28.02227	156801000\\
28.03894	183078000\\
28.05561	177535000\\
28.07228	194640000\\
28.08895	193526000\\
28.10562	205399000\\
28.12229	189141000\\
28.13896	167370000\\
28.15563	141744000\\
28.1723	151052000\\
28.18897	122206000\\
28.20564	109171000\\
28.22231	99485000\\
28.23898	98261100\\
28.25565	96198400\\
28.27232	107328000\\
28.28899	122415000\\
28.30566	141752000\\
28.32233	135412000\\
28.339	128132000\\
28.35567	131401000\\
28.37234	161440000\\
28.38901	182388000\\
28.40568	183446000\\
28.42235	191811000\\
28.43902	192015000\\
28.45569	187923000\\
28.47236	189643000\\
28.48903	211474000\\
28.5057	207843000\\
28.52237	199370000\\
28.53904	176905000\\
28.55571	184416000\\
28.57238	121647000\\
28.58905	114334000\\
28.60572	139215000\\
28.62239	83445200\\
28.63906	106951000\\
28.65573	97310300\\
28.6724	115447000\\
28.68907	97945900\\
28.70574	106002000\\
28.72241	106458000\\
28.73908	133608000\\
28.75575	147392000\\
28.77242	142921000\\
28.78909	145143000\\
28.80576	127760000\\
28.82243	158688000\\
28.8391	145960000\\
28.85577	150134000\\
28.87244	149609000\\
28.88911	139710000\\
28.90578	135635000\\
28.92245	146513000\\
28.93912	153568000\\
28.95579	159996000\\
28.97246	157442000\\
28.98913	165653000\\
29.0058	161403000\\
29.02247	153200000\\
29.03914	162710000\\
29.05581	155858000\\
29.07248	158520000\\
29.08915	139463000\\
29.10582	175533000\\
29.12249	150700000\\
29.13916	166556000\\
29.15583	156866000\\
29.1725	161461000\\
29.18917	176156000\\
29.20584	144488000\\
29.22251	168536000\\
29.23918	156531000\\
29.25585	161180000\\
29.27252	143947000\\
29.28919	159253000\\
29.30586	144390000\\
29.32253	141629000\\
29.3392	140832000\\
29.35587	143360000\\
29.37254	130382000\\
29.38921	151476000\\
29.40588	120617000\\
29.42255	122422000\\
29.43922	144379000\\
29.45589	150515000\\
29.47256	140557000\\
29.48923	116770000\\
29.5059	130819000\\
29.52257	118715000\\
29.53924	129832000\\
29.55591	117425000\\
29.57258	112508000\\
29.58925	116003000\\
29.60592	92939900\\
29.62259	117075000\\
29.63926	116500000\\
29.65593	86957700\\
29.6726	77222300\\
29.68927	122263000\\
29.70594	120817000\\
29.72261	98924200\\
29.73928	114106000\\
29.75595	81987300\\
29.77262	88710800\\
29.78929	89756700\\
29.80596	106339000\\
29.82263	101834000\\
29.8393	94398900\\
29.85597	92316500\\
29.87264	114891000\\
29.88931	127177000\\
29.90598	97025300\\
29.92265	106181000\\
29.93932	112416000\\
29.95599	115910000\\
29.97266	112148000\\
29.98933	135428000\\
};

\end{axis}
\end{tikzpicture}%}
	\end{center}
	\caption{Dry isolation}
	\label{fig:dry}
\end{figure}

First experiment was started by monitoring the resistance with "dry" isolating material. Process is shown on the \figurename{} \ref{fig:dry}. Having collected measurements for 30 minutes, stable value can be observed.

Second set of experiments were performed on the wetted isolation with the needles stuck in the middle of the isolation, deeply -- they pierced inner hole. Process can be observed on the \figurename{} \ref{fig:full_2}. At the beginning of the experiment till more or less \num{7.5} minute resistance was linear with a value close to 2M$\Omega$. After that it started to grow non-linearly to reach a threshold of 30M$\Omega$. It can be caused by gravitation, which got the water inside a isolation to the bottom or because wet surface started to evaporate. For this part three experiments were carried out, however one of them gave results far from these shown on the figure.

\begin{figure}[H]
	\begin{center}
		\scalebox{.7}{\includegraphics{./plots/full_1.pdf}}
	\end{center}
	\caption{Isolation full of water, isolation surface not wiped and needles stuck in the middle of the isolation}
	\label{fig:full_1}
\end{figure}

Second experiment was done as previous one with wet isolation, but this time needles were stuck to the top of isolation in order to not cross the inner hole what resulted in the \figurename{} \ref{fig:full_2}. There is huge diversity of the results in that experiments. Every particular test started from the same point, which is a couple of M$\Omega$ -- yellow curve for example in 5th minute shows a resistance on the level of 12M$\Omega$. After that 3 completely different thresholds were reached after growth till 20th minute. 

\begin{figure}[H]
	\begin{center}
		\scalebox{.7}{\includegraphics{./plots/full_2.pdf}}
	\end{center}
	\caption{Isolation full of water, isolation surface not wiped and needles stuck on the top of the isolation}
	\label{fig:full_2}
\end{figure}

Yellow curve stopped at 1G$\Omega$, orange one at 130M$\Omega$, which is really strange result and can be caused by some failure on the side of \verb|LTC| \verb|Meter|. Blue curve started at 350K$\Omega$ and reached limit of 2M$\Omega$. Conclusion from this experiment is following. In spite of potential failure with measuring device needles stuck to the top of isolation horizontally give very different results, caused by most likely different amount of water in the isolation and the gravitational issue explained during previous experiment. Top of the isolation is not good place to stick the needles.

\begin{figure}[H]
	\begin{center}
		\scalebox{.7}{\input{./plots/full_3.tex}}
	\end{center}
	\caption{Isolation full of water with cable, isolation surface not wiped and needles stuck vertically on the bottom of the isolation}
	\label{fig:full_3}
\end{figure}

Next experiments were performed with a cable tightly plugged inside the isolation. Result of the experiment with needles stuck to the bottom of the \verb|Armaflex| is presented on \figurename{} \ref{fig:full_3}. Result is quite accurate according to the previous tests. Resistance is spread over \num{0.4}M$\Omega$ and \num{1.2}M$\Omega$ throughout 30 minutes of measurements. Differences can be explained by the fact that needles were not stuck exactly in the same place in 3 experiments. Resistance is not growing as fast as during previous experiments because of the cable presence what keeps water more or less in the same place in that short period of time.

Next experiment was done with the isolation wetted only inside the hole -- water got only to some point in the diameter, what is shown on the \figurename{} \ref{fig:inside}. Results are highly inaccurate and it is caused by the difference in water amount within the isolation and potential fake readouts from the device. 

\begin{figure}[H]
	\begin{center}
		\scalebox{.7}{% This file was created by matlab2tikz.
%
%The latest updates can be retrieved from
%  http://www.mathworks.com/matlabcentral/fileexchange/22022-matlab2tikz-matlab2tikz
%where you can also make suggestions and rate matlab2tikz.
%
\definecolor{mycolor1}{rgb}{0.00000,0.44700,0.74100}%
\definecolor{mycolor2}{rgb}{0.85000,0.32500,0.09800}%
\definecolor{mycolor3}{rgb}{0.92900,0.69400,0.12500}%
\definecolor{mycolor4}{rgb}{0.49400,0.18400,0.55600}%
\definecolor{mycolor5}{rgb}{0.46600,0.67400,0.18800}%
\definecolor{mycolor6}{rgb}{0.30100,0.74500,0.93300}%
%
\begin{tikzpicture}

\begin{axis}[%
width=4.521in,
height=3.566in,
at={(0.758in,0.481in)},
scale only axis,
xmin=0,
xmax=29.9999996904444,
xlabel style={font=\color{white!15!black}},
xlabel={t},
ymin=-10000,
ymax=144450993.060976,
ylabel style={font=\color{white!15!black}},
ylabel={R},
axis background/.style={fill=white},
xmajorgrids,
ymajorgrids,
legend style={legend cell align=left, align=left, draw=white!15!black}
]
\addplot [color=mycolor1]
  table[row sep=crcr]{%
0	69014100\\
0.0166666666666667	69395500\\
0.0333333333333333	68665700\\
0.05	71973300\\
0.0666666666666667	70930400\\
0.0833333333333333	72845300\\
0.1	65286700\\
0.116666666666667	71761200\\
0.133333333333333	73572700\\
0.15	68272800\\
0.166666666666667	73093200\\
0.183333333333333	71966200\\
0.2	79781700\\
0.216666666666667	83282600\\
0.233333333333333	101739000\\
0.25	97608400\\
0.266666666666667	75140900\\
0.283333333333333	80223500\\
0.3	81214100\\
0.316666666666667	77432600\\
0.333333333333333	76416200\\
0.35	79052600\\
0.366666666666667	74954900\\
0.383333333333333	77849900\\
0.4	81244100\\
0.416666666666667	72967100\\
0.433333333333333	74784800\\
0.45	78021700\\
0.466666666666667	78906900\\
0.483333333333333	75960500\\
0.5	70901400\\
0.516666666666667	72372400\\
0.533333333333333	69634500\\
0.55	77071000\\
0.566666666666667	73337900\\
0.583333333333333	72586500\\
0.6	70429000\\
0.616666666666667	79039200\\
0.633333333333333	71044000\\
0.65	72860300\\
0.666666666666667	73512000\\
0.683333333333333	70993200\\
0.7	73919200\\
0.716666666666667	72413700\\
0.733333333333333	67923100\\
0.75	65436800\\
0.766666666666667	61619200\\
0.783333333333333	60132100\\
0.8	60708500\\
0.816666666666667	65117600\\
0.833333333333333	63617700\\
0.85	59398700\\
0.866666666666667	58913600\\
0.883333333333333	61138700\\
0.9	58786200\\
0.916666666666667	60743900\\
0.933333333333333	63890400\\
0.95	65021700\\
0.966666666666667	65424100\\
0.983333333333333	65697900\\
1	58967200\\
1.01666666666667	60591000\\
1.03333333333333	61447900\\
1.05	60093600\\
1.06666666666667	55335500\\
1.08333333333333	58937200\\
1.1	58347600\\
1.11666666666667	64722700\\
1.13333333333333	58081200\\
1.15	62926800\\
1.16666666666667	64784900\\
1.18333333333333	62806100\\
1.2	62864400\\
1.21666666666667	54704900\\
1.23333333333333	61405600\\
1.25	63847200\\
1.26666666666667	66990200\\
1.28333333333333	66871400\\
1.3	61235400\\
1.31666666666667	56631400\\
1.33333333333333	60097400\\
1.35	60459200\\
1.36666666666667	54068900\\
1.38333333333333	66078000\\
1.4	71804000\\
1.41666666666667	72424600\\
1.43333333333333	71956500\\
1.45	69843900\\
1.46666666666667	75717600\\
1.48333333333333	72210700\\
1.5	72391300\\
1.51666666666667	69835900\\
1.53333333333333	72429700\\
1.55	68760200\\
1.56666666666667	67734800\\
1.58333333333333	61696600\\
1.6	56143700\\
1.61666666666667	60180100\\
1.63333333333333	61825800\\
1.65	61574900\\
1.66666666666667	57488400\\
1.68333333333333	63151000\\
1.7	63829300\\
1.71666666666667	62312100\\
1.73333333333333	58874400\\
1.75	70356800\\
1.76666666666667	63993300\\
1.78333333333333	72780700\\
1.8	64006100\\
1.81666666666667	66970900\\
1.83333333333333	63449600\\
1.85	67391000\\
1.86666666666667	68575800\\
1.88333333333333	65703400\\
1.9	68472600\\
1.91666666666667	70332800\\
1.93333333333333	67780500\\
1.95	73345400\\
1.96666666666667	73250800\\
1.98333333333333	69444300\\
2	67540300\\
2.01666666666667	71150800\\
2.03333333333333	71127200\\
2.05	75915900\\
2.06666666666667	70536800\\
2.08333333333333	72739100\\
2.1	70031400\\
2.11666666666667	67317200\\
2.13333333333333	76714800\\
2.15	69051900\\
2.16666666666667	71967300\\
2.18333333333333	68670500\\
2.2	72107400\\
2.21666666666667	68070300\\
2.23333333333333	72330200\\
2.25	66794700\\
2.26666666666667	68109500\\
2.28333333333333	65064200\\
2.3	66827100\\
2.31666666666667	69716800\\
2.33333333333333	61734600\\
2.35	62861700\\
2.36666666666667	62425100\\
2.38333333333333	61956400\\
2.4	60746100\\
2.41666666666667	63115700\\
2.43333333333333	66317400\\
2.45	61557500\\
2.46666666666667	67275700\\
2.48333333333333	59979100\\
2.5	60981400\\
2.51666666666667	62852900\\
2.53333333333333	60781800\\
2.55	58348500\\
2.56666666666667	60331700\\
2.58333333333333	61870900\\
2.6	56400400\\
2.61666666666667	58923800\\
2.63333333333333	64816400\\
2.65	62479300\\
2.66666666666667	70105500\\
2.68333333333333	63685500\\
2.7	69603400\\
2.71666666666667	60244500\\
2.73333333333333	65892700\\
2.75	65619600\\
2.76666666666667	61880400\\
2.78333333333333	64041300\\
2.8	67973000\\
2.81666666666667	65390500\\
2.83333333333333	68768400\\
2.85	66017300\\
2.86666666666667	72600800\\
2.88333333333333	73974800\\
2.9	75602400\\
2.91666666666667	81670700\\
2.93333333333333	70674400\\
2.95	75709100\\
2.96666666666667	69485000\\
2.98333333333333	70948000\\
3	65577300\\
3.01666666666667	62637300\\
3.03333333333333	56021200\\
3.05	56561700\\
3.06666666666667	52221900\\
3.08333333333333	50156900\\
3.1	52050000\\
3.11666666666667	54916900\\
3.13333333333333	52788000\\
3.15	59484800\\
3.16666666666667	54111300\\
3.18333333333333	58179800\\
3.2	60835500\\
3.21666666666667	60009400\\
3.23333333333333	59568100\\
3.25	59927200\\
3.26666666666667	66478000\\
3.28333333333333	71946900\\
3.3	68097800\\
3.31666666666667	80527000\\
3.33333333333333	78537000\\
3.35	77664600\\
3.36666666666667	79282800\\
3.38333333333333	86330000\\
3.4	77097100\\
3.41666666666667	78128300\\
3.43333333333333	71379400\\
3.45	81679400\\
3.46666666666667	71628900\\
3.48333333333333	76068200\\
3.5	80395600\\
3.51666666666667	70080800\\
3.53333333333333	71197200\\
3.55	69314700\\
3.56666666666667	71103900\\
3.58333333333333	69213100\\
3.6	71873900\\
3.61666666666667	72103300\\
3.63333333333333	72554600\\
3.65	73637600\\
3.66666666666667	75455600\\
3.68333333333333	73529600\\
3.7	67712600\\
3.71666666666667	69382000\\
3.73333333333333	67231400\\
3.75	65415700\\
3.76666666666667	70963900\\
3.78333333333333	77491700\\
3.8	68451000\\
3.81666666666667	70907100\\
3.83333333333333	73936000\\
3.85	72091800\\
3.86666666666667	69328500\\
3.88333333333333	69518100\\
3.9	69659300\\
3.91666666666667	62042000\\
3.93333333333333	71325600\\
3.95	69601700\\
3.96666666666667	73162800\\
3.98333333333333	71425100\\
4	72783300\\
4.01666666666667	71854500\\
4.03333333333333	63751500\\
4.05	67897000\\
4.06666666666667	68033200\\
4.08333333333333	69131700\\
4.1	63636000\\
4.11666666666667	70782900\\
4.13333333333333	72340100\\
4.15	74805300\\
4.16666666666667	67941400\\
4.18333333333333	73339700\\
4.2	68872000\\
4.21666666666667	75483800\\
4.23333333333333	75423300\\
4.25	69697300\\
4.26666666666667	71343600\\
4.28333333333333	75090200\\
4.3	64440200\\
4.31666666666667	68611300\\
4.33333333333333	59460500\\
4.35	65214900\\
4.36666666666667	64727000\\
4.38333333333333	69997400\\
4.4	71369300\\
4.41666666666667	70370900\\
4.43333333333333	65593100\\
4.45	67522400\\
4.46666666666667	60566300\\
4.48333333333333	63731100\\
4.5	63756800\\
4.51666666666667	70958800\\
4.53333333333333	63162400\\
4.55	63561500\\
4.56666666666667	67627700\\
4.58333333333333	72145400\\
4.6	74844800\\
4.61666666666667	76715800\\
4.63333333333333	71514500\\
4.65	73175800\\
4.66666666666667	74848500\\
4.68333333333333	65531800\\
4.7	69768500\\
4.71666666666667	70988400\\
4.73333333333333	70902700\\
4.75	74970300\\
4.76666666666667	72901500\\
4.78333333333333	70636900\\
4.8	68268000\\
4.81666666666667	72141300\\
4.83333333333333	68901400\\
4.85	72047300\\
4.86666666666667	68484800\\
4.88333333333333	73538800\\
4.9	72763700\\
4.91666666666667	68945100\\
4.93333333333333	70568900\\
4.95	79042200\\
4.96666666666667	71734000\\
4.98333333333333	77935500\\
5	70307900\\
5.01666666666667	75851700\\
5.03333333333333	75831000\\
5.05	73628500\\
5.06666666666667	76537400\\
5.08333333333333	75003500\\
5.1	72574100\\
5.11666666666667	78540100\\
5.13333333333333	76861500\\
5.15	72662500\\
5.16666666666667	75922500\\
5.18333333333333	78926000\\
5.2	78241700\\
5.21666666666667	66773800\\
5.23333333333333	76780800\\
5.25	74783500\\
5.26666666666667	77444600\\
5.28333333333333	72619100\\
5.3	51325000\\
5.31666666666667	70003400\\
5.33333333333333	66685300\\
5.35	72833800\\
5.36666666666667	57439300\\
5.38333333333333	71268500\\
5.4	74390700\\
5.41666666666667	73496500\\
5.43333333333333	71305300\\
5.45	79963000\\
5.46666666666667	65782400\\
5.48333333333333	81465800\\
5.5	73149200\\
5.51666666666667	77903800\\
5.53333333333333	87690300\\
5.55	80596700\\
5.56666666666667	82531500\\
5.58333333333333	83912300\\
5.6	80264700\\
5.61666666666667	87397000\\
5.63333333333333	84319600\\
5.65	74860000\\
5.66666666666667	82533500\\
5.68333333333333	82437300\\
5.7	85346600\\
5.71666666666667	80883000\\
5.73333333333333	97280800\\
5.75	83856100\\
5.76666666666667	86539000\\
5.78333333333333	80254500\\
5.8	77381000\\
5.81666666666667	80278000\\
5.83333333333333	85028200\\
5.85	86059600\\
5.86666666666667	80125500\\
5.88333333333333	83277300\\
5.9	79691000\\
5.91666666666667	75527200\\
5.93333333333333	81163100\\
5.95	78466300\\
5.96666666666667	71490200\\
5.98333333333333	80379100\\
6	68492400\\
6.01666666666667	69703900\\
6.03333333333333	46068200\\
6.05	73263900\\
6.06666666666667	74640600\\
6.08333333333333	79749900\\
6.1	81265400\\
6.11666666666667	75355500\\
6.13333333333333	66963000\\
6.15	70679000\\
6.16666666666667	71606200\\
6.18333333333333	72931200\\
6.2	68602600\\
6.21666666666667	82097900\\
6.23333333333333	73224200\\
6.25	59993400\\
6.26666666666667	78752100\\
6.28333333333333	75892500\\
6.3	64198900\\
6.31666666666667	67389700\\
6.33333333333333	65143400\\
6.35	68732400\\
6.36666666666667	66178800\\
6.38333333333333	71610200\\
6.4	71649400\\
6.41666666666667	72863700\\
6.43333333333333	73661300\\
6.45	75545900\\
6.46666666666667	72397700\\
6.48333333333333	85951100\\
6.5	91186400\\
6.51666666666667	100463000\\
6.53333333333333	93980600\\
6.55	92796800\\
6.56666666666667	96390700\\
6.58333333333333	92437100\\
6.6	87635800\\
6.61666666666667	88188400\\
6.63333333333333	79564700\\
6.65	85650200\\
6.66666666666667	73283700\\
6.68333333333333	67769000\\
6.7	77089600\\
6.71666666666667	52102700\\
6.73333333333333	62320500\\
6.75	58328400\\
6.76666666666667	63746400\\
6.78333333333333	70751000\\
6.8	71651400\\
6.81666666666667	72038700\\
6.83333333333333	73481400\\
6.85	75145200\\
6.86666666666667	83067900\\
6.88333333333333	88071600\\
6.9	88050500\\
6.91666666666667	93774800\\
6.93333333333333	104099000\\
6.95	106257000\\
6.96666666666667	108699000\\
6.98333333333333	100831000\\
7	91675800\\
7.01666666666667	92877300\\
7.03333333333333	87944600\\
7.05	74522700\\
7.06666666666667	78754000\\
7.08333333333333	73038800\\
7.1	75292700\\
7.11666666666667	72723600\\
7.13333333333333	68865000\\
7.15	64304500\\
7.16666666666667	72802200\\
7.18333333333333	76239900\\
7.2	77767700\\
7.21666666666667	79466000\\
7.23333333333333	85285900\\
7.25	90454800\\
7.26666666666667	86319800\\
7.28333333333333	91682700\\
7.3	92486300\\
7.31666666666667	97391100\\
7.33333333333333	87414600\\
7.35	88416100\\
7.36666666666667	88238900\\
7.38333333333333	79111200\\
7.4	76001600\\
7.41666666666667	71711700\\
7.43333333333333	75151600\\
7.45	74917900\\
7.46666666666667	74602000\\
7.48333333333333	75989300\\
7.5	76033200\\
7.51666666666667	78423500\\
7.53333333333333	90345400\\
7.55	91740600\\
7.56666666666667	86416100\\
7.58333333333333	89878200\\
7.6	89009800\\
7.61666666666667	87341900\\
7.63333333333333	88892100\\
7.65	89469400\\
7.66666666666667	85076600\\
7.68333333333333	80608700\\
7.7	82568400\\
7.71666666666667	83014100\\
7.73333333333333	79450800\\
7.75	83798600\\
7.76666666666667	83284000\\
7.78333333333333	81247400\\
7.8	81171500\\
7.81666666666667	82056100\\
7.83333333333333	82796600\\
7.85	83898500\\
7.86666666666667	91758100\\
7.88333333333333	91995600\\
7.9	96787300\\
7.91666666666667	85826800\\
7.93333333333333	85775200\\
7.95	88904800\\
7.96666666666667	92653700\\
7.98333333333333	88818800\\
8	90861200\\
8.01666666666667	94720900\\
8.03333333333333	92224000\\
8.05	94074200\\
8.06666666666667	94436800\\
8.08333333333333	100127000\\
8.1	97697500\\
8.11666666666667	95409000\\
8.13333333333333	97194300\\
8.15	98201100\\
8.16666666666667	96698100\\
8.18333333333333	94859000\\
8.2	94638700\\
8.21666666666667	97420900\\
8.23333333333333	98239200\\
8.25	97576300\\
8.26666666666667	96896100\\
8.28333333333333	88536000\\
8.3	86275400\\
8.31666666666667	87635500\\
8.33333333333333	86536300\\
8.35	89904800\\
8.36666666666667	61620100\\
8.38333333333333	77830400\\
8.4	82838800\\
8.41666666666667	77497200\\
8.43333333333333	77797000\\
8.45	86645100\\
8.46666666666667	83875700\\
8.48333333333333	86561500\\
8.5	87748600\\
8.51666666666667	89113900\\
8.53333333333333	94388800\\
8.55	92316000\\
8.56666666666667	89867700\\
8.58333333333333	91083200\\
8.6	92461900\\
8.61666666666667	91932600\\
8.63333333333333	87780300\\
8.65	89646000\\
8.66666666666667	86987500\\
8.68333333333333	92869500\\
8.7	92888800\\
8.71666666666667	89153400\\
8.73333333333333	90595700\\
8.75	90505900\\
8.76666666666667	89312000\\
8.78333333333333	87218400\\
8.8	89588800\\
8.81666666666667	90124100\\
8.83333333333333	89621100\\
8.85	88476400\\
8.86666666666667	87648700\\
8.88333333333333	90759100\\
8.9	87022400\\
8.91666666666667	92393000\\
8.93333333333333	93139100\\
8.95	92306600\\
8.96666666666667	95186600\\
8.98333333333333	89581000\\
9	93420800\\
9.01666666666667	89131600\\
9.03333333333333	93045600\\
9.05	89358500\\
9.06666666666667	92222200\\
9.08333333333333	97376800\\
9.1	90331900\\
9.11666666666667	88165000\\
9.13333333333333	83752100\\
9.15	87778100\\
9.16666666666667	96866300\\
9.18333333333333	93555900\\
9.2	92964300\\
9.21666666666667	94543300\\
9.23333333333333	95613000\\
9.25	95411400\\
9.26666666666667	95407700\\
9.28333333333333	104604000\\
9.3	100462000\\
9.31666666666667	99366800\\
9.33333333333333	97237700\\
9.35	97224700\\
9.36666666666667	102014000\\
9.38333333333333	97955400\\
9.4	99609200\\
9.41666666666667	101560000\\
9.43333333333333	90884900\\
9.45	93132700\\
9.46666666666667	99062400\\
9.48333333333333	100665000\\
9.5	96651200\\
9.51666666666667	96233100\\
9.53333333333333	66625800\\
9.55	83660900\\
9.56666666666667	83505500\\
9.58333333333333	88077700\\
9.6	87911700\\
9.61666666666667	94622200\\
9.63333333333333	93440700\\
9.65	89782600\\
9.66666666666667	90343900\\
9.68333333333333	90216600\\
9.7	85976000\\
9.71666666666667	86958800\\
9.73333333333333	91992500\\
9.75	86207500\\
9.76666666666667	91325800\\
9.78333333333333	90840600\\
9.8	92733600\\
9.81666666666667	93751300\\
9.83333333333333	90839500\\
9.85	91749700\\
9.86666666666667	88486200\\
9.88333333333333	87842000\\
9.9	91833000\\
9.91666666666667	94209800\\
9.93333333333333	97148200\\
9.95	95104300\\
9.96666666666667	90631600\\
9.98333333333333	93775700\\
10	96772700\\
10.0166666666667	98110400\\
10.0333333333333	99545300\\
10.05	95851500\\
10.0666666666667	101930000\\
10.0833333333333	101369000\\
10.1	102581000\\
10.1166666666667	96720200\\
10.1333333333333	102131000\\
10.15	90826400\\
10.1666666666667	97489500\\
10.1833333333333	93194900\\
10.2	93134400\\
10.2166666666667	96202400\\
10.2333333333333	90208000\\
10.25	87711500\\
10.2666666666667	90823200\\
10.2833333333333	91707100\\
10.3	87514700\\
10.3166666666667	96406800\\
10.3333333333333	90769800\\
10.35	90328300\\
10.3666666666667	91493500\\
10.3833333333333	89489000\\
10.4	96563300\\
10.4166666666667	96934600\\
10.4333333333333	90678900\\
10.45	98460400\\
10.4666666666667	96438300\\
10.4833333333333	100041000\\
10.5	96536700\\
10.5166666666667	99460400\\
10.5333333333333	101691000\\
10.55	102858000\\
10.5666666666667	103650000\\
10.5833333333333	103277000\\
10.6	98510000\\
10.6166666666667	103382000\\
10.6333333333333	103366000\\
10.65	104986000\\
10.6666666666667	99598400\\
10.6833333333333	101251000\\
10.7	97530000\\
10.7166666666667	96660100\\
10.7333333333333	100250000\\
10.75	97608100\\
10.7666666666667	90745700\\
10.7833333333333	92873600\\
10.8	95004900\\
10.8166666666667	89603900\\
10.8333333333333	87513900\\
10.85	94134900\\
10.8666666666667	90926300\\
10.8833333333333	92071700\\
10.9	88672100\\
10.9166666666667	95692700\\
10.9333333333333	98328400\\
10.95	98442300\\
10.9666666666667	99253000\\
10.9833333333333	99193000\\
11	95705800\\
11.0166666666667	92916600\\
11.0333333333333	97998900\\
11.05	95141300\\
11.0666666666667	93843300\\
11.0833333333333	97230500\\
11.1	97923800\\
11.1166666666667	102873000\\
11.1333333333333	98925600\\
11.15	99649400\\
11.1666666666667	99907400\\
11.1833333333333	98734800\\
11.2	103690000\\
11.2166666666667	104712000\\
11.2333333333333	104767000\\
11.25	104222000\\
11.2666666666667	105318000\\
11.2833333333333	93257500\\
11.3	82824400\\
11.3166666666667	89442300\\
11.3333333333333	86902000\\
11.35	77142900\\
11.3666666666667	74424100\\
11.3833333333333	78453700\\
11.4	91223000\\
11.4166666666667	93077500\\
11.4333333333333	82165000\\
11.45	85359200\\
11.4666666666667	67598000\\
11.4833333333333	79118200\\
11.5	80688000\\
11.5166666666667	82307100\\
11.5333333333333	85214200\\
11.55	83274100\\
11.5666666666667	96180600\\
11.5833333333333	81823600\\
11.6	87068300\\
11.6166666666667	92876800\\
11.6333333333333	87524600\\
11.65	90330900\\
11.6666666666667	97262200\\
11.6833333333333	96757500\\
11.7	102373000\\
11.7166666666667	109475000\\
11.7333333333333	109550000\\
11.75	114880000\\
11.7666666666667	114064000\\
11.7833333333333	113962000\\
11.8	112604000\\
11.8166666666667	121342000\\
11.8333333333333	105299000\\
11.85	102063000\\
11.8666666666667	109226000\\
11.8833333333333	114230000\\
11.9	144451000\\
11.9166666666667	102717000\\
11.9333333333333	111015000\\
11.95	98941500\\
11.9666666666667	104355000\\
11.9833333333333	102373000\\
12	95770500\\
12.0166666666667	96489800\\
12.0333333333333	96615600\\
12.05	95014700\\
12.0666666666667	104663000\\
12.0833333333333	97344800\\
12.1	99604100\\
12.1166666666667	96992000\\
12.1333333333333	103048000\\
12.15	108755000\\
12.1666666666667	100492000\\
12.1833333333333	104000000\\
12.2	106711000\\
12.2166666666667	105148000\\
12.2333333333333	103985000\\
12.25	102570000\\
12.2666666666667	99675900\\
12.2833333333333	106308000\\
12.3	101047000\\
12.3166666666667	102319000\\
12.3333333333333	104363000\\
12.35	99352500\\
12.3666666666667	93399000\\
12.3833333333333	98355900\\
12.4	98072300\\
12.4166666666667	90241200\\
12.4333333333333	96518600\\
12.45	93906100\\
12.4666666666667	94981100\\
12.4833333333333	93931900\\
12.5	101212000\\
12.5166666666667	96802000\\
12.5333333333333	103795000\\
12.55	93215000\\
12.5666666666667	97052200\\
12.5833333333333	94202600\\
12.6	90811700\\
12.6166666666667	98405500\\
12.6333333333333	100175000\\
12.65	94037600\\
12.6666666666667	98825000\\
12.6833333333333	100866000\\
12.7	96157100\\
12.7166666666667	103566000\\
12.7333333333333	109205000\\
12.75	103187000\\
12.7666666666667	103160000\\
12.7833333333333	103435000\\
12.8	96828200\\
12.8166666666667	105367000\\
12.8333333333333	105728000\\
12.85	106079000\\
12.8666666666667	101842000\\
12.8833333333333	106073000\\
12.9	97677000\\
12.9166666666667	103505000\\
12.9333333333333	81995200\\
12.95	84539700\\
12.9666666666667	85898800\\
12.9833333333333	82241500\\
13	83448400\\
13.0166666666667	90859700\\
13.0333333333333	85375600\\
13.05	98458900\\
13.0666666666667	103071000\\
13.0833333333333	107149000\\
13.1	110944000\\
13.1166666666667	101217000\\
13.1333333333333	116139000\\
13.15	128693000\\
13.1666666666667	116006000\\
13.1833333333333	124947000\\
13.2	123277000\\
13.2166666666667	117030000\\
13.2333333333333	123290000\\
13.25	126011000\\
13.2666666666667	119040000\\
13.2833333333333	118963000\\
13.3	118245000\\
13.3166666666667	114675000\\
13.3333333333333	122274000\\
13.35	113480000\\
13.3666666666667	107281000\\
13.3833333333333	112398000\\
13.4	110191000\\
13.4166666666667	112856000\\
13.4333333333333	107206000\\
13.45	99887300\\
13.4666666666667	98986000\\
13.4833333333333	94210800\\
13.5	98959400\\
13.5166666666667	98101200\\
13.5333333333333	100095000\\
13.55	100424000\\
13.5666666666667	104153000\\
13.5833333333333	106413000\\
13.6	105187000\\
13.6166666666667	100295000\\
13.6333333333333	107748000\\
13.65	110665000\\
13.6666666666667	101203000\\
13.6833333333333	104007000\\
13.7	108722000\\
13.7166666666667	108333000\\
13.7333333333333	104512000\\
13.75	101983000\\
13.7666666666667	104165000\\
13.7833333333333	103947000\\
13.8	98081400\\
13.8166666666667	98200500\\
13.8333333333333	101395000\\
13.85	98181000\\
13.8666666666667	94758000\\
13.8833333333333	96965100\\
13.9	91664500\\
13.9166666666667	93370600\\
13.9333333333333	95224200\\
13.95	91988700\\
13.9666666666667	96857200\\
13.9833333333333	89053400\\
14	88119400\\
14.0166666666667	90026700\\
14.0333333333333	95619100\\
14.05	92819500\\
14.0666666666667	101484000\\
14.0833333333333	102548000\\
14.1	104258000\\
14.1166666666667	109803000\\
14.1333333333333	102875000\\
14.15	104921000\\
14.1666666666667	108778000\\
14.1833333333333	103655000\\
14.2	107845000\\
14.2166666666667	106027000\\
14.2333333333333	107689000\\
14.25	105844000\\
14.2666666666667	110547000\\
14.2833333333333	106884000\\
14.3	108007000\\
14.3166666666667	104059000\\
14.3333333333333	108205000\\
14.35	111211000\\
14.3666666666667	111650000\\
14.3833333333333	109238000\\
14.4	102208000\\
14.4166666666667	104644000\\
14.4333333333333	102632000\\
14.45	100640000\\
14.4666666666667	107279000\\
14.4833333333333	106984000\\
14.5	111608000\\
14.5166666666667	112003000\\
14.5333333333333	112684000\\
14.55	104673000\\
14.5666666666667	109791000\\
14.5833333333333	104675000\\
14.6	104105000\\
14.6166666666667	104500000\\
14.6333333333333	104956000\\
14.65	108169000\\
14.6666666666667	104307000\\
14.6833333333333	103728000\\
14.7	105948000\\
14.7166666666667	104437000\\
14.7333333333333	103538000\\
14.75	97018600\\
14.7666666666667	99223200\\
14.7833333333333	101109000\\
14.8	92693200\\
14.8166666666667	94836200\\
14.8333333333333	89107900\\
14.85	91913400\\
14.8666666666667	99084300\\
14.8833333333333	96613400\\
14.9	98767500\\
14.9166666666667	95051400\\
14.9333333333333	93903800\\
14.95	93499600\\
14.9666666666667	93793900\\
14.9833333333333	95246600\\
15	94528600\\
15.0166666666667	94103400\\
15.0333333333333	96624400\\
15.05	92605000\\
15.0666666666667	99136300\\
15.0833333333333	94145600\\
15.1	96076600\\
15.1166666666667	99256700\\
15.1333333333333	97089200\\
15.15	96854100\\
15.1666666666667	99393500\\
15.1833333333333	99550300\\
15.2	93012000\\
15.2166666666667	99980600\\
15.2333333333333	97066200\\
15.25	93798000\\
15.2666666666667	93385000\\
15.2833333333333	97826200\\
15.3	96033200\\
15.3166666666667	86232700\\
15.3333333333333	94548600\\
15.35	92806100\\
15.3666666666667	96307800\\
15.3833333333333	97550300\\
15.4	95600200\\
15.4166666666667	95933500\\
15.4333333333333	98407200\\
15.45	99528800\\
15.4666666666667	103650000\\
15.4833333333333	98617900\\
15.5	94888800\\
15.5166666666667	102362000\\
15.5333333333333	106705000\\
15.55	104045000\\
15.5666666666667	101396000\\
15.5833333333333	108306000\\
15.6	106217000\\
15.6166666666667	106014000\\
15.6333333333333	102657000\\
15.65	106859000\\
15.6666666666667	104013000\\
15.6833333333333	103664000\\
15.7	106893000\\
15.7166666666667	107264000\\
15.7333333333333	113045000\\
15.75	107609000\\
15.7666666666667	103862000\\
15.7833333333333	108391000\\
15.8	106467000\\
15.8166666666667	112852000\\
15.8333333333333	101291000\\
15.85	93392000\\
15.8666666666667	96625800\\
15.8833333333333	100786000\\
15.9	94124700\\
15.9166666666667	96475600\\
15.9333333333333	94079100\\
15.95	99456000\\
15.9666666666667	98586400\\
15.9833333333333	99377100\\
16	105891000\\
16.0166666666667	119224000\\
16.0333333333333	109837000\\
16.05	110030000\\
16.0666666666667	111340000\\
16.0833333333333	113944000\\
16.1	114867000\\
16.1166666666667	114367000\\
16.1333333333333	108054000\\
16.15	113471000\\
16.1666666666667	108474000\\
16.1833333333333	108588000\\
16.2	103068000\\
16.2166666666667	98579300\\
16.2333333333333	108835000\\
16.25	98211100\\
16.2666666666667	103325000\\
16.2833333333333	92893900\\
16.3	92543300\\
16.3166666666667	92051000\\
16.3333333333333	94170700\\
16.35	93697400\\
16.3666666666667	92720600\\
16.3833333333333	93183100\\
16.4	93974500\\
16.4166666666667	101186000\\
16.4333333333333	99706800\\
16.45	102294000\\
16.4666666666667	100605000\\
16.4833333333333	101863000\\
16.5	105934000\\
16.5166666666667	98608900\\
16.5333333333333	98901600\\
16.55	100336000\\
16.5666666666667	102372000\\
16.5833333333333	106142000\\
16.6	103399000\\
16.6166666666667	103120000\\
16.6333333333333	106266000\\
16.65	108049000\\
16.6666666666667	102522000\\
16.6833333333333	104251000\\
16.7	99266500\\
16.7166666666667	98484700\\
16.7333333333333	100432000\\
16.75	96833300\\
16.7666666666667	93587800\\
16.7833333333333	96200200\\
16.8	101294000\\
16.8166666666667	99848000\\
16.8333333333333	98735600\\
16.85	93937900\\
16.8666666666667	96467800\\
16.8833333333333	98668900\\
16.9	92600500\\
16.9166666666667	96567000\\
16.9333333333333	94844600\\
16.95	94774100\\
16.9666666666667	99866100\\
16.9833333333333	100757000\\
17	96296400\\
17.0166666666667	97838900\\
17.0333333333333	96129000\\
17.05	98939800\\
17.0666666666667	97197000\\
17.0833333333333	99604600\\
17.1	103350000\\
17.1166666666667	108637000\\
17.1333333333333	105839000\\
17.15	106629000\\
17.1666666666667	103841000\\
17.1833333333333	111550000\\
17.2	112249000\\
17.2166666666667	104394000\\
17.2333333333333	105087000\\
17.25	109960000\\
17.2666666666667	117865000\\
17.2833333333333	110080000\\
17.3	110387000\\
17.3166666666667	115070000\\
17.3333333333333	116651000\\
17.35	113528000\\
17.3666666666667	116030000\\
17.3833333333333	110831000\\
17.4	111803000\\
17.4166666666667	105748000\\
17.4333333333333	108785000\\
17.45	105698000\\
17.4666666666667	105788000\\
17.4833333333333	106742000\\
17.5	109006000\\
17.5166666666667	110006000\\
17.5333333333333	106435000\\
17.55	112475000\\
17.5666666666667	104635000\\
17.5833333333333	108270000\\
17.6	107797000\\
17.6166666666667	109903000\\
17.6333333333333	114086000\\
17.65	111420000\\
17.6666666666667	112479000\\
17.6833333333333	116467000\\
17.7	118879000\\
17.7166666666667	117455000\\
17.7333333333333	121542000\\
17.75	107224000\\
17.7666666666667	105236000\\
17.7833333333333	107816000\\
17.8	123366000\\
17.8166666666667	110749000\\
17.8333333333333	107387000\\
17.85	114188000\\
17.8666666666667	109136000\\
17.8833333333333	113026000\\
17.9	108401000\\
17.9166666666667	106913000\\
17.9333333333333	105018000\\
17.95	111847000\\
17.9666666666667	105965000\\
17.9833333333333	111896000\\
18	104741000\\
18.0166666666667	104298000\\
18.0333333333333	99433400\\
18.05	101364000\\
18.0666666666667	94437500\\
18.0833333333333	97362600\\
18.1	96693600\\
18.1166666666667	97596500\\
18.1333333333333	93660400\\
18.15	95043900\\
18.1666666666667	98489100\\
18.1833333333333	96079300\\
18.2	96933400\\
18.2166666666667	97101700\\
18.2333333333333	99235200\\
18.25	101399000\\
18.2666666666667	104918000\\
18.2833333333333	108057000\\
18.3	113886000\\
18.3166666666667	120625000\\
18.3333333333333	120582000\\
18.35	115326000\\
18.3666666666667	115943000\\
18.3833333333333	123925000\\
18.4	116221000\\
18.4166666666667	113656000\\
18.4333333333333	113077000\\
18.45	112636000\\
18.4666666666667	111928000\\
18.4833333333333	106028000\\
18.5	117675000\\
18.5166666666667	106446000\\
18.5333333333333	111420000\\
18.55	105744000\\
18.5666666666667	109313000\\
18.5833333333333	104265000\\
18.6	125434000\\
18.6166666666667	116357000\\
18.6333333333333	114057000\\
18.65	115188000\\
18.6666666666667	123062000\\
18.6833333333333	117571000\\
18.7	116184000\\
18.7166666666667	117569000\\
18.7333333333333	116212000\\
18.75	105020000\\
18.7666666666667	104465000\\
18.7833333333333	106833000\\
18.8	103207000\\
18.8166666666667	104810000\\
18.8333333333333	98053900\\
18.85	99810900\\
18.8666666666667	103876000\\
18.8833333333333	103395000\\
18.9	97572600\\
18.9166666666667	98382000\\
18.9333333333333	112134000\\
18.95	103216000\\
18.9666666666667	104264000\\
18.9833333333333	105303000\\
19	97026300\\
19.0166666666667	104791000\\
19.0333333333333	109289000\\
19.05	110200000\\
19.0666666666667	109804000\\
19.0833333333333	111203000\\
19.1	108410000\\
19.1166666666667	110086000\\
19.1333333333333	111160000\\
19.15	111682000\\
19.1666666666667	110538000\\
19.1833333333333	113542000\\
19.2	112936000\\
19.2166666666667	115329000\\
19.2333333333333	112482000\\
19.25	110442000\\
19.2666666666667	113948000\\
19.2833333333333	112221000\\
19.3	113132000\\
19.3166666666667	111634000\\
19.3333333333333	109251000\\
19.35	108094000\\
19.3666666666667	105141000\\
19.3833333333333	98909800\\
19.4	100268000\\
19.4166666666667	108281000\\
19.4333333333333	104241000\\
19.45	110642000\\
19.4666666666667	105988000\\
19.4833333333333	105555000\\
19.5	103957000\\
19.5166666666667	106702000\\
19.5333333333333	101325000\\
19.55	105375000\\
19.5666666666667	106871000\\
19.5833333333333	107972000\\
19.6	100284000\\
19.6166666666667	114205000\\
19.6333333333333	104075000\\
19.65	104562000\\
19.6666666666667	102854000\\
19.6833333333333	105317000\\
19.7	113579000\\
19.7166666666667	108256000\\
19.7333333333333	103695000\\
19.75	103470000\\
19.7666666666667	104224000\\
19.7833333333333	109951000\\
19.8	115977000\\
19.8166666666667	107137000\\
19.8333333333333	109069000\\
19.85	113567000\\
19.8666666666667	108587000\\
19.8833333333333	113497000\\
19.9	114206000\\
19.9166666666667	110243000\\
19.9333333333333	110843000\\
19.95	110876000\\
19.9666666666667	104568000\\
19.9833333333333	108125000\\
20	107262000\\
20.0166666666667	111324000\\
20.0333333333333	102215000\\
20.05	105770000\\
20.0666666666667	100785000\\
20.0833333333333	100175000\\
20.1	98269500\\
20.1166666666667	101355000\\
20.1333333333333	102528000\\
20.15	103490000\\
20.1666666666667	102069000\\
20.1833333333333	103283000\\
20.2	104036000\\
20.2166666666667	102504000\\
20.2333333333333	101660000\\
20.25	108422000\\
20.2666666666667	100955000\\
20.2833333333333	106246000\\
20.3	103199000\\
20.3166666666667	102131000\\
20.3333333333333	102680000\\
20.35	103558000\\
20.3666666666667	103539000\\
20.3833333333333	100537000\\
20.4	107342000\\
20.4166666666667	102699000\\
20.4333333333333	107530000\\
20.45	101254000\\
20.4666666666667	106156000\\
20.4833333333333	102846000\\
20.5	103153000\\
20.5166666666667	101528000\\
20.5333333333333	104276000\\
20.55	102415000\\
20.5666666666667	103651000\\
20.5833333333333	107213000\\
20.6	106021000\\
20.6166666666667	104902000\\
20.6333333333333	102294000\\
20.65	101845000\\
20.6666666666667	106121000\\
20.6833333333333	104600000\\
20.7	105840000\\
20.7166666666667	106202000\\
20.7333333333333	108940000\\
20.75	108328000\\
20.7666666666667	107979000\\
20.7833333333333	102988000\\
20.8	112636000\\
20.8166666666667	108959000\\
20.8333333333333	112626000\\
20.85	113749000\\
20.8666666666667	113652000\\
20.8833333333333	111493000\\
20.9	109132000\\
20.9166666666667	113093000\\
20.9333333333333	114217000\\
20.95	114263000\\
20.9666666666667	110446000\\
20.9833333333333	107328000\\
21	109434000\\
21.0166666666667	108755000\\
21.0333333333333	106701000\\
21.05	105144000\\
21.0666666666667	103174000\\
21.0833333333333	102482000\\
21.1	103231000\\
21.1166666666667	104038000\\
21.1333333333333	102513000\\
21.15	100385000\\
21.1666666666667	111099000\\
21.1833333333333	107489000\\
21.2	106097000\\
21.2166666666667	108268000\\
21.2333333333333	108104000\\
21.25	106274000\\
21.2666666666667	105120000\\
21.2833333333333	110319000\\
21.3	116306000\\
21.3166666666667	111107000\\
21.3333333333333	112045000\\
21.35	112748000\\
21.3666666666667	115873000\\
21.3833333333333	115801000\\
21.4	119140000\\
21.4166666666667	115601000\\
21.4333333333333	112550000\\
21.45	117164000\\
21.4666666666667	112989000\\
21.4833333333333	108617000\\
21.5	111566000\\
21.5166666666667	115896000\\
21.5333333333333	113441000\\
21.55	114353000\\
21.5666666666667	112830000\\
21.5833333333333	117184000\\
21.6	113461000\\
21.6166666666667	115770000\\
21.6333333333333	115280000\\
21.65	112518000\\
21.6666666666667	117703000\\
21.6833333333333	111910000\\
21.7	114997000\\
21.7166666666667	123436000\\
21.7333333333333	109023000\\
21.75	118611000\\
21.7666666666667	117589000\\
21.7833333333333	119393000\\
21.8	111481000\\
21.8166666666667	104682000\\
21.8333333333333	120258000\\
21.85	112638000\\
21.8666666666667	118395000\\
21.8833333333333	112545000\\
21.9	111401000\\
21.9166666666667	110649000\\
21.9333333333333	102414000\\
21.95	107627000\\
21.9666666666667	105553000\\
21.9833333333333	105148000\\
22	97397200\\
22.0166666666667	98170500\\
22.0333333333333	102430000\\
22.05	99407000\\
22.0666666666667	97763100\\
22.0833333333333	100859000\\
22.1	103157000\\
22.1166666666667	103547000\\
22.1333333333333	109445000\\
22.15	107696000\\
22.1666666666667	112177000\\
22.1833333333333	99474400\\
22.2	106804000\\
22.2166666666667	111424000\\
22.2333333333333	108137000\\
22.25	103701000\\
22.2666666666667	101923000\\
22.2833333333333	118369000\\
22.3	113749000\\
22.3166666666667	115299000\\
22.3333333333333	108320000\\
22.35	110758000\\
22.3666666666667	113082000\\
22.3833333333333	115025000\\
22.4	109157000\\
22.4166666666667	117102000\\
22.4333333333333	110382000\\
22.45	113149000\\
22.4666666666667	110724000\\
22.4833333333333	117397000\\
22.5	109852000\\
22.5166666666667	115042000\\
22.5333333333333	116093000\\
22.55	113066000\\
22.5666666666667	112175000\\
22.5833333333333	112303000\\
22.6	111209000\\
22.6166666666667	103458000\\
22.6333333333333	110197000\\
22.65	107300000\\
22.6666666666667	116989000\\
22.6833333333333	114560000\\
22.7	111455000\\
22.7166666666667	111793000\\
22.7333333333333	112386000\\
22.75	113083000\\
22.7666666666667	111993000\\
22.7833333333333	110919000\\
22.8	107604000\\
22.8166666666667	114953000\\
22.8333333333333	110478000\\
22.85	107472000\\
22.8666666666667	109694000\\
22.8833333333333	112807000\\
22.9	116483000\\
22.9166666666667	112980000\\
22.9333333333333	109857000\\
22.95	113732000\\
22.9666666666667	113008000\\
22.9833333333333	114202000\\
23	111580000\\
23.0166666666667	114293000\\
23.0333333333333	108623000\\
23.05	109629000\\
23.0666666666667	102867000\\
23.0833333333333	108568000\\
23.1	105067000\\
23.1166666666667	113077000\\
23.1333333333333	115749000\\
23.15	114748000\\
23.1666666666667	114519000\\
23.1833333333333	115080000\\
23.2	116135000\\
23.2166666666667	115629000\\
23.2333333333333	111923000\\
23.25	114524000\\
23.2666666666667	119695000\\
23.2833333333333	117372000\\
23.3	115618000\\
23.3166666666667	112086000\\
23.3333333333333	114134000\\
23.35	116251000\\
23.3666666666667	113376000\\
23.3833333333333	115874000\\
23.4	110883000\\
23.4166666666667	114095000\\
23.4333333333333	111294000\\
23.45	113201000\\
23.4666666666667	109315000\\
23.4833333333333	106463000\\
23.5	107361000\\
23.5166666666667	108404000\\
23.5333333333333	101499000\\
23.55	103972000\\
23.5666666666667	109009000\\
23.5833333333333	109249000\\
23.6	102852000\\
23.6166666666667	111394000\\
23.6333333333333	111429000\\
23.65	102305000\\
23.6666666666667	106478000\\
23.6833333333333	107533000\\
23.7	106125000\\
23.7166666666667	110588000\\
23.7333333333333	114552000\\
23.75	113340000\\
23.7666666666667	109399000\\
23.7833333333333	111063000\\
23.8	117424000\\
23.8166666666667	115720000\\
23.8333333333333	115581000\\
23.85	109886000\\
23.8666666666667	109326000\\
23.8833333333333	111185000\\
23.9	110035000\\
23.9166666666667	114025000\\
23.9333333333333	109484000\\
23.95	107442000\\
23.9666666666667	108276000\\
23.9833333333333	107885000\\
24	107664000\\
24.0166666666667	108669000\\
24.0333333333333	106611000\\
24.05	108983000\\
24.0666666666667	117757000\\
24.0833333333333	116324000\\
24.1	114606000\\
24.1166666666667	118083000\\
24.1333333333333	114963000\\
24.15	116096000\\
24.1666666666667	112522000\\
24.1833333333333	115190000\\
24.2	112570000\\
24.2166666666667	108677000\\
24.2333333333333	113551000\\
24.25	111519000\\
24.2666666666667	110417000\\
24.2833333333333	110241000\\
24.3	108358000\\
24.3166666666667	105135000\\
24.3333333333333	107473000\\
24.35	108182000\\
24.3666666666667	108366000\\
24.3833333333333	110631000\\
24.4	106281000\\
24.4166666666667	113399000\\
24.4333333333333	109333000\\
24.45	114380000\\
24.4666666666667	111209000\\
24.4833333333333	117978000\\
24.5	115104000\\
24.5166666666667	114862000\\
24.5333333333333	117254000\\
24.55	115845000\\
24.5666666666667	118064000\\
24.5833333333333	112862000\\
24.6	117930000\\
24.6166666666667	114894000\\
24.6333333333333	116934000\\
24.65	118432000\\
24.6666666666667	111376000\\
24.6833333333333	116159000\\
24.7	115506000\\
24.7166666666667	111622000\\
24.7333333333333	116850000\\
24.75	114172000\\
24.7666666666667	106811000\\
24.7833333333333	111938000\\
24.8	106867000\\
24.8166666666667	108916000\\
24.8333333333333	111478000\\
24.85	110272000\\
24.8666666666667	113601000\\
24.8833333333333	106204000\\
24.9	108369000\\
24.9166666666667	108425000\\
24.9333333333333	107771000\\
24.95	108546000\\
24.9666666666667	106130000\\
24.9833333333333	103686000\\
25	110222000\\
25.0166666666667	113381000\\
25.0333333333333	119686000\\
25.05	117991000\\
25.0666666666667	113656000\\
25.0833333333333	116100000\\
25.1	114905000\\
25.1166666666667	117825000\\
25.1333333333333	113620000\\
25.15	117540000\\
25.1666666666667	119199000\\
25.1833333333333	118008000\\
25.2	116264000\\
25.2166666666667	118475000\\
25.2333333333333	112140000\\
25.25	112053000\\
25.2666666666667	111040000\\
25.2833333333333	107038000\\
25.3	108177000\\
25.3166666666667	106052000\\
25.3333333333333	105428000\\
25.35	106425000\\
25.3666666666667	106453000\\
25.3833333333333	109981000\\
25.4	109358000\\
25.4166666666667	104103000\\
25.4333333333333	112493000\\
25.45	104997000\\
25.4666666666667	111374000\\
25.4833333333333	111253000\\
25.5	113320000\\
25.5166666666667	114827000\\
25.5333333333333	114333000\\
25.55	116940000\\
25.5666666666667	112418000\\
25.5833333333333	114249000\\
25.6	117850000\\
25.6166666666667	113451000\\
25.6333333333333	117986000\\
25.65	120200000\\
25.6666666666667	118174000\\
25.6833333333333	105524000\\
25.7	112459000\\
25.7166666666667	105591000\\
25.7333333333333	110137000\\
25.75	115741000\\
25.7666666666667	104056000\\
25.7833333333333	103277000\\
25.8	105577000\\
25.8166666666667	113641000\\
25.8333333333333	112759000\\
25.85	114495000\\
25.8666666666667	107514000\\
25.8833333333333	110473000\\
25.9	111233000\\
25.9166666666667	109733000\\
25.9333333333333	109505000\\
25.95	109657000\\
25.9666666666667	114274000\\
25.9833333333333	106609000\\
26	112416000\\
26.0166666666667	105170000\\
26.0333333333333	107086000\\
26.05	112497000\\
26.0666666666667	111104000\\
26.0833333333333	117361000\\
26.1	110337000\\
26.1166666666667	112035000\\
26.1333333333333	111929000\\
26.15	112761000\\
26.1666666666667	113014000\\
26.1833333333333	109243000\\
26.2	107094000\\
26.2166666666667	109816000\\
26.2333333333333	106317000\\
26.25	110333000\\
26.2666666666667	107503000\\
26.2833333333333	108696000\\
26.3	106528000\\
26.3166666666667	106883000\\
26.3333333333333	107100000\\
26.35	111136000\\
26.3666666666667	113151000\\
26.3833333333333	114816000\\
26.4	111822000\\
26.4166666666667	114482000\\
26.4333333333333	117701000\\
26.45	120373000\\
26.4666666666667	117467000\\
26.4833333333333	115501000\\
26.5	120829000\\
26.5166666666667	118262000\\
26.5333333333333	120599000\\
26.55	121609000\\
26.5666666666667	119759000\\
26.5833333333333	118601000\\
26.6	115695000\\
26.6166666666667	118163000\\
26.6333333333333	119706000\\
26.65	122213000\\
26.6666666666667	119631000\\
26.6833333333333	117136000\\
26.7	113460000\\
26.7166666666667	117822000\\
26.7333333333333	116652000\\
26.75	118505000\\
26.7666666666667	115594000\\
26.7833333333333	114209000\\
26.8	118777000\\
26.8166666666667	114052000\\
26.8333333333333	114890000\\
26.85	114759000\\
26.8666666666667	121999000\\
26.8833333333333	119235000\\
26.9	120324000\\
26.9166666666667	118163000\\
26.9333333333333	120802000\\
26.95	119265000\\
26.9666666666667	118396000\\
26.9833333333333	117082000\\
27	114789000\\
27.0166666666667	121941000\\
27.0333333333333	119015000\\
27.05	120705000\\
27.0666666666667	119166000\\
27.0833333333333	117673000\\
27.1	118314000\\
27.1166666666667	116870000\\
27.1333333333333	114307000\\
27.15	116933000\\
27.1666666666667	117370000\\
27.1833333333333	116495000\\
27.2	116070000\\
27.2166666666667	118546000\\
27.2333333333333	109413000\\
27.25	114950000\\
27.2666666666667	118255000\\
27.2833333333333	112055000\\
27.3	107654000\\
27.3166666666667	115318000\\
27.3333333333333	108967000\\
27.35	114431000\\
27.3666666666667	112183000\\
27.3833333333333	113768000\\
27.4	115273000\\
27.4166666666667	114780000\\
27.4333333333333	109193000\\
27.45	114249000\\
27.4666666666667	108235000\\
27.4833333333333	110917000\\
27.5	111755000\\
27.5166666666667	111564000\\
27.5333333333333	114091000\\
27.55	112256000\\
27.5666666666667	113105000\\
27.5833333333333	112630000\\
27.6	115981000\\
27.6166666666667	112398000\\
27.6333333333333	117015000\\
27.65	113667000\\
27.6666666666667	116324000\\
27.6833333333333	117252000\\
27.7	115175000\\
27.7166666666667	119605000\\
27.7333333333333	120820000\\
27.75	115640000\\
27.7666666666667	114552000\\
27.7833333333333	119413000\\
27.8	120558000\\
27.8166666666667	122641000\\
27.8333333333333	116614000\\
27.85	119787000\\
27.8666666666667	121276000\\
27.8833333333333	123639000\\
27.9	118798000\\
27.9166666666667	121594000\\
27.9333333333333	115894000\\
27.95	121608000\\
27.9666666666667	119841000\\
27.9833333333333	118200000\\
28	116493000\\
28.0166666666667	119947000\\
28.0333333333333	118723000\\
28.05	117493000\\
28.0666666666667	119931000\\
28.0833333333333	121093000\\
28.1	120418000\\
28.1166666666667	123089000\\
28.1333333333333	117183000\\
28.15	120102000\\
28.1666666666667	117854000\\
28.1833333333333	117074000\\
28.2	121788000\\
28.2166666666667	117063000\\
28.2333333333333	116050000\\
28.25	116815000\\
28.2666666666667	117838000\\
28.2833333333333	118152000\\
28.3	119886000\\
28.3166666666667	119924000\\
28.3333333333333	120006000\\
28.35	116298000\\
28.3666666666667	118960000\\
28.3833333333333	123470000\\
28.4	123462000\\
28.4166666666667	117433000\\
28.4333333333333	120264000\\
28.45	116530000\\
28.4666666666667	117827000\\
28.4833333333333	118868000\\
28.5	114203000\\
28.5166666666667	112331000\\
28.5333333333333	111866000\\
28.55	111422000\\
28.5666666666667	111033000\\
28.5833333333333	105869000\\
28.6	108197000\\
28.6166666666667	107877000\\
28.6333333333333	109109000\\
28.65	106995000\\
28.6666666666667	131301000\\
28.6833333333333	116658000\\
28.7	115242000\\
28.7166666666667	116161000\\
28.7333333333333	117972000\\
28.75	111192000\\
28.7666666666667	115610000\\
28.7833333333333	113103000\\
28.8	115746000\\
28.8166666666667	115321000\\
28.8333333333333	115137000\\
28.85	115838000\\
28.8666666666667	115566000\\
28.8833333333333	126595000\\
28.9	119610000\\
28.9166666666667	115907000\\
28.9333333333333	120593000\\
28.95	130788000\\
28.9666666666667	115928000\\
28.9833333333333	115271000\\
29	112045000\\
29.0166666666667	111374000\\
29.0333333333333	108064000\\
29.05	114343000\\
29.0666666666667	112087000\\
29.0833333333333	111906000\\
29.1	107392000\\
29.1166666666667	111163000\\
29.1333333333333	110823000\\
29.15	115455000\\
29.1666666666667	112554000\\
29.1833333333333	117004000\\
29.2	115124000\\
29.2166666666667	116479000\\
29.2333333333333	114997000\\
29.25	118473000\\
29.2666666666667	118309000\\
29.2833333333333	113033000\\
29.3	114012000\\
29.3166666666667	117991000\\
29.3333333333333	127010000\\
29.35	118613000\\
29.3666666666667	116580000\\
29.3833333333333	114513000\\
29.4	124548000\\
29.4166666666667	117071000\\
29.4333333333333	119254000\\
29.45	117354000\\
29.4666666666667	116334000\\
29.4833333333333	118632000\\
29.5	116974000\\
29.5166666666667	113562000\\
29.5333333333333	116463000\\
29.55	122629000\\
29.5666666666667	122048000\\
29.5833333333333	112353000\\
29.6	111871000\\
29.6166666666667	114940000\\
29.6333333333333	120266000\\
29.65	113942000\\
29.6666666666667	115713000\\
29.6833333333333	103019000\\
29.7	109196000\\
29.7166666666667	110864000\\
29.7333333333333	110276000\\
29.75	115782000\\
29.7666666666667	111040000\\
29.7833333333333	109092000\\
29.8	113563000\\
29.8166666666667	116286000\\
29.8333333333333	112407000\\
29.85	121672000\\
29.8666666666667	121344000\\
29.8833333333333	118775000\\
29.9	120093000\\
29.9166666666667	121238000\\
29.9333333333333	120628000\\
29.95	119667000\\
29.9666666666667	117171000\\
29.9833333333333	119173000\\
};

\addplot [color=mycolor2]
  table[row sep=crcr]{%
0	1974840\\
0.0166666666666667	1996580\\
0.0333333333333333	2009420\\
0.05	2022060\\
0.0666666666666667	2035680\\
0.0833333333333333	2044870\\
0.1	2053870\\
0.116666666666667	2062920\\
0.133333333333333	2069650\\
0.15	2075450\\
0.166666666666667	2082690\\
0.183333333333333	2091220\\
0.2	2101090\\
0.216666666666667	2107290\\
0.233333333333333	2112730\\
0.25	2120010\\
0.266666666666667	2126500\\
0.283333333333333	2131820\\
0.3	2137030\\
0.316666666666667	2144240\\
0.333333333333333	2152160\\
0.35	2197810\\
0.366666666666667	2210020\\
0.383333333333333	2221560\\
0.4	2231800\\
0.416666666666667	2239500\\
0.433333333333333	2246090\\
0.45	2252190\\
0.466666666666667	2258310\\
0.483333333333333	2265040\\
0.5	2271360\\
0.516666666666667	2276640\\
0.533333333333333	2280310\\
0.55	2283930\\
0.566666666666667	2283250\\
0.583333333333333	2284600\\
0.6	2283480\\
0.616666666666667	2284770\\
0.633333333333333	2285270\\
0.65	2288990\\
0.666666666666667	2291650\\
0.683333333333333	2293780\\
0.7	2298850\\
0.716666666666667	2301930\\
0.733333333333333	2306500\\
0.75	2311490\\
0.766666666666667	2318260\\
0.783333333333333	2321480\\
0.8	2322920\\
0.816666666666667	2323750\\
0.833333333333333	2326300\\
0.85	2327680\\
0.866666666666667	2329950\\
0.883333333333333	2333310\\
0.9	2337340\\
0.916666666666667	2340180\\
0.933333333333333	2344270\\
0.95	2347900\\
0.966666666666667	2352670\\
0.983333333333333	2356420\\
1	2358130\\
1.01666666666667	2360370\\
1.03333333333333	2362720\\
1.05	2365870\\
1.06666666666667	2369370\\
1.08333333333333	2370450\\
1.1	2372090\\
1.11666666666667	2374810\\
1.13333333333333	2377250\\
1.15	2378080\\
1.16666666666667	2379290\\
1.18333333333333	2380840\\
1.2	2384220\\
1.21666666666667	2387920\\
1.23333333333333	2391080\\
1.25	2395620\\
1.26666666666667	2400050\\
1.28333333333333	2403080\\
1.3	2404090\\
1.31666666666667	2404860\\
1.33333333333333	2406080\\
1.35	2408700\\
1.36666666666667	2412160\\
1.38333333333333	2415830\\
1.4	2419720\\
1.41666666666667	2422810\\
1.43333333333333	2424540\\
1.45	2426690\\
1.46666666666667	2428310\\
1.48333333333333	2429450\\
1.5	2430250\\
1.51666666666667	2432150\\
1.53333333333333	2434440\\
1.55	2437700\\
1.56666666666667	2439280\\
1.58333333333333	2417130\\
1.6	2392380\\
1.61666666666667	2380340\\
1.63333333333333	2377760\\
1.65	2318020\\
1.66666666666667	2285240\\
1.68333333333333	2283540\\
1.7	2296330\\
1.71666666666667	2305550\\
1.73333333333333	2308060\\
1.75	2313130\\
1.76666666666667	2318130\\
1.78333333333333	2321340\\
1.8	2324950\\
1.81666666666667	2329360\\
1.83333333333333	2334510\\
1.85	2339150\\
1.86666666666667	2343390\\
1.88333333333333	2348440\\
1.9	2354230\\
1.91666666666667	2359870\\
1.93333333333333	2363140\\
1.95	2368130\\
1.96666666666667	2373880\\
1.98333333333333	2377970\\
2	2381610\\
2.01666666666667	2385460\\
2.03333333333333	2389180\\
2.05	2392740\\
2.06666666666667	2399430\\
2.08333333333333	2407430\\
2.1	2413310\\
2.11666666666667	2420150\\
2.13333333333333	2428070\\
2.15	2433520\\
2.16666666666667	2437820\\
2.18333333333333	2440390\\
2.2	2447370\\
2.21666666666667	2452440\\
2.23333333333333	2456300\\
2.25	2460880\\
2.26666666666667	2465900\\
2.28333333333333	2469260\\
2.3	2471320\\
2.31666666666667	2472550\\
2.33333333333333	2473190\\
2.35	2474750\\
2.36666666666667	2478110\\
2.38333333333333	2482290\\
2.4	2487780\\
2.41666666666667	2492460\\
2.43333333333333	2498930\\
2.45	2505810\\
2.46666666666667	2510600\\
2.48333333333333	2512210\\
2.5	2518800\\
2.51666666666667	2524620\\
2.53333333333333	2529590\\
2.55	2536590\\
2.56666666666667	2541030\\
2.58333333333333	2550330\\
2.6	2558650\\
2.61666666666667	2566930\\
2.63333333333333	2576000\\
2.65	2585120\\
2.66666666666667	2599400\\
2.68333333333333	2604880\\
2.7	2608730\\
2.71666666666667	2615900\\
2.73333333333333	2626900\\
2.75	2633820\\
2.76666666666667	2642130\\
2.78333333333333	2650210\\
2.8	2656850\\
2.81666666666667	2666740\\
2.83333333333333	2677400\\
2.85	2685140\\
2.86666666666667	2691730\\
2.88333333333333	2694030\\
2.9	2704300\\
2.91666666666667	2708350\\
2.93333333333333	2713690\\
2.95	2716770\\
2.96666666666667	2721420\\
2.98333333333333	2722700\\
3	2727040\\
3.01666666666667	2732600\\
3.03333333333333	2735650\\
3.05	2737950\\
3.06666666666667	2742160\\
3.08333333333333	2749560\\
3.1	2758430\\
3.11666666666667	2765130\\
3.13333333333333	2773760\\
3.15	2778970\\
3.16666666666667	2786970\\
3.18333333333333	2801110\\
3.2	2807450\\
3.21666666666667	2813620\\
3.23333333333333	2823680\\
3.25	2838860\\
3.26666666666667	2853660\\
3.28333333333333	2863990\\
3.3	2871050\\
3.31666666666667	2886640\\
3.33333333333333	2905290\\
3.35	2919180\\
3.36666666666667	2924430\\
3.38333333333333	2929330\\
3.4	2944780\\
3.41666666666667	2954450\\
3.43333333333333	2961610\\
3.45	2961960\\
3.46666666666667	2970240\\
3.48333333333333	2973730\\
3.5	2980570\\
3.51666666666667	2988720\\
3.53333333333333	2991230\\
3.55	3001460\\
3.56666666666667	3014950\\
3.58333333333333	3021550\\
3.6	3027150\\
3.61666666666667	3052850\\
3.63333333333333	3087640\\
3.65	3101160\\
3.66666666666667	3127510\\
3.68333333333333	3139910\\
3.7	3161690\\
3.71666666666667	3166240\\
3.73333333333333	3174310\\
3.75	3186660\\
3.76666666666667	3220390\\
3.78333333333333	3261810\\
3.8	3302530\\
3.81666666666667	3331100\\
3.83333333333333	3355730\\
3.85	3369190\\
3.86666666666667	3377560\\
3.88333333333333	3379650\\
3.9	3386640\\
3.91666666666667	3398640\\
3.93333333333333	3411540\\
3.95	3423610\\
3.96666666666667	3446780\\
3.98333333333333	3470140\\
4	3474100\\
4.01666666666667	3480100\\
4.03333333333333	3477800\\
4.05	3475380\\
4.06666666666667	3475300\\
4.08333333333333	3478260\\
4.1	3481560\\
4.11666666666667	3483580\\
4.13333333333333	3483400\\
4.15	3478430\\
4.16666666666667	3469000\\
4.18333333333333	3476300\\
4.2	3517340\\
4.21666666666667	3560820\\
4.23333333333333	3565340\\
4.25	3560020\\
4.26666666666667	3572810\\
4.28333333333333	3595850\\
4.3	3603220\\
4.31666666666667	3592390\\
4.33333333333333	3584180\\
4.35	3587690\\
4.36666666666667	3586840\\
4.38333333333333	3588750\\
4.4	3595480\\
4.41666666666667	3610610\\
4.43333333333333	3645430\\
4.45	3671560\\
4.46666666666667	3686630\\
4.48333333333333	3705470\\
4.5	3696090\\
4.51666666666667	3705290\\
4.53333333333333	3719810\\
4.55	3720640\\
4.56666666666667	3711100\\
4.58333333333333	3709830\\
4.6	3732020\\
4.61666666666667	3729910\\
4.63333333333333	3731820\\
4.65	3730650\\
4.66666666666667	3751360\\
4.68333333333333	3745790\\
4.7	3752120\\
4.71666666666667	3752150\\
4.73333333333333	3760100\\
4.75	3751920\\
4.76666666666667	3753680\\
4.78333333333333	3763330\\
4.8	3763930\\
4.81666666666667	3773210\\
4.83333333333333	3785020\\
4.85	3787180\\
4.86666666666667	3797730\\
4.88333333333333	3825760\\
4.9	3858550\\
4.91666666666667	3860590\\
4.93333333333333	3852020\\
4.95	3835270\\
4.96666666666667	3824950\\
4.98333333333333	3821050\\
5	3840220\\
5.01666666666667	3842250\\
5.03333333333333	3842980\\
5.05	3835630\\
5.06666666666667	3806120\\
5.08333333333333	3802640\\
5.1	3809050\\
5.11666666666667	3811430\\
5.13333333333333	3799640\\
5.15	3809010\\
5.16666666666667	3817310\\
5.18333333333333	3818700\\
5.2	3818570\\
5.21666666666667	3813410\\
5.23333333333333	3815200\\
5.25	3809900\\
5.26666666666667	3804390\\
5.28333333333333	3810010\\
5.3	3818000\\
5.31666666666667	3824760\\
5.33333333333333	3827740\\
5.35	3829540\\
5.36666666666667	3841480\\
5.38333333333333	3844870\\
5.4	3853530\\
5.41666666666667	3865120\\
5.43333333333333	3892690\\
5.45	3883090\\
5.46666666666667	3881860\\
5.48333333333333	3874010\\
5.5	3883970\\
5.51666666666667	3883030\\
5.53333333333333	3791750\\
5.55	3783540\\
5.56666666666667	3782290\\
5.58333333333333	3779930\\
5.6	3784110\\
5.61666666666667	3789460\\
5.63333333333333	3790580\\
5.65	3794810\\
5.66666666666667	3796610\\
5.68333333333333	3800650\\
5.7	3806220\\
5.71666666666667	3802730\\
5.73333333333333	3798790\\
5.75	3798640\\
5.76666666666667	3799390\\
5.78333333333333	3801820\\
5.8	3801430\\
5.81666666666667	3800310\\
5.83333333333333	3801570\\
5.85	3799730\\
5.86666666666667	3799630\\
5.88333333333333	3799400\\
5.9	3802550\\
5.91666666666667	3807050\\
5.93333333333333	3803390\\
5.95	3804940\\
5.96666666666667	3817400\\
5.98333333333333	3824370\\
6	3831130\\
6.01666666666667	3836010\\
6.03333333333333	3835270\\
6.05	3834990\\
6.06666666666667	3835300\\
6.08333333333333	3832940\\
6.1	3831540\\
6.11666666666667	3834800\\
6.13333333333333	3838700\\
6.15	3847560\\
6.16666666666667	3858860\\
6.18333333333333	3862810\\
6.2	3862620\\
6.21666666666667	3864850\\
6.23333333333333	3866920\\
6.25	3871370\\
6.26666666666667	3875590\\
6.28333333333333	3880950\\
6.3	3879260\\
6.31666666666667	3884410\\
6.33333333333333	3884240\\
6.35	3883760\\
6.36666666666667	3890370\\
6.38333333333333	3889060\\
6.4	3885310\\
6.41666666666667	3884900\\
6.43333333333333	3890010\\
6.45	3894080\\
6.46666666666667	3895470\\
6.48333333333333	3891950\\
6.5	3898670\\
6.51666666666667	3898420\\
6.53333333333333	3902240\\
6.55	3908350\\
6.56666666666667	3911240\\
6.58333333333333	3911720\\
6.6	3914660\\
6.61666666666667	3921720\\
6.63333333333333	3930630\\
6.65	3933190\\
6.66666666666667	3934220\\
6.68333333333333	3931330\\
6.7	3933140\\
6.71666666666667	3937010\\
6.73333333333333	3939730\\
6.75	3941900\\
6.76666666666667	3944720\\
6.78333333333333	3942720\\
6.8	3945950\\
6.81666666666667	3948690\\
6.83333333333333	3955510\\
6.85	3958320\\
6.86666666666667	3954970\\
6.88333333333333	3956000\\
6.9	3959480\\
6.91666666666667	3962300\\
6.93333333333333	3964530\\
6.95	3964800\\
6.96666666666667	3969080\\
6.98333333333333	3969780\\
7	3971360\\
7.01666666666667	3973250\\
7.03333333333333	3969340\\
7.05	3966250\\
7.06666666666667	3966550\\
7.08333333333333	3967090\\
7.1	3969240\\
7.11666666666667	3966950\\
7.13333333333333	3966880\\
7.15	3965700\\
7.16666666666667	3966750\\
7.18333333333333	3972140\\
7.2	3969310\\
7.21666666666667	3973560\\
7.23333333333333	3982110\\
7.25	3989680\\
7.26666666666667	3986850\\
7.28333333333333	3995000\\
7.3	3996510\\
7.31666666666667	3993150\\
7.33333333333333	3992860\\
7.35	3994590\\
7.36666666666667	3996740\\
7.38333333333333	3997470\\
7.4	3997010\\
7.41666666666667	3996510\\
7.43333333333333	3997390\\
7.45	4000100\\
7.46666666666667	4003750\\
7.48333333333333	4007590\\
7.5	4014350\\
7.51666666666667	4016880\\
7.53333333333333	4012990\\
7.55	4012300\\
7.56666666666667	4017510\\
7.58333333333333	4021270\\
7.6	4025540\\
7.61666666666667	4026990\\
7.63333333333333	4025820\\
7.65	4026440\\
7.66666666666667	4028880\\
7.68333333333333	4028340\\
7.7	4030490\\
7.71666666666667	4035770\\
7.73333333333333	4038660\\
7.75	3952220\\
7.76666666666667	3943590\\
7.78333333333333	3941880\\
7.8	3939830\\
7.81666666666667	3941180\\
7.83333333333333	3942090\\
7.85	3942390\\
7.86666666666667	3943210\\
7.88333333333333	3944150\\
7.9	3946220\\
7.91666666666667	3948080\\
7.93333333333333	3948450\\
7.95	3949350\\
7.96666666666667	3952200\\
7.98333333333333	3954470\\
8	3952390\\
8.01666666666667	3952230\\
8.03333333333333	3956920\\
8.05	3961250\\
8.06666666666667	3965200\\
8.08333333333333	3966440\\
8.1	3967370\\
8.11666666666667	3968820\\
8.13333333333333	3969880\\
8.15	3970100\\
8.16666666666667	3971120\\
8.18333333333333	3972360\\
8.2	3973990\\
8.21666666666667	3974600\\
8.23333333333333	3973710\\
8.25	3974130\\
8.26666666666667	3974510\\
8.28333333333333	3976000\\
8.3	3977580\\
8.31666666666667	3980080\\
8.33333333333333	3983440\\
8.35	3984760\\
8.36666666666667	3986510\\
8.38333333333333	3987080\\
8.4	3988120\\
8.41666666666667	3989740\\
8.43333333333333	3990610\\
8.45	3994630\\
8.46666666666667	3997760\\
8.48333333333333	3999960\\
8.5	4000230\\
8.51666666666667	4003440\\
8.53333333333333	4004760\\
8.55	4006170\\
8.56666666666667	4006760\\
8.58333333333333	4006980\\
8.6	4007790\\
8.61666666666667	4009560\\
8.63333333333333	4011720\\
8.65	4011310\\
8.66666666666667	4011780\\
8.68333333333333	4011060\\
8.7	4011170\\
8.71666666666667	4013600\\
8.73333333333333	4015650\\
8.75	4018220\\
8.76666666666667	4020470\\
8.78333333333333	4023400\\
8.8	4025130\\
8.81666666666667	4026620\\
8.83333333333333	4027430\\
8.85	4028050\\
8.86666666666667	4030580\\
8.88333333333333	4032460\\
8.9	4034600\\
8.91666666666667	4035190\\
8.93333333333333	4035910\\
8.95	4035910\\
8.96666666666667	4036810\\
8.98333333333333	4035960\\
9	4035770\\
9.01666666666667	4035980\\
9.03333333333333	4036750\\
9.05	4038700\\
9.06666666666667	4039940\\
9.08333333333333	4040870\\
9.1	4042230\\
9.11666666666667	4043020\\
9.13333333333333	4043740\\
9.15	4044260\\
9.16666666666667	4045050\\
9.18333333333333	4045980\\
9.2	4047350\\
9.21666666666667	4049180\\
9.23333333333333	4050260\\
9.25	4050420\\
9.26666666666667	4051620\\
9.28333333333333	4052830\\
9.3	4054310\\
9.31666666666667	4055680\\
9.33333333333333	4056850\\
9.35	4056980\\
9.36666666666667	4057860\\
9.38333333333333	4058700\\
9.4	4058450\\
9.41666666666667	4058230\\
9.43333333333333	4059590\\
9.45	4061350\\
9.46666666666667	4062930\\
9.48333333333333	4063920\\
9.5	4064990\\
9.51666666666667	4066350\\
9.53333333333333	4067820\\
9.55	4069170\\
9.56666666666667	4070960\\
9.58333333333333	4071830\\
9.6	4073190\\
9.61666666666667	4074250\\
9.63333333333333	4074980\\
9.65	4076180\\
9.66666666666667	4077170\\
9.68333333333333	4078370\\
9.7	4079990\\
9.71666666666667	4080690\\
9.73333333333333	4081660\\
9.75	4082840\\
9.76666666666667	4083860\\
9.78333333333333	4084070\\
9.8	4085640\\
9.81666666666667	4086850\\
9.83333333333333	4087840\\
9.85	4089010\\
9.86666666666667	4090680\\
9.88333333333333	4091690\\
9.9	4093430\\
9.91666666666667	4094510\\
9.93333333333333	4095490\\
9.95	4097700\\
9.96666666666667	4100270\\
9.98333333333333	4101820\\
10	4103800\\
10.0166666666667	4105740\\
10.0333333333333	4106970\\
10.05	4107980\\
10.0666666666667	4108920\\
10.0833333333333	4110110\\
10.1	4112240\\
10.1166666666667	4114750\\
10.1333333333333	4116550\\
10.15	4117850\\
10.1666666666667	4119480\\
10.1833333333333	4120200\\
10.2	4121300\\
10.2166666666667	4122450\\
10.2333333333333	4124940\\
10.25	4126110\\
10.2666666666667	4127460\\
10.2833333333333	4128460\\
10.3	4129400\\
10.3166666666667	4131240\\
10.3333333333333	4133360\\
10.35	4135170\\
10.3666666666667	4137180\\
10.3833333333333	4138820\\
10.4	4140320\\
10.4166666666667	4141590\\
10.4333333333333	4143090\\
10.45	4144590\\
10.4666666666667	4146200\\
10.4833333333333	4147240\\
10.5	4148510\\
10.5166666666667	4149520\\
10.5333333333333	4150610\\
10.55	4151110\\
10.5666666666667	4151700\\
10.5833333333333	4153080\\
10.6	4153860\\
10.6166666666667	4154860\\
10.6333333333333	4155830\\
10.65	4156460\\
10.6666666666667	4158160\\
10.6833333333333	4159850\\
10.7	4162790\\
10.7166666666667	4165100\\
10.7333333333333	4167310\\
10.75	4168510\\
10.7666666666667	4169310\\
10.7833333333333	4170930\\
10.8	4171340\\
10.8166666666667	4171700\\
10.8333333333333	4171580\\
10.85	4173000\\
10.8666666666667	4174280\\
10.8833333333333	4175780\\
10.9	4177460\\
10.9166666666667	4178150\\
10.9333333333333	4179500\\
10.95	4181090\\
10.9666666666667	4182290\\
10.9833333333333	4184370\\
11	4185500\\
11.0166666666667	4185500\\
11.0333333333333	4188450\\
11.05	4126900\\
11.0666666666667	4072880\\
11.0833333333333	4002310\\
11.1	3963330\\
11.1166666666667	3923130\\
11.1333333333333	3888730\\
11.15	3861000\\
11.1666666666667	3841390\\
11.1833333333333	3824320\\
11.2	3811790\\
11.2166666666667	3795840\\
11.2333333333333	3774560\\
11.25	3736610\\
11.2666666666667	3721620\\
11.2833333333333	3708300\\
11.3	3691820\\
11.3166666666667	3677470\\
11.3333333333333	3666420\\
11.35	3657100\\
11.3666666666667	3634300\\
11.3833333333333	3626410\\
11.4	3621290\\
11.4166666666667	3616390\\
11.4333333333333	3612110\\
11.45	3608970\\
11.4666666666667	3607100\\
11.4833333333333	3604820\\
11.5	3602380\\
11.5166666666667	3599930\\
11.5333333333333	3598040\\
11.55	3596610\\
11.5666666666667	3594990\\
11.5833333333333	3594870\\
11.6	3594690\\
11.6166666666667	3594620\\
11.6333333333333	3596410\\
11.65	3596770\\
11.6666666666667	3597180\\
11.6833333333333	3596870\\
11.7	3597050\\
11.7166666666667	3596620\\
11.7333333333333	3596570\\
11.75	3598060\\
11.7666666666667	3597560\\
11.7833333333333	3597030\\
11.8	3598180\\
11.8166666666667	3595240\\
11.8333333333333	3594340\\
11.85	3595660\\
11.8666666666667	3595180\\
11.8833333333333	3593320\\
11.9	3592520\\
11.9166666666667	3592820\\
11.9333333333333	3592860\\
11.95	3594110\\
11.9666666666667	3598300\\
11.9833333333333	3600580\\
12	3606040\\
12.0166666666667	3606540\\
12.0333333333333	3606780\\
12.05	3608610\\
12.0666666666667	3608420\\
12.0833333333333	3609310\\
12.1	3611510\\
12.1166666666667	3612280\\
12.1333333333333	3611190\\
12.15	3611380\\
12.1666666666667	3612360\\
12.1833333333333	3614850\\
12.2	3618230\\
12.2166666666667	3619530\\
12.2333333333333	3620580\\
12.25	3621860\\
12.2666666666667	3621840\\
12.2833333333333	3622110\\
12.3	3623680\\
12.3166666666667	3623950\\
12.3333333333333	3624260\\
12.35	3624840\\
12.3666666666667	3625190\\
12.3833333333333	3626850\\
12.4	3627380\\
12.4166666666667	3630270\\
12.4333333333333	3629770\\
12.45	3629700\\
12.4666666666667	3630480\\
12.4833333333333	3630480\\
12.5	3630260\\
12.5166666666667	3631490\\
12.5333333333333	3632860\\
12.55	3633620\\
12.5666666666667	3633810\\
12.5833333333333	3635320\\
12.6	3636660\\
12.6166666666667	3636980\\
12.6333333333333	3638120\\
12.65	3640120\\
12.6666666666667	3640790\\
12.6833333333333	3648240\\
12.7	3652090\\
12.7166666666667	3654140\\
12.7333333333333	3657540\\
12.75	3659650\\
12.7666666666667	3662700\\
12.7833333333333	3659110\\
12.8	3661830\\
12.8166666666667	3663400\\
12.8333333333333	3664760\\
12.85	3666340\\
12.8666666666667	3667170\\
12.8833333333333	3667950\\
12.9	3669110\\
12.9166666666667	3670880\\
12.9333333333333	3671990\\
12.95	3673260\\
12.9666666666667	3674030\\
12.9833333333333	3676160\\
13	3677180\\
13.0166666666667	3677830\\
13.0333333333333	3678940\\
13.05	3679900\\
13.0666666666667	3681330\\
13.0833333333333	3682750\\
13.1	3683820\\
13.1166666666667	3684700\\
13.1333333333333	3684810\\
13.15	3687610\\
13.1666666666667	3691850\\
13.1833333333333	3690780\\
13.2	3691010\\
13.2166666666667	3691720\\
13.2333333333333	3691870\\
13.25	3693180\\
13.2666666666667	3694000\\
13.2833333333333	3696330\\
13.3	3697600\\
13.3166666666667	3699820\\
13.3333333333333	3701260\\
13.35	3702420\\
13.3666666666667	3703580\\
13.3833333333333	3705620\\
13.4	3706220\\
13.4166666666667	3707150\\
13.4333333333333	3709380\\
13.45	3710420\\
13.4666666666667	3711530\\
13.4833333333333	3712100\\
13.5	3712660\\
13.5166666666667	3713020\\
13.5333333333333	3715250\\
13.55	3716640\\
13.5666666666667	3717340\\
13.5833333333333	3717780\\
13.6	3718630\\
13.6166666666667	3719190\\
13.6333333333333	3719940\\
13.65	3720580\\
13.6666666666667	3721570\\
13.6833333333333	3722530\\
13.7	3723790\\
13.7166666666667	3724320\\
13.7333333333333	3725540\\
13.75	3726460\\
13.7666666666667	3728060\\
13.7833333333333	3731400\\
13.8	3736460\\
13.8166666666667	3746340\\
13.8333333333333	3763330\\
13.85	3763480\\
13.8666666666667	3760960\\
13.8833333333333	3757830\\
13.9	3758930\\
13.9166666666667	3760040\\
13.9333333333333	3761310\\
13.95	3764570\\
13.9666666666667	3767910\\
13.9833333333333	3767650\\
14	3767380\\
14.0166666666667	3767940\\
14.0333333333333	3768290\\
14.05	3768890\\
14.0666666666667	3768870\\
14.0833333333333	3768750\\
14.1	3769610\\
14.1166666666667	3770310\\
14.1333333333333	3770770\\
14.15	3771910\\
14.1666666666667	3772830\\
14.1833333333333	3773490\\
14.2	3773660\\
14.2166666666667	3774320\\
14.2333333333333	3775090\\
14.25	3775430\\
14.2666666666667	3775780\\
14.2833333333333	3775960\\
14.3	3778320\\
14.3166666666667	3780200\\
14.3333333333333	3780040\\
14.35	3781610\\
14.3666666666667	3782210\\
14.3833333333333	3782740\\
14.4	3783300\\
14.4166666666667	3784330\\
14.4333333333333	3784770\\
14.45	3785320\\
14.4666666666667	3785640\\
14.4833333333333	3786110\\
14.5	3786640\\
14.5166666666667	3788160\\
14.5333333333333	3789420\\
14.55	3789870\\
14.5666666666667	3791600\\
14.5833333333333	3793460\\
14.6	3793650\\
14.6166666666667	3793810\\
14.6333333333333	3794510\\
14.65	3795920\\
14.6666666666667	3796620\\
14.6833333333333	3797500\\
14.7	3799880\\
14.7166666666667	3803670\\
14.7333333333333	3805380\\
14.75	3804780\\
14.7666666666667	3805630\\
14.7833333333333	3806270\\
14.8	3807680\\
14.8166666666667	3808500\\
14.8333333333333	3808880\\
14.85	3809500\\
14.8666666666667	3801210\\
14.8833333333333	3801660\\
14.9	3799810\\
14.9166666666667	3799970\\
14.9333333333333	3800110\\
14.95	3799860\\
14.9666666666667	3799650\\
14.9833333333333	3800090\\
15	3799900\\
15.0166666666667	3799960\\
15.0333333333333	3800760\\
15.05	3802020\\
15.0666666666667	3803710\\
15.0833333333333	3803330\\
15.1	3803460\\
15.1166666666667	3803320\\
15.1333333333333	3803810\\
15.15	3804660\\
15.1666666666667	3805190\\
15.1833333333333	3805540\\
15.2	3805570\\
15.2166666666667	3806730\\
15.2333333333333	3807900\\
15.25	3808590\\
15.2666666666667	3810590\\
15.2833333333333	3810930\\
15.3	3811520\\
15.3166666666667	3811980\\
15.3333333333333	3812650\\
15.35	3813550\\
15.3666666666667	3814800\\
15.3833333333333	3816060\\
15.4	3816170\\
15.4166666666667	3816620\\
15.4333333333333	3817390\\
15.45	3818500\\
15.4666666666667	3819580\\
15.4833333333333	3820630\\
15.5	3821500\\
15.5166666666667	3822340\\
15.5333333333333	3823430\\
15.55	3824200\\
15.5666666666667	3826550\\
15.5833333333333	3828390\\
15.6	3829580\\
15.6166666666667	3829350\\
15.6333333333333	3828800\\
15.65	3829070\\
15.6666666666667	3830030\\
15.6833333333333	3830350\\
15.7	3830820\\
15.7166666666667	3831530\\
15.7333333333333	3832450\\
15.75	3833690\\
15.7666666666667	3834750\\
15.7833333333333	3835100\\
15.8	3836210\\
15.8166666666667	3837330\\
15.8333333333333	3837960\\
15.85	3840320\\
15.8666666666667	3839100\\
15.8833333333333	3839160\\
15.9	3840670\\
15.9166666666667	3841420\\
15.9333333333333	3839970\\
15.95	3840060\\
15.9666666666667	3840850\\
15.9833333333333	3841030\\
16	3843530\\
16.0166666666667	3842830\\
16.0333333333333	3842440\\
16.05	3842540\\
16.0666666666667	3843230\\
16.0833333333333	3843820\\
16.1	3843640\\
16.1166666666667	3844070\\
16.1333333333333	3844230\\
16.15	3845100\\
16.1666666666667	3847630\\
16.1833333333333	3848740\\
16.2	3849580\\
16.2166666666667	3850130\\
16.2333333333333	3849980\\
16.25	3850110\\
16.2666666666667	3850610\\
16.2833333333333	3836740\\
16.3	3833830\\
16.3166666666667	3834190\\
16.3333333333333	3834660\\
16.35	3835710\\
16.3666666666667	3837150\\
16.3833333333333	3838430\\
16.4	3839210\\
16.4166666666667	3839480\\
16.4333333333333	3839710\\
16.45	3840250\\
16.4666666666667	3840900\\
16.4833333333333	3841140\\
16.5	3841470\\
16.5166666666667	3841890\\
16.5333333333333	3842850\\
16.55	3844750\\
16.5666666666667	3845340\\
16.5833333333333	3846260\\
16.6	3845920\\
16.6166666666667	3847890\\
16.6333333333333	3848300\\
16.65	3849220\\
16.6666666666667	3844850\\
16.6833333333333	3851200\\
16.7	3853540\\
16.7166666666667	3854750\\
16.7333333333333	3855370\\
16.75	3855910\\
16.7666666666667	3856950\\
16.7833333333333	3857940\\
16.8	3858440\\
16.8166666666667	3859420\\
16.8333333333333	3859540\\
16.85	3860250\\
16.8666666666667	3861110\\
16.8833333333333	3862270\\
16.9	3863110\\
16.9166666666667	3862840\\
16.9333333333333	3862790\\
16.95	3862840\\
16.9666666666667	3863360\\
16.9833333333333	3864190\\
17	3865270\\
17.0166666666667	3866340\\
17.0333333333333	3866980\\
17.05	3867880\\
17.0666666666667	3867560\\
17.0833333333333	3867090\\
17.1	3867500\\
17.1166666666667	3868180\\
17.1333333333333	3869500\\
17.15	3870930\\
17.1666666666667	3872710\\
17.1833333333333	3875050\\
17.2	3876330\\
17.2166666666667	3877790\\
17.2333333333333	3879360\\
17.25	3883440\\
17.2666666666667	3886250\\
17.2833333333333	3887750\\
17.3	3888960\\
17.3166666666667	3891960\\
17.3333333333333	3894000\\
17.35	3894590\\
17.3666666666667	3895050\\
17.3833333333333	3896270\\
17.4	3897940\\
17.4166666666667	3899270\\
17.4333333333333	3899920\\
17.45	3900930\\
17.4666666666667	3901790\\
17.4833333333333	3903150\\
17.5	3903490\\
17.5166666666667	3903820\\
17.5333333333333	3904610\\
17.55	3907250\\
17.5666666666667	3909220\\
17.5833333333333	3910760\\
17.6	3911610\\
17.6166666666667	3912740\\
17.6333333333333	3913420\\
17.65	3913980\\
17.6666666666667	3915150\\
17.6833333333333	3917150\\
17.7	3918440\\
17.7166666666667	3919650\\
17.7333333333333	3921010\\
17.75	3921790\\
17.7666666666667	3923920\\
17.7833333333333	3924530\\
17.8	3924880\\
17.8166666666667	3925340\\
17.8333333333333	3926710\\
17.85	3927240\\
17.8666666666667	3927040\\
17.8833333333333	3927080\\
17.9	3927790\\
17.9166666666667	3928780\\
17.9333333333333	3930160\\
17.95	3932690\\
17.9666666666667	3935010\\
17.9833333333333	3935710\\
18	3935000\\
18.0166666666667	3935050\\
18.0333333333333	3935390\\
18.05	3936270\\
18.0666666666667	3938000\\
18.0833333333333	3939770\\
18.1	3941410\\
18.1166666666667	3942800\\
18.1333333333333	3943930\\
18.15	3944290\\
18.1666666666667	3944510\\
18.1833333333333	3945330\\
18.2	3946060\\
18.2166666666667	3947550\\
18.2333333333333	3948730\\
18.25	3948620\\
18.2666666666667	3947820\\
18.2833333333333	3948640\\
18.3	3950960\\
18.3166666666667	3951330\\
18.3333333333333	3952080\\
18.35	3952750\\
18.3666666666667	3952720\\
18.3833333333333	3953530\\
18.4	3954930\\
18.4166666666667	3956120\\
18.4333333333333	3956920\\
18.45	3958760\\
18.4666666666667	3959780\\
18.4833333333333	3960900\\
18.5	3961160\\
18.5166666666667	3960630\\
18.5333333333333	3961330\\
18.55	3963570\\
18.5666666666667	3964870\\
18.5833333333333	3966320\\
18.6	3967610\\
18.6166666666667	3969090\\
18.6333333333333	3970130\\
18.65	3970350\\
18.6666666666667	3970710\\
18.6833333333333	3970820\\
18.7	3970700\\
18.7166666666667	3971400\\
18.7333333333333	3972430\\
18.75	3973870\\
18.7666666666667	3976040\\
18.7833333333333	3977940\\
18.8	3979590\\
18.8166666666667	3979390\\
18.8333333333333	3967300\\
18.85	3977710\\
18.8666666666667	3983080\\
18.8833333333333	3986300\\
18.9	3993740\\
18.9166666666667	3997880\\
18.9333333333333	4000730\\
18.95	4003150\\
18.9666666666667	3950460\\
18.9833333333333	3856300\\
19	3832460\\
19.0166666666667	3814960\\
19.0333333333333	3805670\\
19.05	3799440\\
19.0666666666667	3795090\\
19.0833333333333	3791070\\
19.1	3788350\\
19.1166666666667	3785510\\
19.1333333333333	3783990\\
19.15	3782160\\
19.1666666666667	3780560\\
19.1833333333333	3779530\\
19.2	3778970\\
19.2166666666667	3778800\\
19.2333333333333	3777740\\
19.25	3777550\\
19.2666666666667	3777430\\
19.2833333333333	3777910\\
19.3	3778330\\
19.3166666666667	3778860\\
19.3333333333333	3780170\\
19.35	3781480\\
19.3666666666667	3782340\\
19.3833333333333	3783100\\
19.4	3783800\\
19.4166666666667	3785060\\
19.4333333333333	3786560\\
19.45	3787530\\
19.4666666666667	3788650\\
19.4833333333333	3789210\\
19.5	3790150\\
19.5166666666667	3791300\\
19.5333333333333	3787850\\
19.55	3793130\\
19.5666666666667	3796030\\
19.5833333333333	3797360\\
19.6	3798830\\
19.6166666666667	3800180\\
19.6333333333333	3801440\\
19.65	3802440\\
19.6666666666667	3805370\\
19.6833333333333	3806180\\
19.7	3807560\\
19.7166666666667	3809090\\
19.7333333333333	3810510\\
19.75	3811370\\
19.7666666666667	3811890\\
19.7833333333333	3812740\\
19.8	3814080\\
19.8166666666667	3813410\\
19.8333333333333	3814540\\
19.85	3815140\\
19.8666666666667	3817610\\
19.8833333333333	3819400\\
19.9	3820840\\
19.9166666666667	3821900\\
19.9333333333333	3823180\\
19.95	3824350\\
19.9666666666667	3825540\\
19.9833333333333	3826480\\
20	3827390\\
20.0166666666667	3828370\\
20.0333333333333	3829360\\
20.05	3830640\\
20.0666666666667	3832030\\
20.0833333333333	3832820\\
20.1	3833620\\
20.1166666666667	3834610\\
20.1333333333333	3834850\\
20.15	3836130\\
20.1666666666667	3836710\\
20.1833333333333	3838920\\
20.2	3839400\\
20.2166666666667	3841030\\
20.2333333333333	3842350\\
20.25	3843250\\
20.2666666666667	3844490\\
20.2833333333333	3846120\\
20.3	3847420\\
20.3166666666667	3848580\\
20.3333333333333	3850500\\
20.35	3850660\\
20.3666666666667	3851310\\
20.3833333333333	3852600\\
20.4	3852690\\
20.4166666666667	3853910\\
20.4333333333333	3855440\\
20.45	3857520\\
20.4666666666667	3858600\\
20.4833333333333	3860020\\
20.5	3860600\\
20.5166666666667	3861990\\
20.5333333333333	3863530\\
20.55	3864710\\
20.5666666666667	3866110\\
20.5833333333333	3867060\\
20.6	3868250\\
20.6166666666667	3869140\\
20.6333333333333	3870500\\
20.65	3871720\\
20.6666666666667	3873260\\
20.6833333333333	3875240\\
20.7	3880180\\
20.7166666666667	3879460\\
20.7333333333333	3880650\\
20.75	3881410\\
20.7666666666667	3882430\\
20.7833333333333	3883210\\
20.8	3883980\\
20.8166666666667	3884960\\
20.8333333333333	3885910\\
20.85	3886810\\
20.8666666666667	3887720\\
20.8833333333333	3888640\\
20.9	3889800\\
20.9166666666667	3890400\\
20.9333333333333	3891640\\
20.95	3892950\\
20.9666666666667	3893670\\
20.9833333333333	3895100\\
21	3895510\\
21.0166666666667	3896970\\
21.0333333333333	3896730\\
21.05	3898420\\
21.0666666666667	3899680\\
21.0833333333333	3900170\\
21.1	3900880\\
21.1166666666667	3901490\\
21.1333333333333	3902410\\
21.15	3903190\\
21.1666666666667	3904250\\
21.1833333333333	3904940\\
21.2	3905760\\
21.2166666666667	3906400\\
21.2333333333333	3907400\\
21.25	3908260\\
21.2666666666667	3909240\\
21.2833333333333	3910210\\
21.3	3911200\\
21.3166666666667	3912030\\
21.3333333333333	3913300\\
21.35	3914360\\
21.3666666666667	3915480\\
21.3833333333333	3916290\\
21.4	3917600\\
21.4166666666667	3918590\\
21.4333333333333	3919700\\
21.45	3920600\\
21.4666666666667	3921810\\
21.4833333333333	3922700\\
21.5	3923830\\
21.5166666666667	3924800\\
21.5333333333333	3925930\\
21.55	3927010\\
21.5666666666667	3927990\\
21.5833333333333	3929060\\
21.6	3929790\\
21.6166666666667	3930920\\
21.6333333333333	3931680\\
21.65	3933140\\
21.6666666666667	3934000\\
21.6833333333333	3935250\\
21.7	3935860\\
21.7166666666667	3936340\\
21.7333333333333	3936620\\
21.75	3937990\\
21.7666666666667	3938470\\
21.7833333333333	3938820\\
21.8	3939320\\
21.8166666666667	3939710\\
21.8333333333333	3940670\\
21.85	3941720\\
21.8666666666667	3942230\\
21.8833333333333	3943450\\
21.9	3944250\\
21.9166666666667	3945880\\
21.9333333333333	3947130\\
21.95	3948090\\
21.9666666666667	3949130\\
21.9833333333333	3949850\\
22	3950280\\
22.0166666666667	3951190\\
22.0333333333333	3952270\\
22.05	3952910\\
22.0666666666667	3953890\\
22.0833333333333	3954990\\
22.1	3956010\\
22.1166666666667	3956480\\
22.1333333333333	3956900\\
22.15	3957270\\
22.1666666666667	3957980\\
22.1833333333333	3958490\\
22.2	3959330\\
22.2166666666667	3959580\\
22.2333333333333	3960280\\
22.25	3960520\\
22.2666666666667	3961010\\
22.2833333333333	3962260\\
22.3	3962630\\
22.3166666666667	3962660\\
22.3333333333333	3962380\\
22.35	3962540\\
22.3666666666667	3963230\\
22.3833333333333	3963900\\
22.4	3965040\\
22.4166666666667	3965710\\
22.4333333333333	3966500\\
22.45	3967450\\
22.4666666666667	3968600\\
22.4833333333333	3969610\\
22.5	3970850\\
22.5166666666667	3972130\\
22.5333333333333	3973130\\
22.55	3974340\\
22.5666666666667	3975600\\
22.5833333333333	3976940\\
22.6	3978150\\
22.6166666666667	3979750\\
22.6333333333333	3980830\\
22.65	3981990\\
22.6666666666667	3983380\\
22.6833333333333	3984720\\
22.7	3986340\\
22.7166666666667	3987880\\
22.7333333333333	3989250\\
22.75	3990650\\
22.7666666666667	3992070\\
22.7833333333333	3993670\\
22.8	3995020\\
22.8166666666667	3996330\\
22.8333333333333	3997520\\
22.85	3998360\\
22.8666666666667	3999770\\
22.8833333333333	4000540\\
22.9	4001190\\
22.9166666666667	4001970\\
22.9333333333333	4002940\\
22.95	4003930\\
22.9666666666667	4005150\\
22.9833333333333	4006330\\
23	4007000\\
23.0166666666667	4007770\\
23.0333333333333	4009090\\
23.05	4009660\\
23.0666666666667	4010430\\
23.0833333333333	4010980\\
23.1	4011960\\
23.1166666666667	4012890\\
23.1333333333333	4013170\\
23.15	4013730\\
23.1666666666667	4014770\\
23.1833333333333	4015560\\
23.2	4017890\\
23.2166666666667	4020460\\
23.2333333333333	4020180\\
23.25	4020780\\
23.2666666666667	4021600\\
23.2833333333333	4022740\\
23.3	4023310\\
23.3166666666667	4023820\\
23.3333333333333	4024450\\
23.35	4025120\\
23.3666666666667	4026120\\
23.3833333333333	4026540\\
23.4	4026790\\
23.4166666666667	4027460\\
23.4333333333333	4028170\\
23.45	4029190\\
23.4666666666667	4029830\\
23.4833333333333	4030770\\
23.5	4031720\\
23.5166666666667	4032510\\
23.5333333333333	4033850\\
23.55	4035000\\
23.5666666666667	4035650\\
23.5833333333333	4036740\\
23.6	4037410\\
23.6166666666667	4038170\\
23.6333333333333	4038100\\
23.65	4038850\\
23.6666666666667	4039390\\
23.6833333333333	4039990\\
23.7	4040460\\
23.7166666666667	4041290\\
23.7333333333333	4042160\\
23.75	4042800\\
23.7666666666667	4042810\\
23.7833333333333	4043990\\
23.8	4048070\\
23.8166666666667	4050370\\
23.8333333333333	4050550\\
23.85	4049800\\
23.8666666666667	4050110\\
23.8833333333333	4050110\\
23.9	4050000\\
23.9166666666667	4050520\\
23.9333333333333	4050720\\
23.95	4051600\\
23.9666666666667	4052840\\
23.9833333333333	4053790\\
24	4054570\\
24.0166666666667	4054750\\
24.0333333333333	4054730\\
24.05	4055570\\
24.0666666666667	4056550\\
24.0833333333333	4056160\\
24.1	4057160\\
24.1166666666667	4057850\\
24.1333333333333	4058370\\
24.15	4058640\\
24.1666666666667	4059190\\
24.1833333333333	4059890\\
24.2	4060750\\
24.2166666666667	4062210\\
24.2333333333333	4062960\\
24.25	4063310\\
24.2666666666667	4063940\\
24.2833333333333	4063950\\
24.3	4064560\\
24.3166666666667	4065160\\
24.3333333333333	4066040\\
24.35	4066340\\
24.3666666666667	4066940\\
24.3833333333333	4067420\\
24.4	4067960\\
24.4166666666667	4068450\\
24.4333333333333	4069090\\
24.45	4069450\\
24.4666666666667	4070400\\
24.4833333333333	4071530\\
24.5	4072000\\
24.5166666666667	4072850\\
24.5333333333333	4073790\\
24.55	4074160\\
24.5666666666667	4075850\\
24.5833333333333	4076370\\
24.6	4078130\\
24.6166666666667	4079710\\
24.6333333333333	4081880\\
24.65	4084200\\
24.6666666666667	4086990\\
24.6833333333333	4089490\\
24.7	4091750\\
24.7166666666667	4093620\\
24.7333333333333	4095840\\
24.75	4098310\\
24.7666666666667	4100490\\
24.7833333333333	4102440\\
24.8	4104330\\
24.8166666666667	4106180\\
24.8333333333333	4107730\\
24.85	4108950\\
24.8666666666667	4110460\\
24.8833333333333	4111970\\
24.9	4113690\\
24.9166666666667	4115680\\
24.9333333333333	4116950\\
24.95	4118140\\
24.9666666666667	4119680\\
24.9833333333333	4122090\\
25	4123090\\
25.0166666666667	4125070\\
25.0333333333333	4126330\\
25.05	4126900\\
25.0666666666667	4127580\\
25.0833333333333	4127970\\
25.1	4128530\\
25.1166666666667	4129370\\
25.1333333333333	4130550\\
25.15	4131430\\
25.1666666666667	4131960\\
25.1833333333333	4132600\\
25.2	4133650\\
25.2166666666667	4134390\\
25.2333333333333	4135130\\
25.25	4135960\\
25.2666666666667	4136920\\
25.2833333333333	4137480\\
25.3	4138340\\
25.3166666666667	4139250\\
25.3333333333333	4139580\\
25.35	4140450\\
25.3666666666667	4141350\\
25.3833333333333	4142290\\
25.4	4143690\\
25.4166666666667	4144410\\
25.4333333333333	4145430\\
25.45	4146810\\
25.4666666666667	4148610\\
25.4833333333333	4150530\\
25.5	4151960\\
25.5166666666667	4153040\\
25.5333333333333	4154860\\
25.55	4156230\\
25.5666666666667	4157330\\
25.5833333333333	4158980\\
25.6	4160080\\
25.6166666666667	4161550\\
25.6333333333333	4163140\\
25.65	4163950\\
25.6666666666667	4165020\\
25.6833333333333	4165900\\
25.7	4166690\\
25.7166666666667	4167100\\
25.7333333333333	4167800\\
25.75	4168390\\
25.7666666666667	4168890\\
25.7833333333333	4169650\\
25.8	4171880\\
25.8166666666667	4172650\\
25.8333333333333	4173490\\
25.85	4175000\\
25.8666666666667	4175890\\
25.8833333333333	4176620\\
25.9	4177390\\
25.9166666666667	4178250\\
25.9333333333333	4178690\\
25.95	4179900\\
25.9666666666667	4181000\\
25.9833333333333	4182120\\
26	4183380\\
26.0166666666667	4184490\\
26.0333333333333	4185370\\
26.05	4186240\\
26.0666666666667	4187390\\
26.0833333333333	4188340\\
26.1	4189880\\
26.1166666666667	4190450\\
26.1333333333333	4190830\\
26.15	4191740\\
26.1666666666667	4192390\\
26.1833333333333	4193550\\
26.2	4194450\\
26.2166666666667	4194920\\
26.2333333333333	4195430\\
26.25	4195770\\
26.2666666666667	4196530\\
26.2833333333333	4196850\\
26.3	4197550\\
26.3166666666667	4198780\\
26.3333333333333	4199190\\
26.35	4199500\\
26.3666666666667	4200400\\
26.3833333333333	4201600\\
26.4	4201860\\
26.4166666666667	4203580\\
26.4333333333333	4203170\\
26.45	4204500\\
26.4666666666667	4205310\\
26.4833333333333	4206360\\
26.5	4207100\\
26.5166666666667	4208070\\
26.5333333333333	4208120\\
26.55	4209690\\
26.5666666666667	4210700\\
26.5833333333333	4211490\\
26.6	4212580\\
26.6166666666667	4213530\\
26.6333333333333	4214310\\
26.65	4215210\\
26.6666666666667	4216300\\
26.6833333333333	4217610\\
26.7	4218250\\
26.7166666666667	4219340\\
26.7333333333333	4219650\\
26.75	4220430\\
26.7666666666667	4220760\\
26.7833333333333	4221270\\
26.8	4221190\\
26.8166666666667	4221800\\
26.8333333333333	4222750\\
26.85	4224170\\
26.8666666666667	4225320\\
26.8833333333333	4226270\\
26.9	4226810\\
26.9166666666667	4226900\\
26.9333333333333	4227590\\
26.95	4227300\\
26.9666666666667	4227640\\
26.9833333333333	4227940\\
27	4227780\\
27.0166666666667	4228570\\
27.0333333333333	4229380\\
27.05	4229570\\
27.0666666666667	4230750\\
27.0833333333333	4231270\\
27.1	4231580\\
27.1166666666667	4231900\\
27.1333333333333	4232290\\
27.15	4232750\\
27.1666666666667	4233380\\
27.1833333333333	4233730\\
27.2	4234400\\
27.2166666666667	4235180\\
27.2333333333333	4236100\\
27.25	4236800\\
27.2666666666667	4237120\\
27.2833333333333	4237650\\
27.3	4238960\\
27.3166666666667	4240110\\
27.3333333333333	4241290\\
27.35	4241300\\
27.3666666666667	4241670\\
27.3833333333333	4243280\\
27.4	4243430\\
27.4166666666667	4244610\\
27.4333333333333	4244600\\
27.45	4245420\\
27.4666666666667	4245520\\
27.4833333333333	4246080\\
27.5	4246890\\
27.5166666666667	4245880\\
27.5333333333333	4245850\\
27.55	4245260\\
27.5666666666667	4244750\\
27.5833333333333	4243910\\
27.6	4243760\\
27.6166666666667	4243340\\
27.6333333333333	4242840\\
27.65	4242890\\
27.6666666666667	4242950\\
27.6833333333333	4243050\\
27.7	4242670\\
27.7166666666667	4242370\\
27.7333333333333	4242390\\
27.75	4242070\\
27.7666666666667	4242230\\
27.7833333333333	4242420\\
27.8	4241910\\
27.8166666666667	4242060\\
27.8333333333333	4242670\\
27.85	4242310\\
27.8666666666667	4242520\\
27.8833333333333	4242890\\
27.9	4242840\\
27.9166666666667	4243460\\
27.9333333333333	4243360\\
27.95	4243450\\
27.9666666666667	4243810\\
27.9833333333333	4244070\\
28	4244500\\
28.0166666666667	4244520\\
28.0333333333333	4244460\\
28.05	4244680\\
28.0666666666667	4245710\\
28.0833333333333	4246260\\
28.1	4246110\\
28.1166666666667	4245600\\
28.1333333333333	4245850\\
28.15	4246130\\
28.1666666666667	4245370\\
28.1833333333333	4245430\\
28.2	4245770\\
28.2166666666667	4246170\\
28.2333333333333	4246470\\
28.25	4246640\\
28.2666666666667	4246790\\
28.2833333333333	4241560\\
28.3	4231200\\
28.3166666666667	4230150\\
28.3333333333333	4228940\\
28.35	4228590\\
28.3666666666667	4228590\\
28.3833333333333	4228240\\
28.4	4228140\\
28.4166666666667	4227920\\
28.4333333333333	4228300\\
28.45	4228310\\
28.4666666666667	4228210\\
28.4833333333333	4228760\\
28.5	4229680\\
28.5166666666667	4229430\\
28.5333333333333	4229900\\
28.55	4229760\\
28.5666666666667	4229900\\
28.5833333333333	4229850\\
28.6	4229790\\
28.6166666666667	4230350\\
28.6333333333333	4230960\\
28.65	4231620\\
28.6666666666667	4232310\\
28.6833333333333	4232720\\
28.7	4233600\\
28.7166666666667	4234520\\
28.7333333333333	4235220\\
28.75	4236040\\
28.7666666666667	4236760\\
28.7833333333333	4237070\\
28.8	4237380\\
28.8166666666667	4238590\\
28.8333333333333	4239210\\
28.85	4239960\\
28.8666666666667	4240480\\
28.8833333333333	4241500\\
28.9	4242410\\
28.9166666666667	4242910\\
28.9333333333333	4243610\\
28.95	4244020\\
28.9666666666667	4244610\\
28.9833333333333	4245060\\
29	4245060\\
29.0166666666667	4245220\\
29.0333333333333	4245990\\
29.05	4246630\\
29.0666666666667	4247400\\
29.0833333333333	4248240\\
29.1	4248720\\
29.1166666666667	4248870\\
29.1333333333333	4249230\\
29.15	4250000\\
29.1666666666667	4250480\\
29.1833333333333	4250690\\
29.2	4251370\\
29.2166666666667	4252160\\
29.2333333333333	4252870\\
29.25	4253810\\
29.2666666666667	4254200\\
29.2833333333333	4255220\\
29.3	4256160\\
29.3166666666667	4256700\\
29.3333333333333	4257380\\
29.35	4257700\\
29.3666666666667	4257970\\
29.3833333333333	4258880\\
29.4	4259300\\
29.4166666666667	4259600\\
29.4333333333333	4260350\\
29.45	4261350\\
29.4666666666667	4262340\\
29.4833333333333	4262380\\
29.5	4263150\\
29.5166666666667	4264240\\
29.5333333333333	4264460\\
29.55	4265250\\
29.5666666666667	4265550\\
29.5833333333333	4266180\\
29.6	4267400\\
29.6166666666667	4267570\\
29.6333333333333	4267890\\
29.65	4268360\\
29.6666666666667	4269080\\
29.6833333333333	4269580\\
29.7	4270500\\
29.7166666666667	4271230\\
29.7333333333333	4272180\\
29.75	4272830\\
29.7666666666667	4273560\\
29.7833333333333	4274240\\
29.8	4274560\\
29.8166666666667	4275230\\
29.8333333333333	4276000\\
29.85	4276740\\
29.8666666666667	4277580\\
29.8833333333333	4278430\\
29.9	4278780\\
29.9166666666667	4279870\\
29.9333333333333	4280360\\
29.95	4280910\\
29.9666666666667	4281880\\
29.9833333333333	4282560\\
};

\addplot [color=mycolor3]
  table[row sep=crcr]{%
0	11608800\\
0.0166666666666667	12234600\\
0.0333333333333333	12866800\\
0.05	14140100\\
0.0666666666666667	15834700\\
0.0833333333333333	16345200\\
0.1	16397300\\
0.116666666666667	16068100\\
0.133333333333333	16561700\\
0.15	18850400\\
0.166666666666667	20019600\\
0.183333333333333	19242200\\
0.2	17989600\\
0.216666666666667	21921700\\
0.233333333333333	19043200\\
0.25	19347700\\
0.266666666666667	19152200\\
0.283333333333333	18900900\\
0.3	18875000\\
0.316666666666667	18852200\\
0.333333333333333	18833300\\
0.35	18655600\\
0.366666666666667	18570100\\
0.383333333333333	19306000\\
0.4	20586800\\
0.416666666666667	22581900\\
0.433333333333333	24575300\\
0.45	26503400\\
0.466666666666667	25805500\\
0.483333333333333	27169200\\
0.5	29009400\\
0.516666666666667	31229100\\
0.533333333333333	34307300\\
0.55	39728800\\
0.566666666666667	43243200\\
0.583333333333333	44592100\\
0.6	42500400\\
0.616666666666667	40894100\\
0.633333333333333	45089500\\
0.65	50181900\\
0.666666666666667	51713900\\
0.683333333333333	52767900\\
0.7	52265300\\
0.716666666666667	57317100\\
0.733333333333333	60389100\\
0.75	54133400\\
0.766666666666667	63276200\\
0.783333333333333	64790500\\
0.8	58431400\\
0.816666666666667	53951700\\
0.833333333333333	52594000\\
0.85	51740000\\
0.866666666666667	56694600\\
0.883333333333333	57202800\\
0.9	51798900\\
0.916666666666667	53135200\\
0.933333333333333	55076100\\
0.95	62996500\\
0.966666666666667	63544400\\
0.983333333333333	60008200\\
1	57656900\\
1.01666666666667	53885300\\
1.03333333333333	55780000\\
1.05	55625100\\
1.06666666666667	55058100\\
1.08333333333333	55117300\\
1.1	59250300\\
1.11666666666667	63840300\\
1.13333333333333	64582300\\
1.15	63483300\\
1.16666666666667	68400600\\
1.18333333333333	67750900\\
1.2	69902400\\
1.21666666666667	75652100\\
1.23333333333333	79038700\\
1.25	79996800\\
1.26666666666667	77805300\\
1.28333333333333	74920200\\
1.3	77374500\\
1.31666666666667	86458100\\
1.33333333333333	96061900\\
1.35	104769000\\
1.36666666666667	114763000\\
1.38333333333333	119047000\\
1.4	118008000\\
1.41666666666667	112847000\\
1.43333333333333	109900000\\
1.45	105254000\\
1.46666666666667	103375000\\
1.48333333333333	105194000\\
1.5	106062000\\
1.51666666666667	109049000\\
1.53333333333333	111281000\\
1.55	112887000\\
1.56666666666667	118100000\\
1.58333333333333	120699000\\
1.6	121873000\\
1.61666666666667	121458000\\
1.63333333333333	121731000\\
1.65	122598000\\
1.66666666666667	122192000\\
1.68333333333333	121786000\\
1.7	122260000\\
1.71666666666667	122897000\\
1.73333333333333	122122000\\
1.75	121834000\\
1.76666666666667	121712000\\
1.78333333333333	121675000\\
1.8	121277000\\
1.81666666666667	119619000\\
1.83333333333333	116800000\\
1.85	112452000\\
1.86666666666667	108868000\\
1.88333333333333	107869000\\
1.9	107692000\\
1.91666666666667	109365000\\
1.93333333333333	111115000\\
1.95	111361000\\
1.96666666666667	111474000\\
1.98333333333333	108489000\\
2	110046000\\
2.01666666666667	113337000\\
2.03333333333333	116865000\\
2.05	120264000\\
2.06666666666667	121188000\\
2.08333333333333	119401000\\
2.1	118663000\\
2.11666666666667	117986000\\
2.13333333333333	114833000\\
2.15	116543000\\
2.16666666666667	119150000\\
2.18333333333333	119425000\\
2.2	118766000\\
2.21666666666667	118966000\\
2.23333333333333	118881000\\
2.25	120711000\\
2.26666666666667	121243000\\
2.28333333333333	122400000\\
2.3	122001000\\
2.31666666666667	121890000\\
2.33333333333333	122127000\\
2.35	122321000\\
2.36666666666667	122641000\\
2.38333333333333	123105000\\
2.4	120878000\\
2.41666666666667	118413000\\
2.43333333333333	118145000\\
2.45	119873000\\
2.46666666666667	119094000\\
2.48333333333333	118395000\\
2.5	117446000\\
2.51666666666667	112985000\\
2.53333333333333	107418000\\
2.55	106123000\\
2.56666666666667	108149000\\
2.58333333333333	103607000\\
2.6	100091000\\
2.61666666666667	101218000\\
2.63333333333333	104390000\\
2.65	102146000\\
2.66666666666667	94708700\\
2.68333333333333	88812500\\
2.7	81410300\\
2.71666666666667	77126100\\
2.73333333333333	74643400\\
2.75	74807700\\
2.76666666666667	76946400\\
2.78333333333333	75034200\\
2.8	76384900\\
2.81666666666667	76739700\\
2.83333333333333	79558700\\
2.85	81134100\\
2.86666666666667	86179100\\
2.88333333333333	94437600\\
2.9	102011000\\
2.91666666666667	104283000\\
2.93333333333333	103068000\\
2.95	103889000\\
2.96666666666667	104876000\\
2.98333333333333	106185000\\
3	109549000\\
3.01666666666667	111290000\\
3.03333333333333	108383000\\
3.05	105342000\\
3.06666666666667	105534000\\
3.08333333333333	108412000\\
3.1	104501000\\
3.11666666666667	99261000\\
3.13333333333333	100178000\\
3.15	97259900\\
3.16666666666667	95091400\\
3.18333333333333	89856000\\
3.2	85155800\\
3.21666666666667	80250700\\
3.23333333333333	76194000\\
3.25	74392100\\
3.26666666666667	75687400\\
3.28333333333333	73196400\\
3.3	74389500\\
3.31666666666667	75638900\\
3.33333333333333	77083100\\
3.35	81047400\\
3.36666666666667	80944000\\
3.38333333333333	78590000\\
3.4	76294600\\
3.41666666666667	75975900\\
3.43333333333333	79195100\\
3.45	86149400\\
3.46666666666667	97512000\\
3.48333333333333	98575700\\
3.5	95638300\\
3.51666666666667	97756700\\
3.53333333333333	103060000\\
3.55	109760000\\
3.56666666666667	104806000\\
3.58333333333333	105307000\\
3.6	103900000\\
3.61666666666667	107878000\\
3.63333333333333	109830000\\
3.65	102245000\\
3.66666666666667	92293600\\
3.68333333333333	82813300\\
3.7	77866500\\
3.71666666666667	75442700\\
3.73333333333333	80621300\\
3.75	83728500\\
3.76666666666667	82757700\\
3.78333333333333	78541500\\
3.8	83255600\\
3.81666666666667	91973200\\
3.83333333333333	96283600\\
3.85	101714000\\
3.86666666666667	99215800\\
3.88333333333333	99801000\\
3.9	101336000\\
3.91666666666667	100956000\\
3.93333333333333	99986400\\
3.95	96618500\\
3.96666666666667	85617900\\
3.98333333333333	77569900\\
4	73531300\\
4.01666666666667	72201300\\
4.03333333333333	69912700\\
4.05	70857000\\
4.06666666666667	68958500\\
4.08333333333333	68265000\\
4.1	67662500\\
4.11666666666667	66931200\\
4.13333333333333	66111200\\
4.15	65292000\\
4.16666666666667	64171500\\
4.18333333333333	63604600\\
4.2	61804200\\
4.21666666666667	61391100\\
4.23333333333333	63312600\\
4.25	63322200\\
4.26666666666667	63894500\\
4.28333333333333	65693100\\
4.3	66679700\\
4.31666666666667	67964500\\
4.33333333333333	68272500\\
4.35	67083700\\
4.36666666666667	67307200\\
4.38333333333333	67992700\\
4.4	67411100\\
4.41666666666667	67567600\\
4.43333333333333	69295000\\
4.45	70848700\\
4.46666666666667	73451600\\
4.48333333333333	74330700\\
4.5	75924800\\
4.51666666666667	76371600\\
4.53333333333333	79804400\\
4.55	81497000\\
4.56666666666667	80897700\\
4.58333333333333	78128800\\
4.6	76909700\\
4.61666666666667	81105200\\
4.63333333333333	80353500\\
4.65	80832700\\
4.66666666666667	80855900\\
4.68333333333333	79756000\\
4.7	79847500\\
4.71666666666667	80974800\\
4.73333333333333	80790400\\
4.75	77679100\\
4.76666666666667	71633200\\
4.78333333333333	67963100\\
4.8	65013700\\
4.81666666666667	62866200\\
4.83333333333333	62871200\\
4.85	64962000\\
4.86666666666667	68755400\\
4.88333333333333	70406400\\
4.9	72240200\\
4.91666666666667	73220700\\
4.93333333333333	74035000\\
4.95	76305900\\
4.96666666666667	76002400\\
4.98333333333333	75102100\\
5	77326200\\
5.01666666666667	77590300\\
5.03333333333333	77025200\\
5.05	77427000\\
5.06666666666667	78624000\\
5.08333333333333	79199800\\
5.1	78647000\\
5.11666666666667	77130400\\
5.13333333333333	75698800\\
5.15	79473400\\
5.16666666666667	87930200\\
5.18333333333333	94711600\\
5.2	96696100\\
5.21666666666667	95760600\\
5.23333333333333	94786500\\
5.25	93221700\\
5.26666666666667	91695900\\
5.28333333333333	93141100\\
5.3	93276700\\
5.31666666666667	89175600\\
5.33333333333333	85833500\\
5.35	84401000\\
5.36666666666667	89299100\\
5.38333333333333	92685600\\
5.4	91512200\\
5.41666666666667	92626000\\
5.43333333333333	92241700\\
5.45	93815100\\
5.46666666666667	95782900\\
5.48333333333333	94870800\\
5.5	96310900\\
5.51666666666667	92199400\\
5.53333333333333	90701000\\
5.55	87389500\\
5.56666666666667	87129000\\
5.58333333333333	91980200\\
5.6	93980400\\
5.61666666666667	97175200\\
5.63333333333333	97346700\\
5.65	96291800\\
5.66666666666667	95526500\\
5.68333333333333	95617500\\
5.7	94268800\\
5.71666666666667	93927100\\
5.73333333333333	92140200\\
5.75	89183000\\
5.76666666666667	86556900\\
5.78333333333333	93782800\\
5.8	97073300\\
5.81666666666667	95262200\\
5.83333333333333	93305000\\
5.85	91403800\\
5.86666666666667	91866600\\
5.88333333333333	92538800\\
5.9	89985000\\
5.91666666666667	86011000\\
5.93333333333333	84385700\\
5.95	82393000\\
5.96666666666667	78737700\\
5.98333333333333	74757000\\
6	76262300\\
6.01666666666667	79723700\\
6.03333333333333	79037600\\
6.05	77162900\\
6.06666666666667	75368800\\
6.08333333333333	73405500\\
6.1	70069000\\
6.11666666666667	69655300\\
6.13333333333333	69803500\\
6.15	73629600\\
6.16666666666667	71909000\\
6.18333333333333	73289500\\
6.2	71964500\\
6.21666666666667	71392900\\
6.23333333333333	67915400\\
6.25	68245900\\
6.26666666666667	67212800\\
6.28333333333333	69706300\\
6.3	71029500\\
6.31666666666667	73016600\\
6.33333333333333	73994700\\
6.35	74358600\\
6.36666666666667	74953100\\
6.38333333333333	78074200\\
6.4	82100800\\
6.41666666666667	82821400\\
6.43333333333333	85729300\\
6.45	91420000\\
6.46666666666667	93852700\\
6.48333333333333	89929700\\
6.5	88254200\\
6.51666666666667	86471200\\
6.53333333333333	85919500\\
6.55	82521100\\
6.56666666666667	77460400\\
6.58333333333333	74492700\\
6.6	71374500\\
6.61666666666667	66718300\\
6.63333333333333	65553600\\
6.65	64537700\\
6.66666666666667	65710200\\
6.68333333333333	67282800\\
6.7	68649300\\
6.71666666666667	70263600\\
6.73333333333333	73087000\\
6.75	72839100\\
6.76666666666667	71297800\\
6.78333333333333	68009200\\
6.8	64003600\\
6.81666666666667	62939000\\
6.83333333333333	62998800\\
6.85	64137100\\
6.86666666666667	64191900\\
6.88333333333333	65080000\\
6.9	64718200\\
6.91666666666667	62317400\\
6.93333333333333	60576500\\
6.95	57595200\\
6.96666666666667	58245500\\
6.98333333333333	59285100\\
7	58609900\\
7.01666666666667	60310700\\
7.03333333333333	59923800\\
7.05	62162200\\
7.06666666666667	64469000\\
7.08333333333333	65534900\\
7.1	64741200\\
7.11666666666667	65645900\\
7.13333333333333	67636600\\
7.15	67044000\\
7.16666666666667	69238100\\
7.18333333333333	71155500\\
7.2	68762900\\
7.21666666666667	66637500\\
7.23333333333333	66290300\\
7.25	61921100\\
7.26666666666667	60308200\\
7.28333333333333	59693300\\
7.3	58359900\\
7.31666666666667	56616700\\
7.33333333333333	55742600\\
7.35	54123700\\
7.36666666666667	54495000\\
7.38333333333333	54620200\\
7.4	54280200\\
7.41666666666667	54232800\\
7.43333333333333	54864700\\
7.45	55513700\\
7.46666666666667	53907400\\
7.48333333333333	53540200\\
7.5	54161100\\
7.51666666666667	53627400\\
7.53333333333333	54600100\\
7.55	55187500\\
7.56666666666667	55077900\\
7.58333333333333	54421000\\
7.6	54807200\\
7.61666666666667	55331800\\
7.63333333333333	56150900\\
7.65	57249000\\
7.66666666666667	60609000\\
7.68333333333333	60893000\\
7.7	62849900\\
7.71666666666667	62454900\\
7.73333333333333	62966400\\
7.75	65149800\\
7.76666666666667	65066200\\
7.78333333333333	65947100\\
7.8	67746700\\
7.81666666666667	65819900\\
7.83333333333333	65754200\\
7.85	64228400\\
7.86666666666667	63355400\\
7.88333333333333	61270400\\
7.9	62542900\\
7.91666666666667	62092500\\
7.93333333333333	63724400\\
7.95	63065800\\
7.96666666666667	64989200\\
7.98333333333333	68324700\\
8	67237800\\
8.01666666666667	66981100\\
8.03333333333333	70235500\\
8.05	69045800\\
8.06666666666667	70192000\\
8.08333333333333	69729500\\
8.1	70326100\\
8.11666666666667	71214900\\
8.13333333333333	70578100\\
8.15	70636500\\
8.16666666666667	69051700\\
8.18333333333333	70260400\\
8.2	68461600\\
8.21666666666667	69675700\\
8.23333333333333	70988500\\
8.25	72686300\\
8.26666666666667	72319800\\
8.28333333333333	71358000\\
8.3	71318500\\
8.31666666666667	68764700\\
8.33333333333333	67612700\\
8.35	65374300\\
8.36666666666667	64761800\\
8.38333333333333	65618800\\
8.4	65373100\\
8.41666666666667	68370500\\
8.43333333333333	74201800\\
8.45	74339300\\
8.46666666666667	73529200\\
8.48333333333333	69065200\\
8.5	67744000\\
8.51666666666667	66776000\\
8.53333333333333	64308400\\
8.55	66134400\\
8.56666666666667	66557300\\
8.58333333333333	65872200\\
8.6	63760600\\
8.61666666666667	62789600\\
8.63333333333333	60283200\\
8.65	60539200\\
8.66666666666667	58852000\\
8.68333333333333	63246700\\
8.7	64186500\\
8.71666666666667	63475900\\
8.73333333333333	63584500\\
8.75	63678400\\
8.76666666666667	64575500\\
8.78333333333333	66584700\\
8.8	71226300\\
8.81666666666667	72352500\\
8.83333333333333	70456800\\
8.85	69587300\\
8.86666666666667	69552600\\
8.88333333333333	70750800\\
8.9	68611700\\
8.91666666666667	70020600\\
8.93333333333333	68693300\\
8.95	67844200\\
8.96666666666667	70235000\\
8.98333333333333	71090200\\
9	71462400\\
9.01666666666667	71943200\\
9.03333333333333	72137500\\
9.05	69621000\\
9.06666666666667	68387200\\
9.08333333333333	66043000\\
9.1	64725200\\
9.11666666666667	63305900\\
9.13333333333333	59551000\\
9.15	58154500\\
9.16666666666667	56465500\\
9.18333333333333	56183900\\
9.2	55516100\\
9.21666666666667	55851900\\
9.23333333333333	55743400\\
9.25	57126200\\
9.26666666666667	55186300\\
9.28333333333333	54810000\\
9.3	54208200\\
9.31666666666667	53905800\\
9.33333333333333	53648100\\
9.35	53318900\\
9.36666666666667	54595400\\
9.38333333333333	54926500\\
9.4	55142800\\
9.41666666666667	56475900\\
9.43333333333333	57762600\\
9.45	59512500\\
9.46666666666667	60165100\\
9.48333333333333	60815100\\
9.5	60728100\\
9.51666666666667	60112000\\
9.53333333333333	60486100\\
9.55	60380100\\
9.56666666666667	60521800\\
9.58333333333333	60055000\\
9.6	61060800\\
9.61666666666667	59852800\\
9.63333333333333	58990800\\
9.65	59159000\\
9.66666666666667	59718100\\
9.68333333333333	58578500\\
9.7	58043900\\
9.71666666666667	57810400\\
9.73333333333333	59880500\\
9.75	60032200\\
9.76666666666667	62379300\\
9.78333333333333	62717400\\
9.8	61888000\\
9.81666666666667	62453300\\
9.83333333333333	63540700\\
9.85	65086700\\
9.86666666666667	65973400\\
9.88333333333333	64996400\\
9.9	64422000\\
9.91666666666667	63989000\\
9.93333333333333	64321500\\
9.95	60135600\\
9.96666666666667	62158200\\
9.98333333333333	63923300\\
10	60137600\\
10.0166666666667	62618400\\
10.0333333333333	61575800\\
10.05	61033100\\
10.0666666666667	59372100\\
10.0833333333333	58471700\\
10.1	56792300\\
10.1166666666667	53545400\\
10.1333333333333	56979500\\
10.15	56096300\\
10.1666666666667	55942500\\
10.1833333333333	55610500\\
10.2	54152000\\
10.2166666666667	54547700\\
10.2333333333333	54945000\\
10.25	54386600\\
10.2666666666667	54995200\\
10.2833333333333	53904500\\
10.3	53864600\\
10.3166666666667	53877900\\
10.3333333333333	55005200\\
10.35	53503700\\
10.3666666666667	53275400\\
10.3833333333333	54403800\\
10.4	56202500\\
10.4166666666667	57921300\\
10.4333333333333	56068300\\
10.45	53781100\\
10.4666666666667	55914600\\
10.4833333333333	56444200\\
10.5	55358800\\
10.5166666666667	56425000\\
10.5333333333333	55772200\\
10.55	56192700\\
10.5666666666667	55635900\\
10.5833333333333	56940000\\
10.6	56600100\\
10.6166666666667	56230600\\
10.6333333333333	55614900\\
10.65	55745100\\
10.6666666666667	58461800\\
10.6833333333333	56151700\\
10.7	55030200\\
10.7166666666667	55098000\\
10.7333333333333	54460500\\
10.75	53590200\\
10.7666666666667	53896700\\
10.7833333333333	53582400\\
10.8	50750300\\
10.8166666666667	51347600\\
10.8333333333333	50872400\\
10.85	51445200\\
10.8666666666667	52709000\\
10.8833333333333	52117300\\
10.9	52234800\\
10.9166666666667	53164100\\
10.9333333333333	52601700\\
10.95	52100900\\
10.9666666666667	51372100\\
10.9833333333333	50898200\\
11	52355200\\
11.0166666666667	51316400\\
11.0333333333333	52355000\\
11.05	51787600\\
11.0666666666667	51858800\\
11.0833333333333	50183900\\
11.1	51140700\\
11.1166666666667	50636200\\
11.1333333333333	52082500\\
11.15	51914400\\
11.1666666666667	54314500\\
11.1833333333333	55250600\\
11.2	55675000\\
11.2166666666667	56209600\\
11.2333333333333	59112000\\
11.25	57277100\\
11.2666666666667	58810700\\
11.2833333333333	59282200\\
11.3	59357800\\
11.3166666666667	57388200\\
11.3333333333333	57841600\\
11.35	58623700\\
11.3666666666667	58953700\\
11.3833333333333	58455300\\
11.4	58116200\\
11.4166666666667	59218100\\
11.4333333333333	60366100\\
11.45	60605000\\
11.4666666666667	59672400\\
11.4833333333333	59215900\\
11.5	58455900\\
11.5166666666667	59601500\\
11.5333333333333	59981400\\
11.55	60899600\\
11.5666666666667	61856400\\
11.5833333333333	59779700\\
11.6	59528700\\
11.6166666666667	59734700\\
11.6333333333333	59656000\\
11.65	60112500\\
11.6666666666667	59431200\\
11.6833333333333	59098800\\
11.7	59476100\\
11.7166666666667	58469900\\
11.7333333333333	57822400\\
11.75	58089700\\
11.7666666666667	59064800\\
11.7833333333333	59281700\\
11.8	58785500\\
11.8166666666667	59148300\\
11.8333333333333	58054100\\
11.85	57142800\\
11.8666666666667	57747600\\
11.8833333333333	57266300\\
11.9	58078100\\
11.9166666666667	57735300\\
11.9333333333333	57591700\\
11.95	58403200\\
11.9666666666667	57160500\\
11.9833333333333	57271400\\
12	58958500\\
12.0166666666667	59449800\\
12.0333333333333	59532300\\
12.05	58800600\\
12.0666666666667	59441100\\
12.0833333333333	59792500\\
12.1	61205200\\
12.1166666666667	59831900\\
12.1333333333333	60761200\\
12.15	62000400\\
12.1666666666667	60586800\\
12.1833333333333	61131700\\
12.2	61007700\\
12.2166666666667	59952500\\
12.2333333333333	60097000\\
12.25	59683700\\
12.2666666666667	58625800\\
12.2833333333333	57327200\\
12.3	57620200\\
12.3166666666667	57355800\\
12.3333333333333	55481500\\
12.35	56458600\\
12.3666666666667	55377300\\
12.3833333333333	54693600\\
12.4	52938000\\
12.4166666666667	53947400\\
12.4333333333333	53597400\\
12.45	53257800\\
12.4666666666667	53782800\\
12.4833333333333	55570200\\
12.5	54065200\\
12.5166666666667	54887500\\
12.5333333333333	55613900\\
12.55	56078700\\
12.5666666666667	57505700\\
12.5833333333333	58940400\\
12.6	59046400\\
12.6166666666667	59555500\\
12.6333333333333	59309600\\
12.65	59396400\\
12.6666666666667	60296300\\
12.6833333333333	60033600\\
12.7	59081600\\
12.7166666666667	59003800\\
12.7333333333333	59277300\\
12.75	58035100\\
12.7666666666667	57906600\\
12.7833333333333	58315500\\
12.8	55276400\\
12.8166666666667	55209800\\
12.8333333333333	55465700\\
12.85	55632700\\
12.8666666666667	55810000\\
12.8833333333333	56390000\\
12.9	56181500\\
12.9166666666667	55743800\\
12.9333333333333	56961100\\
12.95	57322200\\
12.9666666666667	55901000\\
12.9833333333333	54039400\\
13	55072600\\
13.0166666666667	56100800\\
13.0333333333333	54837000\\
13.05	54898900\\
13.0666666666667	55520900\\
13.0833333333333	55511900\\
13.1	54599700\\
13.1166666666667	55132700\\
13.1333333333333	56090200\\
13.15	56680200\\
13.1666666666667	58002600\\
13.1833333333333	58544900\\
13.2	58334200\\
13.2166666666667	58591600\\
13.2333333333333	58066700\\
13.25	58792300\\
13.2666666666667	59494800\\
13.2833333333333	59143800\\
13.3	60484700\\
13.3166666666667	60165300\\
13.3333333333333	61448300\\
13.35	59980100\\
13.3666666666667	60298000\\
13.3833333333333	59563300\\
13.4	58572800\\
13.4166666666667	57944100\\
13.4333333333333	58505900\\
13.45	58705000\\
13.4666666666667	58198900\\
13.4833333333333	57863100\\
13.5	56768200\\
13.5166666666667	56670900\\
13.5333333333333	55009800\\
13.55	56342600\\
13.5666666666667	56428200\\
13.5833333333333	57515200\\
13.6	57897300\\
13.6166666666667	58361700\\
13.6333333333333	57730200\\
13.65	58061400\\
13.6666666666667	59374200\\
13.6833333333333	60113300\\
13.7	60764000\\
13.7166666666667	60776200\\
13.7333333333333	60945200\\
13.75	60912100\\
13.7666666666667	60730100\\
13.7833333333333	60223900\\
13.8	60771100\\
13.8166666666667	59858200\\
13.8333333333333	60346400\\
13.85	58770600\\
13.8666666666667	57968900\\
13.8833333333333	57264800\\
13.9	56022900\\
13.9166666666667	55358000\\
13.9333333333333	55488300\\
13.95	54192700\\
13.9666666666667	53991200\\
13.9833333333333	55529100\\
14	56957700\\
14.0166666666667	58364500\\
14.0333333333333	59300900\\
14.05	59782600\\
14.0666666666667	59671500\\
14.0833333333333	58968800\\
14.1	59126700\\
14.1166666666667	58226900\\
14.1333333333333	56505900\\
14.15	56802100\\
14.1666666666667	56724700\\
14.1833333333333	56691900\\
14.2	56423300\\
14.2166666666667	57477600\\
14.2333333333333	57111200\\
14.25	57052400\\
14.2666666666667	56901500\\
14.2833333333333	58339800\\
14.3	60243100\\
14.3166666666667	58626800\\
14.3333333333333	60683700\\
14.35	60493900\\
14.3666666666667	59261400\\
14.3833333333333	57680900\\
14.4	57719700\\
14.4166666666667	57083300\\
14.4333333333333	57228900\\
14.45	56641400\\
14.4666666666667	56079300\\
14.4833333333333	57689600\\
14.5	56828400\\
14.5166666666667	58029200\\
14.5333333333333	59349900\\
14.55	59400400\\
14.5666666666667	60456600\\
14.5833333333333	62224600\\
14.6	60180400\\
14.6166666666667	61559700\\
14.6333333333333	62102200\\
14.65	59632600\\
14.6666666666667	59572000\\
14.6833333333333	56959700\\
14.7	57650800\\
14.7166666666667	54876300\\
14.7333333333333	52862900\\
14.75	54689000\\
14.7666666666667	53320400\\
14.7833333333333	54950600\\
14.8	53825600\\
14.8166666666667	54166400\\
14.8333333333333	55662000\\
14.85	54090500\\
14.8666666666667	54291300\\
14.8833333333333	57901100\\
14.9	55875200\\
14.9166666666667	57674400\\
14.9333333333333	58731900\\
14.95	58257500\\
14.9666666666667	57280800\\
14.9833333333333	59245600\\
15	58761300\\
15.0166666666667	59330000\\
15.0333333333333	61250800\\
15.05	59688900\\
15.0666666666667	59177500\\
15.0833333333333	58775500\\
15.1	59396200\\
15.1166666666667	59058900\\
15.1333333333333	59997400\\
15.15	58601400\\
15.1666666666667	59192900\\
15.1833333333333	59186900\\
15.2	58515500\\
15.2166666666667	59391300\\
15.2333333333333	60270800\\
15.25	60040100\\
15.2666666666667	59538000\\
15.2833333333333	60062800\\
15.3	60432900\\
15.3166666666667	60995100\\
15.3333333333333	59283600\\
15.35	59725200\\
15.3666666666667	59387700\\
15.3833333333333	59537500\\
15.4	59743400\\
15.4166666666667	59292800\\
15.4333333333333	60292000\\
15.45	60059000\\
15.4666666666667	60342600\\
15.4833333333333	60720400\\
15.5	61413200\\
15.5166666666667	61813700\\
15.5333333333333	61237100\\
15.55	60664500\\
15.5666666666667	62352000\\
15.5833333333333	61706000\\
15.6	61411300\\
15.6166666666667	60488600\\
15.6333333333333	61046200\\
15.65	61045600\\
15.6666666666667	61010400\\
15.6833333333333	60855200\\
15.7	61444500\\
15.7166666666667	60358200\\
15.7333333333333	61145500\\
15.75	60283600\\
15.7666666666667	61492400\\
15.7833333333333	60117200\\
15.8	61606800\\
15.8166666666667	60065900\\
15.8333333333333	61541500\\
15.85	59309100\\
15.8666666666667	60256000\\
15.8833333333333	58644400\\
15.9	58392200\\
15.9166666666667	59783600\\
15.9333333333333	56988700\\
15.95	57437700\\
15.9666666666667	56359800\\
15.9833333333333	55019600\\
16	54860000\\
16.0166666666667	54729000\\
16.0333333333333	55653400\\
16.05	54191000\\
16.0666666666667	56527400\\
16.0833333333333	54475500\\
16.1	57844700\\
16.1166666666667	57661800\\
16.1333333333333	57858200\\
16.15	59041200\\
16.1666666666667	58375000\\
16.1833333333333	59751500\\
16.2	61611800\\
16.2166666666667	59121100\\
16.2333333333333	59981700\\
16.25	60355700\\
16.2666666666667	61431600\\
16.2833333333333	62244300\\
16.3	62095200\\
16.3166666666667	62610800\\
16.3333333333333	61173500\\
16.35	59403000\\
16.3666666666667	58718900\\
16.3833333333333	57587600\\
16.4	58167200\\
16.4166666666667	58434700\\
16.4333333333333	59729400\\
16.45	57425500\\
16.4666666666667	57975800\\
16.4833333333333	57887800\\
16.5	58305500\\
16.5166666666667	57688500\\
16.5333333333333	58342600\\
16.55	56418100\\
16.5666666666667	57687100\\
16.5833333333333	58592100\\
16.6	56193600\\
16.6166666666667	60319300\\
16.6333333333333	55735500\\
16.65	56793500\\
16.6666666666667	58334800\\
16.6833333333333	56823600\\
16.7	55049400\\
16.7166666666667	57983400\\
16.7333333333333	55864200\\
16.75	57501600\\
16.7666666666667	55945300\\
16.7833333333333	57013200\\
16.8	57892300\\
16.8166666666667	56390700\\
16.8333333333333	56989200\\
16.85	57163000\\
16.8666666666667	57212700\\
16.8833333333333	57375000\\
16.9	57063700\\
16.9166666666667	56619400\\
16.9333333333333	57277900\\
16.95	58602200\\
16.9666666666667	57614400\\
16.9833333333333	56177600\\
17	57969100\\
17.0166666666667	59965100\\
17.0333333333333	58641700\\
17.05	59162600\\
17.0666666666667	59053900\\
17.0833333333333	58552300\\
17.1	58289200\\
17.1166666666667	58742900\\
17.1333333333333	59130800\\
17.15	58800200\\
17.1666666666667	58632300\\
17.1833333333333	58812900\\
17.2	59674800\\
17.2166666666667	59189900\\
17.2333333333333	57920400\\
17.25	58487000\\
17.2666666666667	56089800\\
17.2833333333333	56247400\\
17.3	57417300\\
17.3166666666667	56378000\\
17.3333333333333	54622600\\
17.35	56906900\\
17.3666666666667	58098500\\
17.3833333333333	56663000\\
17.4	58284000\\
17.4166666666667	58097200\\
17.4333333333333	59043100\\
17.45	59790700\\
17.4666666666667	60855800\\
17.4833333333333	60675400\\
17.5	59955100\\
17.5166666666667	62189000\\
17.5333333333333	61198300\\
17.55	61214400\\
17.5666666666667	60747400\\
17.5833333333333	60141600\\
17.6	60810900\\
17.6166666666667	60335300\\
17.6333333333333	59949300\\
17.65	58481100\\
17.6666666666667	58628300\\
17.6833333333333	58100000\\
17.7	58171700\\
17.7166666666667	56306300\\
17.7333333333333	58879400\\
17.75	56982100\\
17.7666666666667	57336800\\
17.7833333333333	55996700\\
17.8	55910300\\
17.8166666666667	55188300\\
17.8333333333333	51684900\\
17.85	54085000\\
17.8666666666667	54067100\\
17.8833333333333	54630900\\
17.9	56002400\\
17.9166666666667	55125000\\
17.9333333333333	57217700\\
17.95	58005200\\
17.9666666666667	58154800\\
17.9833333333333	60637900\\
18	59729500\\
18.0166666666667	60581600\\
18.0333333333333	59917600\\
18.05	58005500\\
18.0666666666667	58136200\\
18.0833333333333	57359700\\
18.1	56020200\\
18.1166666666667	55956300\\
18.1333333333333	53509800\\
18.15	53883800\\
18.1666666666667	52480800\\
18.1833333333333	53160900\\
18.2	53557100\\
18.2166666666667	54646400\\
18.2333333333333	54696500\\
18.25	54782400\\
18.2666666666667	55450000\\
18.2833333333333	57477100\\
18.3	56785400\\
18.3166666666667	58000900\\
18.3333333333333	59493700\\
18.35	59785500\\
18.3666666666667	60185000\\
18.3833333333333	58865800\\
18.4	58816500\\
18.4166666666667	60200300\\
18.4333333333333	59250700\\
18.45	57927000\\
18.4666666666667	58457400\\
18.4833333333333	59202000\\
18.5	56907500\\
18.5166666666667	56268200\\
18.5333333333333	56134800\\
18.55	54257700\\
18.5666666666667	54827300\\
18.5833333333333	54622200\\
18.6	53159700\\
18.6166666666667	54025400\\
18.6333333333333	54039200\\
18.65	53775200\\
18.6666666666667	52876900\\
18.6833333333333	53504500\\
18.7	53772500\\
18.7166666666667	53942000\\
18.7333333333333	52519400\\
18.75	53162800\\
18.7666666666667	53368800\\
18.7833333333333	53729000\\
18.8	53078100\\
18.8166666666667	53398900\\
18.8333333333333	53894100\\
18.85	53257900\\
18.8666666666667	53616500\\
18.8833333333333	53838100\\
18.9	53326700\\
18.9166666666667	53600400\\
18.9333333333333	55629000\\
18.95	53583200\\
18.9666666666667	53837100\\
18.9833333333333	52412600\\
19	53476900\\
19.0166666666667	52851800\\
19.0333333333333	53311900\\
19.05	52827600\\
19.0666666666667	52739800\\
19.0833333333333	52779700\\
19.1	51739600\\
19.1166666666667	52179700\\
19.1333333333333	54340600\\
19.15	53508200\\
19.1666666666667	53772500\\
19.1833333333333	54210900\\
19.2	54437400\\
19.2166666666667	53071500\\
19.2333333333333	52790500\\
19.25	52292600\\
19.2666666666667	53460500\\
19.2833333333333	53516800\\
19.3	54157500\\
19.3166666666667	54123000\\
19.3333333333333	53507400\\
19.35	53473900\\
19.3666666666667	52457200\\
19.3833333333333	52304400\\
19.4	53615200\\
19.4166666666667	54060900\\
19.4333333333333	54637600\\
19.45	53900600\\
19.4666666666667	52448800\\
19.4833333333333	52021400\\
19.5	54606200\\
19.5166666666667	52696600\\
19.5333333333333	53967200\\
19.55	53346900\\
19.5666666666667	53159200\\
19.5833333333333	53017900\\
19.6	53887100\\
19.6166666666667	53489000\\
19.6333333333333	54198300\\
19.65	54988400\\
19.6666666666667	53341100\\
19.6833333333333	54166300\\
19.7	54758900\\
19.7166666666667	55027100\\
19.7333333333333	55202900\\
19.75	56902800\\
19.7666666666667	58337300\\
19.7833333333333	57222100\\
19.8	58837000\\
19.8166666666667	56802200\\
19.8333333333333	57943000\\
19.85	57026000\\
19.8666666666667	56273500\\
19.8833333333333	56161400\\
19.9	54734500\\
19.9166666666667	54069600\\
19.9333333333333	53687200\\
19.95	52615300\\
19.9666666666667	52182100\\
19.9833333333333	52152300\\
20	51500500\\
20.0166666666667	51475200\\
20.0333333333333	51865800\\
20.05	51471500\\
20.0666666666667	52580300\\
20.0833333333333	52372600\\
20.1	53717700\\
20.1166666666667	53257500\\
20.1333333333333	54541000\\
20.15	55557500\\
20.1666666666667	55720200\\
20.1833333333333	56923900\\
20.2	54899800\\
20.2166666666667	56853800\\
20.2333333333333	57155100\\
20.25	54525200\\
20.2666666666667	55956700\\
20.2833333333333	55856200\\
20.3	55869300\\
20.3166666666667	55621200\\
20.3333333333333	55157300\\
20.35	54389100\\
20.3666666666667	53827300\\
20.3833333333333	52166500\\
20.4	52712800\\
20.4166666666667	52214500\\
20.4333333333333	51876900\\
20.45	51532000\\
20.4666666666667	52830800\\
20.4833333333333	51329700\\
20.5	51921300\\
20.5166666666667	51664800\\
20.5333333333333	53822100\\
20.55	53496300\\
20.5666666666667	54984600\\
20.5833333333333	55796900\\
20.6	56735100\\
20.6166666666667	55551700\\
20.6333333333333	56588000\\
20.65	56599800\\
20.6666666666667	56405600\\
20.6833333333333	55648100\\
20.7	56191900\\
20.7166666666667	55494400\\
20.7333333333333	54342300\\
20.75	54636000\\
20.7666666666667	52756700\\
20.7833333333333	52934600\\
20.8	51765500\\
20.8166666666667	51115100\\
20.8333333333333	51873300\\
20.85	52297900\\
20.8666666666667	53100300\\
20.8833333333333	53032000\\
20.9	53691900\\
20.9166666666667	54135600\\
20.9333333333333	55283400\\
20.95	55083400\\
20.9666666666667	57915400\\
20.9833333333333	58253500\\
21	56930000\\
21.0166666666667	56958100\\
21.0333333333333	57264500\\
21.05	57465900\\
21.0666666666667	58283300\\
21.0833333333333	56771600\\
21.1	56630900\\
21.1166666666667	56610700\\
21.1333333333333	54841300\\
21.15	55206400\\
21.1666666666667	54528300\\
21.1833333333333	53352900\\
21.2	53922600\\
21.2166666666667	53444700\\
21.2333333333333	52243500\\
21.25	52789200\\
21.2666666666667	53242800\\
21.2833333333333	53616800\\
21.3	52976900\\
21.3166666666667	52497000\\
21.3333333333333	52863400\\
21.35	53344400\\
21.3666666666667	51957900\\
21.3833333333333	52285000\\
21.4	53630300\\
21.4166666666667	53141400\\
21.4333333333333	53145300\\
21.45	53851900\\
21.4666666666667	54037300\\
21.4833333333333	53332100\\
21.5	55070100\\
21.5166666666667	55094400\\
21.5333333333333	55615400\\
21.55	57313600\\
21.5666666666667	56366600\\
21.5833333333333	57925500\\
21.6	58301200\\
21.6166666666667	58608800\\
21.6333333333333	57129300\\
21.65	57154600\\
21.6666666666667	57011100\\
21.6833333333333	55436300\\
21.7	55599700\\
21.7166666666667	54130400\\
21.7333333333333	54151700\\
21.75	52690200\\
21.7666666666667	54289800\\
21.7833333333333	53778100\\
21.8	53493600\\
21.8166666666667	54507800\\
21.8333333333333	56198400\\
21.85	55900200\\
21.8666666666667	56945300\\
21.8833333333333	57667400\\
21.9	58372200\\
21.9166666666667	58173400\\
21.9333333333333	56468700\\
21.95	56238000\\
21.9666666666667	54778000\\
21.9833333333333	53608700\\
22	52749900\\
22.0166666666667	53614300\\
22.0333333333333	53305400\\
22.05	55203700\\
22.0666666666667	56471800\\
22.0833333333333	57173100\\
22.1	57729800\\
22.1166666666667	56342700\\
22.1333333333333	58224700\\
22.15	58259100\\
22.1666666666667	58234700\\
22.1833333333333	57144700\\
22.2	56609600\\
22.2166666666667	55055800\\
22.2333333333333	55709000\\
22.25	53623600\\
22.2666666666667	53517300\\
22.2833333333333	53317300\\
22.3	54193200\\
22.3166666666667	54252900\\
22.3333333333333	54004800\\
22.35	54277100\\
22.3666666666667	55851000\\
22.3833333333333	55265700\\
22.4	57048800\\
22.4166666666667	56658700\\
22.4333333333333	57357200\\
22.45	57949400\\
22.4666666666667	58631700\\
22.4833333333333	59703300\\
22.5	59762600\\
22.5166666666667	59742000\\
22.5333333333333	58935300\\
22.55	58829400\\
22.5666666666667	57895300\\
22.5833333333333	59126600\\
22.6	56447500\\
22.6166666666667	58152500\\
22.6333333333333	55160100\\
22.65	56345900\\
22.6666666666667	55326700\\
22.6833333333333	55656100\\
22.7	56403900\\
22.7166666666667	56215700\\
22.7333333333333	56868900\\
22.75	57266900\\
22.7666666666667	57856100\\
22.7833333333333	59408400\\
22.8	61010100\\
22.8166666666667	60096400\\
22.8333333333333	61071300\\
22.85	60695000\\
22.8666666666667	59171200\\
22.8833333333333	59670400\\
22.9	58973400\\
22.9166666666667	57558300\\
22.9333333333333	57431100\\
22.95	56585700\\
22.9666666666667	55959600\\
22.9833333333333	55723100\\
23	55665500\\
23.0166666666667	56000000\\
23.0333333333333	56137500\\
23.05	58324100\\
23.0666666666667	57068100\\
23.0833333333333	58109000\\
23.1	59603000\\
23.1166666666667	59322800\\
23.1333333333333	61162300\\
23.15	59286300\\
23.1666666666667	60170800\\
23.1833333333333	59841800\\
23.2	58848400\\
23.2166666666667	58320100\\
23.2333333333333	57636600\\
23.25	56281100\\
23.2666666666667	55365500\\
23.2833333333333	54975900\\
23.3	53897100\\
23.3166666666667	55522300\\
23.3333333333333	55590700\\
23.35	56453000\\
23.3666666666667	56506300\\
23.3833333333333	57646900\\
23.4	56543700\\
23.4166666666667	58659700\\
23.4333333333333	59486000\\
23.45	59694800\\
23.4666666666667	59231500\\
23.4833333333333	59766100\\
23.5	60225600\\
23.5166666666667	60593800\\
23.5333333333333	60954300\\
23.55	59465900\\
23.5666666666667	59352400\\
23.5833333333333	57704100\\
23.6	59453500\\
23.6166666666667	57631100\\
23.6333333333333	57818200\\
23.65	57326100\\
23.6666666666667	56432700\\
23.6833333333333	56299600\\
23.7	56037400\\
23.7166666666667	56650000\\
23.7333333333333	56746800\\
23.75	57243200\\
23.7666666666667	58754100\\
23.7833333333333	58750400\\
23.8	60442100\\
23.8166666666667	60368200\\
23.8333333333333	60228000\\
23.85	61900400\\
23.8666666666667	61118300\\
23.8833333333333	61002100\\
23.9	59833800\\
23.9166666666667	58564400\\
23.9333333333333	59167400\\
23.95	58650700\\
23.9666666666667	58685900\\
23.9833333333333	56606500\\
24	57567700\\
24.0166666666667	56229500\\
24.0333333333333	57186500\\
24.05	55991100\\
24.0666666666667	55941000\\
24.0833333333333	56535900\\
24.1	56907000\\
24.1166666666667	57571700\\
24.1333333333333	56731500\\
24.15	56714100\\
24.1666666666667	58165600\\
24.1833333333333	57513400\\
24.2	57275500\\
24.2166666666667	58927800\\
24.2333333333333	59105800\\
24.25	59156600\\
24.2666666666667	60249700\\
24.2833333333333	59686100\\
24.3	60696500\\
24.3166666666667	61411400\\
24.3333333333333	61911500\\
24.35	62935000\\
24.3666666666667	62787100\\
24.3833333333333	61687100\\
24.4	61442200\\
24.4166666666667	61079900\\
24.4333333333333	60721700\\
24.45	58643600\\
24.4666666666667	58100300\\
24.4833333333333	59767900\\
24.5	57922600\\
24.5166666666667	57555800\\
24.5333333333333	58249300\\
24.55	58239400\\
24.5666666666667	58804700\\
24.5833333333333	59416500\\
24.6	60821900\\
24.6166666666667	61177400\\
24.6333333333333	63062100\\
24.65	64430600\\
24.6666666666667	61748500\\
24.6833333333333	63362600\\
24.7	63726700\\
24.7166666666667	62835600\\
24.7333333333333	61941900\\
24.75	60869600\\
24.7666666666667	61014000\\
24.7833333333333	60547900\\
24.8	60754900\\
24.8166666666667	59934700\\
24.8333333333333	58907700\\
24.85	58724100\\
24.8666666666667	59484900\\
24.8833333333333	59782600\\
24.9	58690200\\
24.9166666666667	59929000\\
24.9333333333333	59082900\\
24.95	60799600\\
24.9666666666667	60494700\\
24.9833333333333	61288600\\
25	63210900\\
25.0166666666667	64070400\\
25.0333333333333	64515800\\
25.05	63381100\\
25.0666666666667	63973600\\
25.0833333333333	64099600\\
25.1	63669000\\
25.1166666666667	62967200\\
25.1333333333333	62866400\\
25.15	61362700\\
25.1666666666667	60811900\\
25.1833333333333	59289800\\
25.2	58460600\\
25.2166666666667	58385900\\
25.2333333333333	59430200\\
25.25	59894900\\
25.2666666666667	60378200\\
25.2833333333333	60911700\\
25.3	61128500\\
25.3166666666667	62399000\\
25.3333333333333	63184200\\
25.35	64141700\\
25.3666666666667	62126300\\
25.3833333333333	63519900\\
25.4	63238900\\
25.4166666666667	62953600\\
25.4333333333333	60276900\\
25.45	59274700\\
25.4666666666667	61255200\\
25.4833333333333	60287300\\
25.5	58382200\\
25.5166666666667	59220400\\
25.5333333333333	59567400\\
25.55	62002300\\
25.5666666666667	61142500\\
25.5833333333333	63019300\\
25.6	64542100\\
25.6166666666667	64204200\\
25.6333333333333	64861800\\
25.65	63871000\\
25.6666666666667	64388100\\
25.6833333333333	63646600\\
25.7	62334800\\
25.7166666666667	61712100\\
25.7333333333333	61976400\\
25.75	61005000\\
25.7666666666667	61515900\\
25.7833333333333	60530600\\
25.8	60556900\\
25.8166666666667	60473200\\
25.8333333333333	60933800\\
25.85	60719000\\
25.8666666666667	60709700\\
25.8833333333333	61849500\\
25.9	61395700\\
25.9166666666667	64918800\\
25.9333333333333	63420300\\
25.95	64074100\\
25.9666666666667	64728100\\
25.9833333333333	65869900\\
26	66118600\\
26.0166666666667	65775300\\
26.0333333333333	65698400\\
26.05	66452000\\
26.0666666666667	66538900\\
26.0833333333333	64671500\\
26.1	65347700\\
26.1166666666667	63703000\\
26.1333333333333	62497500\\
26.15	62517500\\
26.1666666666667	62760200\\
26.1833333333333	61542500\\
26.2	62876400\\
26.2166666666667	61847200\\
26.2333333333333	61758800\\
26.25	64939300\\
26.2666666666667	63047600\\
26.2833333333333	65690000\\
26.3	64931500\\
26.3166666666667	67194600\\
26.3333333333333	67366500\\
26.35	67420500\\
26.3666666666667	67489500\\
26.3833333333333	67475200\\
26.4	67319500\\
26.4166666666667	68022300\\
26.4333333333333	66872700\\
26.45	67193700\\
26.4666666666667	67367000\\
26.4833333333333	66782100\\
26.5	66916700\\
26.5166666666667	66195700\\
26.5333333333333	66496800\\
26.55	65818800\\
26.5666666666667	65059600\\
26.5833333333333	65271400\\
26.6	64089100\\
26.6166666666667	64810500\\
26.6333333333333	63943600\\
26.65	63579900\\
26.6666666666667	63507800\\
26.6833333333333	63207200\\
26.7	63691500\\
26.7166666666667	63187200\\
26.7333333333333	63445200\\
26.75	63367000\\
26.7666666666667	63597200\\
26.7833333333333	64349000\\
26.8	65970700\\
26.8166666666667	65168300\\
26.8333333333333	66030100\\
26.85	66605900\\
26.8666666666667	66782300\\
26.8833333333333	67307900\\
26.9	66434900\\
26.9166666666667	67911800\\
26.9333333333333	68719200\\
26.95	68207200\\
26.9666666666667	67761200\\
26.9833333333333	68741700\\
27	69875500\\
27.0166666666667	68066200\\
27.0333333333333	69292700\\
27.05	67563900\\
27.0666666666667	68572500\\
27.0833333333333	68279400\\
27.1	67877300\\
27.1166666666667	68963700\\
27.1333333333333	68001600\\
27.15	67951600\\
27.1666666666667	68325400\\
27.1833333333333	67565800\\
27.2	67915800\\
27.2166666666667	67076800\\
27.2333333333333	66971800\\
27.25	67036500\\
27.2666666666667	66182100\\
27.2833333333333	66923700\\
27.3	65716300\\
27.3166666666667	66162500\\
27.3333333333333	65740000\\
27.35	64890300\\
27.3666666666667	65385200\\
27.3833333333333	64340000\\
27.4	64163600\\
27.4166666666667	65950000\\
27.4333333333333	65293200\\
27.45	64814000\\
27.4666666666667	64780900\\
27.4833333333333	66013800\\
27.5	65417100\\
27.5166666666667	67047900\\
27.5333333333333	66900700\\
27.55	68717000\\
27.5666666666667	71229900\\
27.5833333333333	70342900\\
27.6	70986200\\
27.6166666666667	71047200\\
27.6333333333333	71683600\\
27.65	70263100\\
27.6666666666667	70684900\\
27.6833333333333	69389600\\
27.7	68799100\\
27.7166666666667	68197500\\
27.7333333333333	67038200\\
27.75	67004100\\
27.7666666666667	65944900\\
27.7833333333333	68149000\\
27.8	68107000\\
27.8166666666667	69001100\\
27.8333333333333	68446200\\
27.85	68966600\\
27.8666666666667	67968600\\
27.8833333333333	68960200\\
27.9	69198500\\
27.9166666666667	69324000\\
27.9333333333333	69960400\\
27.95	70499400\\
27.9666666666667	71633200\\
27.9833333333333	71182200\\
28	72352400\\
28.0166666666667	72299800\\
28.0333333333333	72857300\\
28.05	71941300\\
28.0666666666667	74017900\\
28.0833333333333	72446300\\
28.1	73969000\\
28.1166666666667	72483400\\
28.1333333333333	73175400\\
28.15	72401100\\
28.1666666666667	72985100\\
28.1833333333333	71513100\\
28.2	71823400\\
28.2166666666667	70442900\\
28.2333333333333	70390400\\
28.25	69743600\\
28.2666666666667	70586700\\
28.2833333333333	69089400\\
28.3	69048700\\
28.3166666666667	68836500\\
28.3333333333333	69335900\\
28.35	68469600\\
28.3666666666667	70041900\\
28.3833333333333	69680500\\
28.4	70393500\\
28.4166666666667	70051100\\
28.4333333333333	72193900\\
28.45	72676500\\
28.4666666666667	71862900\\
28.4833333333333	73074000\\
28.5	73071400\\
28.5166666666667	73308700\\
28.5333333333333	75553600\\
28.55	74540600\\
28.5666666666667	74104200\\
28.5833333333333	75797800\\
28.6	75448600\\
28.6166666666667	75860600\\
28.6333333333333	76167800\\
28.65	75469100\\
28.6666666666667	75337500\\
28.6833333333333	76031000\\
28.7	76296700\\
28.7166666666667	76512500\\
28.7333333333333	76784500\\
28.75	76489000\\
28.7666666666667	76291000\\
28.7833333333333	76346100\\
28.8	76566500\\
28.8166666666667	75375900\\
28.8333333333333	76592200\\
28.85	75646200\\
28.8666666666667	73700400\\
28.8833333333333	73645400\\
28.9	72568800\\
28.9166666666667	73041700\\
28.9333333333333	71822400\\
28.95	71439100\\
28.9666666666667	70742900\\
28.9833333333333	70052600\\
29	71040600\\
29.0166666666667	72382400\\
29.0333333333333	72686000\\
29.05	74204500\\
29.0666666666667	75306400\\
29.0833333333333	76269600\\
29.1	76562200\\
29.1166666666667	76106200\\
29.1333333333333	76131100\\
29.15	76979800\\
29.1666666666667	75074400\\
29.1833333333333	74153100\\
29.2	74283500\\
29.2166666666667	71122900\\
29.2333333333333	72171000\\
29.25	72539400\\
29.2666666666667	71728500\\
29.2833333333333	72688100\\
29.3	73807300\\
29.3166666666667	75653900\\
29.3333333333333	76704000\\
29.35	77254700\\
29.3666666666667	76754400\\
29.3833333333333	75321800\\
29.4	75053600\\
29.4166666666667	72745800\\
29.4333333333333	71670100\\
29.45	71371300\\
29.4666666666667	73831100\\
29.4833333333333	74031700\\
29.5	76307100\\
29.5166666666667	77989900\\
29.5333333333333	77146700\\
29.55	75333400\\
29.5666666666667	76085000\\
29.5833333333333	73082300\\
29.6	72123400\\
29.6166666666667	71437900\\
29.6333333333333	72705100\\
29.65	72710000\\
29.6666666666667	74989600\\
29.6833333333333	75361000\\
29.7	76277000\\
29.7166666666667	76440000\\
29.7333333333333	78148200\\
29.75	76077800\\
29.7666666666667	75099400\\
29.7833333333333	73113300\\
29.8	73335700\\
29.8166666666667	72537100\\
29.8333333333333	72903500\\
29.85	72330600\\
29.8666666666667	73696400\\
29.8833333333333	73014600\\
29.9	74636400\\
29.9166666666667	75856900\\
29.9333333333333	76978500\\
29.95	77875000\\
29.9666666666667	78092100\\
29.9833333333333	78838500\\
};

\addplot [color=mycolor4]
  table[row sep=crcr]{%
0	1986200\\
0.0166666666666667	1999790\\
0.0333333333333333	2011500\\
0.05	2021330\\
0.0666666666666667	2031660\\
0.0833333333333333	2043190\\
0.1	2051140\\
0.116666666666667	2059640\\
0.133333333333333	2067070\\
0.15	2075380\\
0.166666666666667	2086440\\
0.183333333333333	2095230\\
0.2	2104940\\
0.216666666666667	2113030\\
0.233333333333333	2121630\\
0.25	2129690\\
0.266666666666667	2137530\\
0.283333333333333	2143360\\
0.3	2150800\\
0.316666666666667	2155220\\
0.333333333333333	2159980\\
0.35	2165680\\
0.366666666666667	2171310\\
0.383333333333333	2178150\\
0.4	2184380\\
0.416666666666667	2190200\\
0.433333333333333	2196560\\
0.45	2203530\\
0.466666666666667	2210870\\
0.483333333333333	2218670\\
0.5	2226280\\
0.516666666666667	2233560\\
0.533333333333333	2240810\\
0.55	2248170\\
0.566666666666667	2255670\\
0.583333333333333	2262910\\
0.6	2269410\\
0.616666666666667	2275530\\
0.633333333333333	2281400\\
0.65	2287380\\
0.666666666666667	2292700\\
0.683333333333333	2298230\\
0.7	2304230\\
0.716666666666667	2309240\\
0.733333333333333	2314170\\
0.75	2318810\\
0.766666666666667	2323370\\
0.783333333333333	2327570\\
0.8	2331600\\
0.816666666666667	2335070\\
0.833333333333333	2338310\\
0.85	2341320\\
0.866666666666667	2344510\\
0.883333333333333	2347620\\
0.9	2350860\\
0.916666666666667	2354910\\
0.933333333333333	2358670\\
0.95	2362380\\
0.966666666666667	2366130\\
0.983333333333333	2369960\\
1	2373900\\
1.01666666666667	2377740\\
1.03333333333333	2381650\\
1.05	2385870\\
1.06666666666667	2390240\\
1.08333333333333	2394470\\
1.1	2398590\\
1.11666666666667	2403010\\
1.13333333333333	2407150\\
1.15	2411150\\
1.16666666666667	2415090\\
1.18333333333333	2419000\\
1.2	2422910\\
1.21666666666667	2426390\\
1.23333333333333	2428900\\
1.25	2428660\\
1.26666666666667	2430900\\
1.28333333333333	2433500\\
1.3	2438280\\
1.31666666666667	2442220\\
1.33333333333333	2445370\\
1.35	2448580\\
1.36666666666667	2452010\\
1.38333333333333	2455500\\
1.4	2458930\\
1.41666666666667	2462300\\
1.43333333333333	2465330\\
1.45	2468080\\
1.46666666666667	2470820\\
1.48333333333333	2473640\\
1.5	2476430\\
1.51666666666667	2479400\\
1.53333333333333	2482430\\
1.55	2485150\\
1.56666666666667	2488080\\
1.58333333333333	2491150\\
1.6	2494060\\
1.61666666666667	2496760\\
1.63333333333333	2499940\\
1.65	2502980\\
1.66666666666667	2505910\\
1.68333333333333	2509010\\
1.7	2512140\\
1.71666666666667	2514530\\
1.73333333333333	2517320\\
1.75	2518150\\
1.76666666666667	2517370\\
1.78333333333333	2520260\\
1.8	2523260\\
1.81666666666667	2526420\\
1.83333333333333	2526390\\
1.85	2525740\\
1.86666666666667	2528170\\
1.88333333333333	2530490\\
1.9	2533230\\
1.91666666666667	2535870\\
1.93333333333333	2537860\\
1.95	2540460\\
1.96666666666667	2542730\\
1.98333333333333	2545020\\
2	2547520\\
2.01666666666667	2550280\\
2.03333333333333	2552960\\
2.05	2555350\\
2.06666666666667	2557860\\
2.08333333333333	2560350\\
2.1	2563100\\
2.11666666666667	2565760\\
2.13333333333333	2568130\\
2.15	2570520\\
2.16666666666667	2435720\\
2.18333333333333	2715660\\
2.2	2729730\\
2.21666666666667	2734250\\
2.23333333333333	2738450\\
2.25	2742050\\
2.26666666666667	2745520\\
2.28333333333333	2748400\\
2.3	2750200\\
2.31666666666667	2751650\\
2.33333333333333	2753840\\
2.35	2755920\\
2.36666666666667	2757830\\
2.38333333333333	2760260\\
2.4	2762270\\
2.41666666666667	2767050\\
2.43333333333333	2773350\\
2.45	2775690\\
2.46666666666667	2777810\\
2.48333333333333	2780140\\
2.5	2782480\\
2.51666666666667	2784670\\
2.53333333333333	2786950\\
2.55	2789340\\
2.56666666666667	2791410\\
2.58333333333333	2794680\\
2.6	2796630\\
2.61666666666667	2798650\\
2.63333333333333	2800260\\
2.65	2801650\\
2.66666666666667	2803280\\
2.68333333333333	2804250\\
2.7	2805550\\
2.71666666666667	2806910\\
2.73333333333333	2807930\\
2.75	2808960\\
2.76666666666667	2810560\\
2.78333333333333	2812420\\
2.8	2814440\\
2.81666666666667	2816480\\
2.83333333333333	2818910\\
2.85	2821380\\
2.86666666666667	2823460\\
2.88333333333333	2825390\\
2.9	2827330\\
2.91666666666667	2829440\\
2.93333333333333	2831590\\
2.95	2834780\\
2.96666666666667	2837340\\
2.98333333333333	2839690\\
3	2841640\\
3.01666666666667	2843950\\
3.03333333333333	2846080\\
3.05	2849490\\
3.06666666666667	2852840\\
3.08333333333333	2854510\\
3.1	2856370\\
3.11666666666667	2858350\\
3.13333333333333	2860160\\
3.15	2862200\\
3.16666666666667	2864190\\
3.18333333333333	2866150\\
3.2	2868070\\
3.21666666666667	2870140\\
3.23333333333333	2872280\\
3.25	2874280\\
3.26666666666667	2876660\\
3.28333333333333	2879240\\
3.3	2882100\\
3.31666666666667	2884570\\
3.33333333333333	2887160\\
3.35	2889720\\
3.36666666666667	2892160\\
3.38333333333333	2894620\\
3.4	2897110\\
3.41666666666667	2899300\\
3.43333333333333	2901260\\
3.45	2903020\\
3.46666666666667	2904870\\
3.48333333333333	2906190\\
3.5	2907610\\
3.51666666666667	2906740\\
3.53333333333333	2908630\\
3.55	2910340\\
3.56666666666667	2912040\\
3.58333333333333	2913740\\
3.6	2915600\\
3.61666666666667	2917340\\
3.63333333333333	2919370\\
3.65	2921380\\
3.66666666666667	2923580\\
3.68333333333333	2925820\\
3.7	2927970\\
3.71666666666667	2929980\\
3.73333333333333	2932400\\
3.75	2934290\\
3.76666666666667	2935960\\
3.78333333333333	2937440\\
3.8	2939090\\
3.81666666666667	2940830\\
3.83333333333333	2942570\\
3.85	2944350\\
3.86666666666667	2946090\\
3.88333333333333	2947560\\
3.9	2949310\\
3.91666666666667	2951040\\
3.93333333333333	2952740\\
3.95	2954200\\
3.96666666666667	2955630\\
3.98333333333333	2956820\\
4	2958160\\
4.01666666666667	2959560\\
4.03333333333333	2961090\\
4.05	2962900\\
4.06666666666667	2964860\\
4.08333333333333	2966950\\
4.1	2969140\\
4.11666666666667	2971360\\
4.13333333333333	2973500\\
4.15	2975680\\
4.16666666666667	2977760\\
4.18333333333333	2979530\\
4.2	2981230\\
4.21666666666667	2983050\\
4.23333333333333	2984810\\
4.25	2986560\\
4.26666666666667	2988170\\
4.28333333333333	2989760\\
4.3	2991200\\
4.31666666666667	2992820\\
4.33333333333333	2994310\\
4.35	2995960\\
4.36666666666667	2997500\\
4.38333333333333	2999250\\
4.4	3000730\\
4.41666666666667	3002170\\
4.43333333333333	3003680\\
4.45	3005430\\
4.46666666666667	3007280\\
4.48333333333333	3008430\\
4.5	3008890\\
4.51666666666667	3012610\\
4.53333333333333	3015480\\
4.55	3018280\\
4.56666666666667	3020570\\
4.58333333333333	3022430\\
4.6	3024140\\
4.61666666666667	3025730\\
4.63333333333333	3027520\\
4.65	3028940\\
4.66666666666667	3030930\\
4.68333333333333	3032300\\
4.7	3033880\\
4.71666666666667	3035240\\
4.73333333333333	3036590\\
4.75	3037770\\
4.76666666666667	3039190\\
4.78333333333333	3040500\\
4.8	3042650\\
4.81666666666667	3044540\\
4.83333333333333	3045820\\
4.85	3047250\\
4.86666666666667	3048650\\
4.88333333333333	3049910\\
4.9	3050910\\
4.91666666666667	3051940\\
4.93333333333333	3053480\\
4.95	3055170\\
4.96666666666667	3056550\\
4.98333333333333	3057840\\
5	3058830\\
5.01666666666667	3059890\\
5.03333333333333	3061350\\
5.05	3063200\\
5.06666666666667	3064310\\
5.08333333333333	3065480\\
5.1	3066390\\
5.11666666666667	3067360\\
5.13333333333333	3068270\\
5.15	3069120\\
5.16666666666667	3069970\\
5.18333333333333	3070990\\
5.2	3071940\\
5.21666666666667	3073140\\
5.23333333333333	3074110\\
5.25	3075300\\
5.26666666666667	3076510\\
5.28333333333333	3078210\\
5.3	3079700\\
5.31666666666667	3081310\\
5.33333333333333	3082810\\
5.35	3084360\\
5.36666666666667	3086320\\
5.38333333333333	3088530\\
5.4	3090860\\
5.41666666666667	3092990\\
5.43333333333333	3094980\\
5.45	3096700\\
5.46666666666667	3098240\\
5.48333333333333	3099850\\
5.5	3101490\\
5.51666666666667	3102900\\
5.53333333333333	3104160\\
5.55	3105330\\
5.56666666666667	3106560\\
5.58333333333333	3107860\\
5.6	3108820\\
5.61666666666667	3109930\\
5.63333333333333	3111060\\
5.65	3112230\\
5.66666666666667	3113540\\
5.68333333333333	3114690\\
5.7	3115770\\
5.71666666666667	3117080\\
5.73333333333333	3118430\\
5.75	3119870\\
5.76666666666667	3121380\\
5.78333333333333	3123130\\
5.8	3125000\\
5.81666666666667	3126560\\
5.83333333333333	3128050\\
5.85	3129490\\
5.86666666666667	3131020\\
5.88333333333333	3132550\\
5.9	3134030\\
5.91666666666667	3135360\\
5.93333333333333	3136730\\
5.95	3137900\\
5.96666666666667	3138840\\
5.98333333333333	3140000\\
6	3141130\\
6.01666666666667	3142480\\
6.03333333333333	3143780\\
6.05	3145180\\
6.06666666666667	3146690\\
6.08333333333333	3148320\\
6.1	3149780\\
6.11666666666667	3151300\\
6.13333333333333	3152620\\
6.15	3154030\\
6.16666666666667	3155370\\
6.18333333333333	3157360\\
6.2	3158820\\
6.21666666666667	3160080\\
6.23333333333333	3161500\\
6.25	3162830\\
6.26666666666667	3164260\\
6.28333333333333	3165680\\
6.3	3167140\\
6.31666666666667	3168550\\
6.33333333333333	3170020\\
6.35	3171400\\
6.36666666666667	3172750\\
6.38333333333333	3174050\\
6.4	3175170\\
6.41666666666667	3176280\\
6.43333333333333	3177350\\
6.45	3178750\\
6.46666666666667	3179900\\
6.48333333333333	3181120\\
6.5	3182120\\
6.51666666666667	3183350\\
6.53333333333333	3184700\\
6.55	3186010\\
6.56666666666667	3187190\\
6.58333333333333	3188470\\
6.6	3189780\\
6.61666666666667	3191070\\
6.63333333333333	3192140\\
6.65	3193250\\
6.66666666666667	3194400\\
6.68333333333333	3195570\\
6.7	3196840\\
6.71666666666667	3198070\\
6.73333333333333	3199390\\
6.75	3200610\\
6.76666666666667	3201890\\
6.78333333333333	3203000\\
6.8	3204140\\
6.81666666666667	3205260\\
6.83333333333333	3206360\\
6.85	3207580\\
6.86666666666667	3209010\\
6.88333333333333	3210400\\
6.9	3211660\\
6.91666666666667	3212840\\
6.93333333333333	3213940\\
6.95	3214930\\
6.96666666666667	3215960\\
6.98333333333333	3216900\\
7	3218060\\
7.01666666666667	3219120\\
7.03333333333333	3220180\\
7.05	3221270\\
7.06666666666667	3222600\\
7.08333333333333	3224070\\
7.1	3226150\\
7.11666666666667	3227530\\
7.13333333333333	3228840\\
7.15	3230190\\
7.16666666666667	3231610\\
7.18333333333333	3232930\\
7.2	3234350\\
7.21666666666667	3235710\\
7.23333333333333	3237010\\
7.25	3238020\\
7.26666666666667	3239120\\
7.28333333333333	3240470\\
7.3	3241590\\
7.31666666666667	3242660\\
7.33333333333333	3243810\\
7.35	3245050\\
7.36666666666667	3246280\\
7.38333333333333	3247290\\
7.4	3248510\\
7.41666666666667	3249720\\
7.43333333333333	3251020\\
7.45	3252380\\
7.46666666666667	3253690\\
7.48333333333333	3254800\\
7.5	3255790\\
7.51666666666667	3256730\\
7.53333333333333	3251260\\
7.55	3256000\\
7.56666666666667	3258430\\
7.58333333333333	3259070\\
7.6	3259250\\
7.61666666666667	3261370\\
7.63333333333333	3262330\\
7.65	3263060\\
7.66666666666667	3264390\\
7.68333333333333	3266120\\
7.7	3267180\\
7.71666666666667	3268610\\
7.73333333333333	3270380\\
7.75	3271850\\
7.76666666666667	3273010\\
7.78333333333333	3274020\\
7.8	3274820\\
7.81666666666667	3275350\\
7.83333333333333	3276400\\
7.85	3276770\\
7.86666666666667	3279630\\
7.88333333333333	3281010\\
7.9	3283770\\
7.91666666666667	3284730\\
7.93333333333333	3285430\\
7.95	3286530\\
7.96666666666667	3287610\\
7.98333333333333	3288790\\
8	3289950\\
8.01666666666667	3290930\\
8.03333333333333	3291980\\
8.05	3293310\\
8.06666666666667	3294320\\
8.08333333333333	3295520\\
8.1	3296720\\
8.11666666666667	3297780\\
8.13333333333333	3299080\\
8.15	3300360\\
8.16666666666667	3301910\\
8.18333333333333	3303140\\
8.2	3304500\\
8.21666666666667	3305800\\
8.23333333333333	3306930\\
8.25	3308100\\
8.26666666666667	3309340\\
8.28333333333333	3310590\\
8.3	3311890\\
8.31666666666667	3312780\\
8.33333333333333	3313970\\
8.35	3314970\\
8.36666666666667	3316050\\
8.38333333333333	3316940\\
8.4	3317970\\
8.41666666666667	3318690\\
8.43333333333333	3319330\\
8.45	3319930\\
8.46666666666667	3320700\\
8.48333333333333	3321450\\
8.5	3322120\\
8.51666666666667	3323020\\
8.53333333333333	3323860\\
8.55	3324580\\
8.56666666666667	3325400\\
8.58333333333333	3326340\\
8.6	3327100\\
8.61666666666667	3327900\\
8.63333333333333	3328550\\
8.65	3329330\\
8.66666666666667	3330160\\
8.68333333333333	3331160\\
8.7	3332070\\
8.71666666666667	3333160\\
8.73333333333333	3334420\\
8.75	3335800\\
8.76666666666667	3336950\\
8.78333333333333	3338110\\
8.8	3339580\\
8.81666666666667	3340680\\
8.83333333333333	3341850\\
8.85	3342900\\
8.86666666666667	3343800\\
8.88333333333333	3344600\\
8.9	3344730\\
8.91666666666667	3345550\\
8.93333333333333	3346580\\
8.95	3347470\\
8.96666666666667	3348370\\
8.98333333333333	3349280\\
9	3350270\\
9.01666666666667	3351120\\
9.03333333333333	3351980\\
9.05	3352840\\
9.06666666666667	3353850\\
9.08333333333333	3354670\\
9.1	3355490\\
9.11666666666667	3356270\\
9.13333333333333	3356760\\
9.15	3357750\\
9.16666666666667	3358550\\
9.18333333333333	3359320\\
9.2	3360340\\
9.21666666666667	3361160\\
9.23333333333333	3362040\\
9.25	3362960\\
9.26666666666667	3363830\\
9.28333333333333	3364700\\
9.3	3365630\\
9.31666666666667	3366570\\
9.33333333333333	3367490\\
9.35	3368460\\
9.36666666666667	3369420\\
9.38333333333333	3370230\\
9.4	3371240\\
9.41666666666667	3372110\\
9.43333333333333	3373060\\
9.45	3373920\\
9.46666666666667	3374810\\
9.48333333333333	3375750\\
9.5	3376730\\
9.51666666666667	3377690\\
9.53333333333333	3378620\\
9.55	3379580\\
9.56666666666667	3380670\\
9.58333333333333	3381610\\
9.6	3382530\\
9.61666666666667	3383400\\
9.63333333333333	3384150\\
9.65	3385000\\
9.66666666666667	3385830\\
9.68333333333333	3386610\\
9.7	3387380\\
9.71666666666667	3388030\\
9.73333333333333	3388720\\
9.75	3389510\\
9.76666666666667	3390400\\
9.78333333333333	3391470\\
9.8	3392290\\
9.81666666666667	3393100\\
9.83333333333333	3393980\\
9.85	3394830\\
9.86666666666667	3395820\\
9.88333333333333	3396800\\
9.9	3397830\\
9.91666666666667	3398670\\
9.93333333333333	3399860\\
9.95	3400700\\
9.96666666666667	3401700\\
9.98333333333333	3402640\\
10	3403690\\
10.0166666666667	3404810\\
10.0333333333333	3405760\\
10.05	3406690\\
10.0666666666667	3407720\\
10.0833333333333	3408610\\
10.1	3409690\\
10.1166666666667	3410900\\
10.1333333333333	3412140\\
10.15	3413360\\
10.1666666666667	3414620\\
10.1833333333333	3415750\\
10.2	3416950\\
10.2166666666667	3418110\\
10.2333333333333	3419400\\
10.25	3420300\\
10.2666666666667	3421280\\
10.2833333333333	3422230\\
10.3	3423290\\
10.3166666666667	3424510\\
10.3333333333333	3425850\\
10.35	3426810\\
10.3666666666667	3427890\\
10.3833333333333	3428890\\
10.4	3429920\\
10.4166666666667	3431030\\
10.4333333333333	3432220\\
10.45	3433370\\
10.4666666666667	3434510\\
10.4833333333333	3435520\\
10.5	3436490\\
10.5166666666667	3437580\\
10.5333333333333	3438810\\
10.55	3439560\\
10.5666666666667	3440640\\
10.5833333333333	3441510\\
10.6	3442340\\
10.6166666666667	3443240\\
10.6333333333333	3444010\\
10.65	3445000\\
10.6666666666667	3445690\\
10.6833333333333	3446650\\
10.7	3447530\\
10.7166666666667	3448680\\
10.7333333333333	3449770\\
10.75	3450780\\
10.7666666666667	3451920\\
10.7833333333333	3453220\\
10.8	3454330\\
10.8166666666667	3455380\\
10.8333333333333	3456430\\
10.85	3457510\\
10.8666666666667	3458660\\
10.8833333333333	3459400\\
10.9	3460280\\
10.9166666666667	3461080\\
10.9333333333333	3461830\\
10.95	3462480\\
10.9666666666667	3463320\\
10.9833333333333	3464150\\
11	3464970\\
11.0166666666667	3465790\\
11.0333333333333	3466470\\
11.05	3467290\\
11.0666666666667	3468060\\
11.0833333333333	3468850\\
11.1	3469560\\
11.1166666666667	3470330\\
11.1333333333333	3471250\\
11.15	3472050\\
11.1666666666667	3472930\\
11.1833333333333	3473960\\
11.2	3474920\\
11.2166666666667	3475940\\
11.2333333333333	3476820\\
11.25	3477860\\
11.2666666666667	3478760\\
11.2833333333333	3479690\\
11.3	3480730\\
11.3166666666667	3481680\\
11.3333333333333	3482480\\
11.35	3483400\\
11.3666666666667	3484200\\
11.3833333333333	3484990\\
11.4	3485720\\
11.4166666666667	3486400\\
11.4333333333333	3487560\\
11.45	3488410\\
11.4666666666667	3489350\\
11.4833333333333	3490240\\
11.5	3491180\\
11.5166666666667	3492110\\
11.5333333333333	3492990\\
11.55	3493970\\
11.5666666666667	3494860\\
11.5833333333333	3495900\\
11.6	3496830\\
11.6166666666667	3497520\\
11.6333333333333	3498300\\
11.65	3498900\\
11.6666666666667	3499150\\
11.6833333333333	3499640\\
11.7	3500260\\
11.7166666666667	3500810\\
11.7333333333333	3501490\\
11.75	3502210\\
11.7666666666667	3503060\\
11.7833333333333	3503680\\
11.8	3504350\\
11.8166666666667	3505100\\
11.8333333333333	3505920\\
11.85	3506800\\
11.8666666666667	3507640\\
11.8833333333333	3508380\\
11.9	3509250\\
11.9166666666667	3510180\\
11.9333333333333	3511050\\
11.95	3512030\\
11.9666666666667	3512960\\
11.9833333333333	3513930\\
12	3514760\\
12.0166666666667	3515810\\
12.0333333333333	3516780\\
12.05	3517700\\
12.0666666666667	3518550\\
12.0833333333333	3519470\\
12.1	3520460\\
12.1166666666667	3521280\\
12.1333333333333	3522020\\
12.15	3522790\\
12.1666666666667	3523750\\
12.1833333333333	3524570\\
12.2	3525420\\
12.2166666666667	3526190\\
12.2333333333333	3526980\\
12.25	3527730\\
12.2666666666667	3528620\\
12.2833333333333	3529360\\
12.3	3530040\\
12.3166666666667	3531020\\
12.3333333333333	3531860\\
12.35	3532710\\
12.3666666666667	3533530\\
12.3833333333333	3534510\\
12.4	3535350\\
12.4166666666667	3536180\\
12.4333333333333	3536970\\
12.45	3537760\\
12.4666666666667	3538560\\
12.4833333333333	3539370\\
12.5	3540280\\
12.5166666666667	3541220\\
12.5333333333333	3542160\\
12.55	3542920\\
12.5666666666667	3543600\\
12.5833333333333	3544040\\
12.6	3545070\\
12.6166666666667	3546190\\
12.6333333333333	3546690\\
12.65	3547360\\
12.6666666666667	3548110\\
12.6833333333333	3548850\\
12.7	3549620\\
12.7166666666667	3550510\\
12.7333333333333	3551210\\
12.75	3551920\\
12.7666666666667	3552760\\
12.7833333333333	3553520\\
12.8	3554230\\
12.8166666666667	3555070\\
12.8333333333333	3555740\\
12.85	3556710\\
12.8666666666667	3557540\\
12.8833333333333	3558300\\
12.9	3559210\\
12.9166666666667	3559890\\
12.9333333333333	3560750\\
12.95	3561380\\
12.9666666666667	3562190\\
12.9833333333333	3562900\\
13	3563650\\
13.0166666666667	3564390\\
13.0333333333333	3565080\\
13.05	3565730\\
13.0666666666667	3566300\\
13.0833333333333	3566860\\
13.1	3567500\\
13.1166666666667	3568320\\
13.1333333333333	3569010\\
13.15	3569750\\
13.1666666666667	3570530\\
13.1833333333333	3571360\\
13.2	3572040\\
13.2166666666667	3572600\\
13.2333333333333	3573540\\
13.25	3574460\\
13.2666666666667	3575010\\
13.2833333333333	3575780\\
13.3	3576690\\
13.3166666666667	3577910\\
13.3333333333333	3578720\\
13.35	3579660\\
13.3666666666667	3580560\\
13.3833333333333	3581360\\
13.4	3582220\\
13.4166666666667	3583270\\
13.4333333333333	3584150\\
13.45	3585000\\
13.4666666666667	3586040\\
13.4833333333333	3587000\\
13.5	3587900\\
13.5166666666667	3588760\\
13.5333333333333	3589540\\
13.55	3590390\\
13.5666666666667	3591200\\
13.5833333333333	3592090\\
13.6	3592920\\
13.6166666666667	3593930\\
13.6333333333333	3594690\\
13.65	3595510\\
13.6666666666667	3596470\\
13.6833333333333	3596920\\
13.7	3597630\\
13.7166666666667	3598530\\
13.7333333333333	3599460\\
13.75	3600170\\
13.7666666666667	3600980\\
13.7833333333333	3601850\\
13.8	3602170\\
13.8166666666667	3598980\\
13.8333333333333	3595170\\
13.85	3595790\\
13.8666666666667	3596830\\
13.8833333333333	3597880\\
13.9	3599030\\
13.9166666666667	3599820\\
13.9333333333333	3600480\\
13.95	3601580\\
13.9666666666667	3602220\\
13.9833333333333	3603080\\
14	3603940\\
14.0166666666667	3604590\\
14.0333333333333	3605400\\
14.05	3606280\\
14.0666666666667	3607100\\
14.0833333333333	3607870\\
14.1	3608450\\
14.1166666666667	3609280\\
14.1333333333333	3609960\\
14.15	3610680\\
14.1666666666667	3611480\\
14.1833333333333	3612290\\
14.2	3612900\\
14.2166666666667	3613880\\
14.2333333333333	3614430\\
14.25	3615250\\
14.2666666666667	3615950\\
14.2833333333333	3616680\\
14.3	3617270\\
14.3166666666667	3618080\\
14.3333333333333	3618880\\
14.35	3619750\\
14.3666666666667	3620590\\
14.3833333333333	3621460\\
14.4	3622280\\
14.4166666666667	3623090\\
14.4333333333333	3623790\\
14.45	3624510\\
14.4666666666667	3625320\\
14.4833333333333	3625980\\
14.5	3626920\\
14.5166666666667	3627820\\
14.5333333333333	3628690\\
14.55	3629610\\
14.5666666666667	3630410\\
14.5833333333333	3631310\\
14.6	3632050\\
14.6166666666667	3632970\\
14.6333333333333	3633820\\
14.65	3634650\\
14.6666666666667	3635570\\
14.6833333333333	3636480\\
14.7	3635810\\
14.7166666666667	3638030\\
14.7333333333333	3638830\\
14.75	3639520\\
14.7666666666667	3640220\\
14.7833333333333	3641120\\
14.8	3641930\\
14.8166666666667	3642670\\
14.8333333333333	3643480\\
14.85	3644200\\
14.8666666666667	3644980\\
14.8833333333333	3645800\\
14.9	3646740\\
14.9166666666667	3647440\\
14.9333333333333	3648340\\
14.95	3649280\\
14.9666666666667	3650220\\
14.9833333333333	3650970\\
15	3651970\\
15.0166666666667	3652950\\
15.0333333333333	3653850\\
15.05	3654820\\
15.0666666666667	3655740\\
15.0833333333333	3656880\\
15.1	3657950\\
15.1166666666667	3658930\\
15.1333333333333	3659920\\
15.15	3660980\\
15.1666666666667	3661940\\
15.1833333333333	3662810\\
15.2	3663960\\
15.2166666666667	3664830\\
15.2333333333333	3665890\\
15.25	3667040\\
15.2666666666667	3668020\\
15.2833333333333	3668960\\
15.3	3670070\\
15.3166666666667	3671220\\
15.3333333333333	3672250\\
15.35	3673250\\
15.3666666666667	3674070\\
15.3833333333333	3675250\\
15.4	3676340\\
15.4166666666667	3677430\\
15.4333333333333	3678300\\
15.45	3679420\\
15.4666666666667	3680520\\
15.4833333333333	3681290\\
15.5	3682310\\
15.5166666666667	3683320\\
15.5333333333333	3684200\\
15.55	3685310\\
15.5666666666667	3686310\\
15.5833333333333	3687310\\
15.6	3688290\\
15.6166666666667	3689330\\
15.6333333333333	3690310\\
15.65	3691160\\
15.6666666666667	3692110\\
15.6833333333333	3693020\\
15.7	3693950\\
15.7166666666667	3694780\\
15.7333333333333	3695590\\
15.75	3696470\\
15.7666666666667	3697370\\
15.7833333333333	3698240\\
15.8	3698980\\
15.8166666666667	3699790\\
15.8333333333333	3700540\\
15.85	3701390\\
15.8666666666667	3702310\\
15.8833333333333	3703370\\
15.9	3704330\\
15.9166666666667	3705170\\
15.9333333333333	3706100\\
15.95	3706940\\
15.9666666666667	3707790\\
15.9833333333333	3708660\\
16	3709370\\
16.0166666666667	3710280\\
16.0333333333333	3711040\\
16.05	3711760\\
16.0666666666667	3712820\\
16.0833333333333	3713810\\
16.1	3714590\\
16.1166666666667	3715470\\
16.1333333333333	3716390\\
16.15	3717440\\
16.1666666666667	3718020\\
16.1833333333333	3718790\\
16.2	3719680\\
16.2166666666667	3720650\\
16.2333333333333	3721540\\
16.25	3722280\\
16.2666666666667	3723210\\
16.2833333333333	3724130\\
16.3	3725030\\
16.3166666666667	3725790\\
16.3333333333333	3726650\\
16.35	3727590\\
16.3666666666667	3728420\\
16.3833333333333	3729210\\
16.4	3730040\\
16.4166666666667	3730840\\
16.4333333333333	3731640\\
16.45	3732430\\
16.4666666666667	3733210\\
16.4833333333333	3734050\\
16.5	3734870\\
16.5166666666667	3735580\\
16.5333333333333	3736430\\
16.55	3737270\\
16.5666666666667	3738090\\
16.5833333333333	3738830\\
16.6	3739660\\
16.6166666666667	3740300\\
16.6333333333333	3740850\\
16.65	3741750\\
16.6666666666667	3742680\\
16.6833333333333	3743600\\
16.7	3744280\\
16.7166666666667	3745110\\
16.7333333333333	3745980\\
16.75	3746830\\
16.7666666666667	3747630\\
16.7833333333333	3748640\\
16.8	3749660\\
16.8166666666667	3750590\\
16.8333333333333	3751590\\
16.85	3752560\\
16.8666666666667	3753520\\
16.8833333333333	3754400\\
16.9	3755220\\
16.9166666666667	3756180\\
16.9333333333333	3757040\\
16.95	3757910\\
16.9666666666667	3758830\\
16.9833333333333	3759750\\
17	3760700\\
17.0166666666667	3761650\\
17.0333333333333	3762400\\
17.05	3763150\\
17.0666666666667	3764080\\
17.0833333333333	3764790\\
17.1	3765450\\
17.1166666666667	3766230\\
17.1333333333333	3766880\\
17.15	3767550\\
17.1666666666667	3768090\\
17.1833333333333	3768550\\
17.2	3768200\\
17.2166666666667	3767350\\
17.2333333333333	3768550\\
17.25	3769320\\
17.2666666666667	3770110\\
17.2833333333333	3770970\\
17.3	3771590\\
17.3166666666667	3772830\\
17.3333333333333	3773940\\
17.35	3774690\\
17.3666666666667	3775470\\
17.3833333333333	3776150\\
17.4	3776610\\
17.4166666666667	3777360\\
17.4333333333333	3777640\\
17.45	3778140\\
17.4666666666667	3778770\\
17.4833333333333	3779140\\
17.5	3779810\\
17.5166666666667	3780420\\
17.5333333333333	3781030\\
17.55	3781710\\
17.5666666666667	3782400\\
17.5833333333333	3783330\\
17.6	3784180\\
17.6166666666667	3785270\\
17.6333333333333	3786360\\
17.65	3787520\\
17.6666666666667	3788550\\
17.6833333333333	3789480\\
17.7	3790270\\
17.7166666666667	3791030\\
17.7333333333333	3791780\\
17.75	3792420\\
17.7666666666667	3793230\\
17.7833333333333	3793820\\
17.8	3794420\\
17.8166666666667	3794970\\
17.8333333333333	3795740\\
17.85	3796190\\
17.8666666666667	3796800\\
17.8833333333333	3797390\\
17.9	3797970\\
17.9166666666667	3798280\\
17.9333333333333	3798350\\
17.95	3798900\\
17.9666666666667	3799590\\
17.9833333333333	3800380\\
18	3801010\\
18.0166666666667	3801620\\
18.0333333333333	3802350\\
18.05	3803220\\
18.0666666666667	3803960\\
18.0833333333333	3804760\\
18.1	3805830\\
18.1166666666667	3806640\\
18.1333333333333	3807440\\
18.15	3808140\\
18.1666666666667	3808850\\
18.1833333333333	3809640\\
18.2	3810410\\
18.2166666666667	3810930\\
18.2333333333333	3811430\\
18.25	3811990\\
18.2666666666667	3812470\\
18.2833333333333	3812960\\
18.3	3813520\\
18.3166666666667	3814170\\
18.3333333333333	3814570\\
18.35	3815110\\
18.3666666666667	3815730\\
18.3833333333333	3816260\\
18.4	3816980\\
18.4166666666667	3817430\\
18.4333333333333	3818110\\
18.45	3818610\\
18.4666666666667	3819240\\
18.4833333333333	3819890\\
18.5	3820720\\
18.5166666666667	3821480\\
18.5333333333333	3822300\\
18.55	3822740\\
18.5666666666667	3823390\\
18.5833333333333	3824080\\
18.6	3824740\\
18.6166666666667	3825370\\
18.6333333333333	3826030\\
18.65	3826780\\
18.6666666666667	3827440\\
18.6833333333333	3828130\\
18.7	3828840\\
18.7166666666667	3829460\\
18.7333333333333	3830200\\
18.75	3831020\\
18.7666666666667	3831860\\
18.7833333333333	3832560\\
18.8	3833510\\
18.8166666666667	3834280\\
18.8333333333333	3834900\\
18.85	3835650\\
18.8666666666667	3836360\\
18.8833333333333	3837160\\
18.9	3837930\\
18.9166666666667	3838470\\
18.9333333333333	3839410\\
18.95	3839940\\
18.9666666666667	3840690\\
18.9833333333333	3841220\\
19	3841870\\
19.0166666666667	3842470\\
19.0333333333333	3843060\\
19.05	3843640\\
19.0666666666667	3844080\\
19.0833333333333	3844780\\
19.1	3845140\\
19.1166666666667	3845760\\
19.1333333333333	3846320\\
19.15	3846870\\
19.1666666666667	3847450\\
19.1833333333333	3848110\\
19.2	3848830\\
19.2166666666667	3849620\\
19.2333333333333	3850310\\
19.25	3851230\\
19.2666666666667	3852330\\
19.2833333333333	3853180\\
19.3	3853990\\
19.3166666666667	3854810\\
19.3333333333333	3855630\\
19.35	3856360\\
19.3666666666667	3857170\\
19.3833333333333	3857980\\
19.4	3858560\\
19.4166666666667	3859430\\
19.4333333333333	3859990\\
19.45	3860670\\
19.4666666666667	3861370\\
19.4833333333333	3861950\\
19.5	3862690\\
19.5166666666667	3863650\\
19.5333333333333	3864110\\
19.55	3864970\\
19.5666666666667	3865500\\
19.5833333333333	3866260\\
19.6	3866770\\
19.6166666666667	3867510\\
19.6333333333333	3868120\\
19.65	3868840\\
19.6666666666667	3869560\\
19.6833333333333	3870350\\
19.7	3870860\\
19.7166666666667	3871650\\
19.7333333333333	3872260\\
19.75	3873190\\
19.7666666666667	3873730\\
19.7833333333333	3874650\\
19.8	3875230\\
19.8166666666667	3875850\\
19.8333333333333	3876780\\
19.85	3877850\\
19.8666666666667	3878840\\
19.8833333333333	3880150\\
19.9	3881160\\
19.9166666666667	3881820\\
19.9333333333333	3882360\\
19.95	3883100\\
19.9666666666667	3883810\\
19.9833333333333	3884510\\
20	3885430\\
20.0166666666667	3886220\\
20.0333333333333	3886940\\
20.05	3887870\\
20.0666666666667	3888230\\
20.0833333333333	3889080\\
20.1	3890000\\
20.1166666666667	3890770\\
20.1333333333333	3891280\\
20.15	3892070\\
20.1666666666667	3892690\\
20.1833333333333	3893180\\
20.2	3893540\\
20.2166666666667	3893980\\
20.2333333333333	3894440\\
20.25	3894800\\
20.2666666666667	3895300\\
20.2833333333333	3895560\\
20.3	3895990\\
20.3166666666667	3896520\\
20.3333333333333	3897070\\
20.35	3897590\\
20.3666666666667	3898160\\
20.3833333333333	3898950\\
20.4	3899890\\
20.4166666666667	3900950\\
20.4333333333333	3902260\\
20.45	3903280\\
20.4666666666667	3904030\\
20.4833333333333	3904980\\
20.5	3905810\\
20.5166666666667	3906630\\
20.5333333333333	3907250\\
20.55	3907850\\
20.5666666666667	3908720\\
20.5833333333333	3909440\\
20.6	3909990\\
20.6166666666667	3910770\\
20.6333333333333	3911400\\
20.65	3912050\\
20.6666666666667	3912740\\
20.6833333333333	3913210\\
20.7	3913960\\
20.7166666666667	3914380\\
20.7333333333333	3915010\\
20.75	3915450\\
20.7666666666667	3916090\\
20.7833333333333	3916570\\
20.8	3916980\\
20.8166666666667	3917580\\
20.8333333333333	3918160\\
20.85	3918710\\
20.8666666666667	3919240\\
20.8833333333333	3919800\\
20.9	3920310\\
20.9166666666667	3921020\\
20.9333333333333	3921510\\
20.95	3922210\\
20.9666666666667	3922970\\
20.9833333333333	3923620\\
21	3924220\\
21.0166666666667	3924650\\
21.0333333333333	3925370\\
21.05	3926050\\
21.0666666666667	3926340\\
21.0833333333333	3926960\\
21.1	3927150\\
21.1166666666667	3927250\\
21.1333333333333	3927920\\
21.15	3928590\\
21.1666666666667	3929190\\
21.1833333333333	3929900\\
21.2	3930480\\
21.2166666666667	3931160\\
21.2333333333333	3931840\\
21.25	3932430\\
21.2666666666667	3933030\\
21.2833333333333	3933700\\
21.3	3934300\\
21.3166666666667	3934870\\
21.3333333333333	3935650\\
21.35	3936340\\
21.3666666666667	3937140\\
21.3833333333333	3937640\\
21.4	3938550\\
21.4166666666667	3939200\\
21.4333333333333	3939960\\
21.45	3940680\\
21.4666666666667	3941580\\
21.4833333333333	3942190\\
21.5	3942830\\
21.5166666666667	3943540\\
21.5333333333333	3944080\\
21.55	3944820\\
21.5666666666667	3945500\\
21.5833333333333	3946030\\
21.6	3946760\\
21.6166666666667	3947400\\
21.6333333333333	3948070\\
21.65	3948700\\
21.6666666666667	3949210\\
21.6833333333333	3949780\\
21.7	3949960\\
21.7166666666667	3950890\\
21.7333333333333	3951600\\
21.75	3952610\\
21.7666666666667	3953240\\
21.7833333333333	3953880\\
21.8	3954240\\
21.8166666666667	3954690\\
21.8333333333333	3955510\\
21.85	3956060\\
21.8666666666667	3956960\\
21.8833333333333	3957470\\
21.9	3958250\\
21.9166666666667	3958880\\
21.9333333333333	3959400\\
21.95	3960170\\
21.9666666666667	3960790\\
21.9833333333333	3961370\\
22	3961800\\
22.0166666666667	3962350\\
22.0333333333333	3962730\\
22.05	3963150\\
22.0666666666667	3963430\\
22.0833333333333	3963790\\
22.1	3933340\\
22.1166666666667	3473830\\
22.1333333333333	3568520\\
22.15	2377730\\
22.1666666666667	3035540\\
22.1833333333333	3039390\\
22.2	3054600\\
22.2166666666667	3072540\\
22.2333333333333	3090140\\
22.25	3106850\\
22.2666666666667	3126480\\
22.2833333333333	3146300\\
22.3	3165900\\
22.3166666666667	3185480\\
22.3333333333333	3204140\\
22.35	3221620\\
22.3666666666667	3237880\\
22.3833333333333	3253460\\
22.4	3267040\\
22.4166666666667	3275450\\
22.4333333333333	3287240\\
22.45	3296990\\
22.4666666666667	3305850\\
22.4833333333333	3314320\\
22.5	3322020\\
22.5166666666667	3329430\\
22.5333333333333	3335960\\
22.55	3342270\\
22.5666666666667	3348260\\
22.5833333333333	3353710\\
22.6	3359150\\
22.6166666666667	3364240\\
22.6333333333333	3369270\\
22.65	3374010\\
22.6666666666667	3378280\\
22.6833333333333	3382440\\
22.7	3386130\\
22.7166666666667	3389550\\
22.7333333333333	3393290\\
22.75	3396600\\
22.7666666666667	3399770\\
22.7833333333333	3403090\\
22.8	3406000\\
22.8166666666667	3408670\\
22.8333333333333	3411060\\
22.85	3413490\\
22.8666666666667	3415790\\
22.8833333333333	3419100\\
22.9	3421070\\
22.9166666666667	3423180\\
22.9333333333333	3425030\\
22.95	3426780\\
22.9666666666667	3428570\\
22.9833333333333	3430550\\
23	3432030\\
23.0166666666667	3433170\\
23.0333333333333	3434390\\
23.05	3434980\\
23.0666666666667	3436130\\
23.0833333333333	3437910\\
23.1	3438960\\
23.1166666666667	3440300\\
23.1333333333333	3441790\\
23.15	3442930\\
23.1666666666667	3444240\\
23.1833333333333	3445810\\
23.2	3447080\\
23.2166666666667	3448550\\
23.2333333333333	3406780\\
23.25	3146580\\
23.2666666666667	1824620\\
23.2833333333333	2894760\\
23.3	3105410\\
23.3166666666667	3200920\\
23.3333333333333	3275860\\
23.35	3478800\\
23.3666666666667	3559880\\
23.3833333333333	3614370\\
23.4	3659220\\
23.4166666666667	3695760\\
23.4333333333333	3730330\\
23.45	3761290\\
23.4666666666667	3787200\\
23.4833333333333	3812170\\
23.5	3834560\\
23.5166666666667	3855240\\
23.5333333333333	3874150\\
23.55	3892500\\
23.5666666666667	3909930\\
23.5833333333333	3925770\\
23.6	3941540\\
23.6166666666667	3956590\\
23.6333333333333	3970870\\
23.65	3984660\\
23.6666666666667	3997070\\
23.6833333333333	4009660\\
23.7	4021970\\
23.7166666666667	4033480\\
23.7333333333333	4045190\\
23.75	4056350\\
23.7666666666667	4067670\\
23.7833333333333	4079780\\
23.8	4092430\\
23.8166666666667	4103210\\
23.8333333333333	4113270\\
23.85	4123190\\
23.8666666666667	4132810\\
23.8833333333333	4143410\\
23.9	4153160\\
23.9166666666667	4162280\\
23.9333333333333	4171020\\
23.95	4179710\\
23.9666666666667	4188260\\
23.9833333333333	4196470\\
24	4204350\\
24.0166666666667	4211850\\
24.0333333333333	4219190\\
24.05	4227170\\
24.0666666666667	4234780\\
24.0833333333333	4241320\\
24.1	4248760\\
24.1166666666667	4256470\\
24.1333333333333	4262930\\
24.15	4269380\\
24.1666666666667	4276100\\
24.1833333333333	4283820\\
24.2	4289900\\
24.2166666666667	4295990\\
24.2333333333333	4302010\\
24.25	4308340\\
24.2666666666667	4314300\\
24.2833333333333	4320450\\
24.3	4326210\\
24.3166666666667	4331920\\
24.3333333333333	4337440\\
24.35	4342670\\
24.3666666666667	4347860\\
24.3833333333333	4353130\\
24.4	4358750\\
24.4166666666667	4363880\\
24.4333333333333	4369340\\
24.45	4373810\\
24.4666666666667	4379050\\
24.4833333333333	4382950\\
24.5	4387410\\
24.5166666666667	4392330\\
24.5333333333333	4396940\\
24.55	4401880\\
24.5666666666667	4406410\\
24.5833333333333	4410920\\
24.6	4415340\\
24.6166666666667	4419580\\
24.6333333333333	4423910\\
24.65	4428150\\
24.6666666666667	4432510\\
24.6833333333333	4436560\\
24.7	4440470\\
24.7166666666667	4444610\\
24.7333333333333	4448390\\
24.75	4452500\\
24.7666666666667	4456570\\
24.7833333333333	4460800\\
24.8	4465140\\
24.8166666666667	4469040\\
24.8333333333333	4472730\\
24.85	4476710\\
24.8666666666667	4480350\\
24.8833333333333	4484300\\
24.9	4487910\\
24.9166666666667	4491440\\
24.9333333333333	4495250\\
24.95	4498740\\
24.9666666666667	4501840\\
24.9833333333333	4505190\\
25	4508540\\
25.0166666666667	4512040\\
25.0333333333333	4515350\\
25.05	4518870\\
25.0666666666667	4522070\\
25.0833333333333	4525320\\
25.1	4528670\\
25.1166666666667	4531720\\
25.1333333333333	4534600\\
25.15	4537710\\
25.1666666666667	4540740\\
25.1833333333333	4543850\\
25.2	4546840\\
25.2166666666667	4553580\\
25.2333333333333	4557030\\
25.25	4559730\\
25.2666666666667	4562070\\
25.2833333333333	4564340\\
25.3	4567020\\
25.3166666666667	4570140\\
25.3333333333333	4576110\\
25.35	4578500\\
25.3666666666667	4580840\\
25.3833333333333	4583500\\
25.4	4586210\\
25.4166666666667	4589070\\
25.4333333333333	4591850\\
25.45	4594420\\
25.4666666666667	4597090\\
25.4833333333333	4599600\\
25.5	4602300\\
25.5166666666667	4604920\\
25.5333333333333	4607320\\
25.55	4610050\\
25.5666666666667	4612580\\
25.5833333333333	4614880\\
25.6	4617480\\
25.6166666666667	4620270\\
25.6333333333333	4622730\\
25.65	4625070\\
25.6666666666667	4627390\\
25.6833333333333	4629780\\
25.7	4632340\\
25.7166666666667	4634630\\
25.7333333333333	4636980\\
25.75	4639240\\
25.7666666666667	4641620\\
25.7833333333333	4643310\\
25.8	4645550\\
25.8166666666667	4647720\\
25.8333333333333	4649640\\
25.85	4651850\\
25.8666666666667	4653860\\
25.8833333333333	4656120\\
25.9	4658700\\
25.9166666666667	4660510\\
25.9333333333333	4662820\\
25.95	4664830\\
25.9666666666667	4666930\\
25.9833333333333	4668820\\
26	4670880\\
26.0166666666667	4672880\\
26.0333333333333	4675100\\
26.05	4677230\\
26.0666666666667	4679220\\
26.0833333333333	4681140\\
26.1	4683110\\
26.1166666666667	4685250\\
26.1333333333333	4687650\\
26.15	4689830\\
26.1666666666667	4692000\\
26.1833333333333	4693620\\
26.2	4695150\\
26.2166666666667	4696860\\
26.2333333333333	4698380\\
26.25	4699060\\
26.2666666666667	4700330\\
26.2833333333333	4702260\\
26.3	4703600\\
26.3166666666667	4705860\\
26.3333333333333	4707400\\
26.35	4708570\\
26.3666666666667	4709840\\
26.3833333333333	4710420\\
26.4	4712120\\
26.4166666666667	4713600\\
26.4333333333333	4715250\\
26.45	4717630\\
26.4666666666667	4719320\\
26.4833333333333	4720860\\
26.5	4722720\\
26.5166666666667	4724310\\
26.5333333333333	4725820\\
26.55	4727350\\
26.5666666666667	4729290\\
26.5833333333333	4731180\\
26.6	4732080\\
26.6166666666667	4734000\\
26.6333333333333	4735010\\
26.65	4736610\\
26.6666666666667	4737980\\
26.6833333333333	4739430\\
26.7	4739990\\
26.7166666666667	4741520\\
26.7333333333333	4742740\\
26.75	4744110\\
26.7666666666667	4745470\\
26.7833333333333	4746870\\
26.8	4748260\\
26.8166666666667	4749860\\
26.8333333333333	4751390\\
26.85	4752690\\
26.8666666666667	4754170\\
26.8833333333333	4755550\\
26.9	4756570\\
26.9166666666667	4757850\\
26.9333333333333	4759320\\
26.95	4761160\\
26.9666666666667	4762480\\
26.9833333333333	4764220\\
27	4765290\\
27.0166666666667	4766540\\
27.0333333333333	4767960\\
27.05	4769210\\
27.0666666666667	4770670\\
27.0833333333333	4771590\\
27.1	4772900\\
27.1166666666667	4774070\\
27.1333333333333	4774450\\
27.15	4775680\\
27.1666666666667	4776610\\
27.1833333333333	4777690\\
27.2	4778750\\
27.2166666666667	4779790\\
27.2333333333333	4781190\\
27.25	4782560\\
27.2666666666667	4783720\\
27.2833333333333	4784750\\
27.3	4785880\\
27.3166666666667	4787280\\
27.3333333333333	4788330\\
27.35	4789490\\
27.3666666666667	4790690\\
27.3833333333333	4792000\\
27.4	4792910\\
27.4166666666667	4793820\\
27.4333333333333	4794660\\
27.45	4795970\\
27.4666666666667	4797040\\
27.4833333333333	4798000\\
27.5	4799230\\
27.5166666666667	4800530\\
27.5333333333333	4801720\\
27.55	4803020\\
27.5666666666667	4804090\\
27.5833333333333	4804900\\
27.6	4806240\\
27.6166666666667	4807050\\
27.6333333333333	4808060\\
27.65	4809070\\
27.6666666666667	4809960\\
27.6833333333333	4810850\\
27.7	4811750\\
27.7166666666667	4812660\\
27.7333333333333	4813600\\
27.75	4814270\\
27.7666666666667	4814990\\
27.7833333333333	4815930\\
27.8	4816740\\
27.8166666666667	4817200\\
27.8333333333333	4818300\\
27.85	4819260\\
27.8666666666667	4819770\\
27.8833333333333	4820670\\
27.9	4821770\\
27.9166666666667	4822720\\
27.9333333333333	4823770\\
27.95	4824980\\
27.9666666666667	4825960\\
27.9833333333333	4827010\\
28	4828300\\
28.0166666666667	4829020\\
28.0333333333333	4830010\\
28.05	4830850\\
28.0666666666667	4831550\\
28.0833333333333	4832690\\
28.1	4833450\\
28.1166666666667	4834520\\
28.1333333333333	4835480\\
28.15	4836460\\
28.1666666666667	4837690\\
28.1833333333333	4838220\\
28.2	4839440\\
28.2166666666667	4840230\\
28.2333333333333	4841070\\
28.25	4841980\\
28.2666666666667	4842780\\
28.2833333333333	4843940\\
28.3	4844000\\
28.3166666666667	4844690\\
28.3333333333333	4845660\\
28.35	4846400\\
28.3666666666667	4846980\\
28.3833333333333	4847920\\
28.4	4848680\\
28.4166666666667	4849450\\
28.4333333333333	4850310\\
28.45	4851090\\
28.4666666666667	4851660\\
28.4833333333333	4852620\\
28.5	4853350\\
28.5166666666667	4854100\\
28.5333333333333	4855060\\
28.55	4855790\\
28.5666666666667	4856400\\
28.5833333333333	4857140\\
28.6	4857700\\
28.6166666666667	4858390\\
28.6333333333333	4859150\\
28.65	4860730\\
28.6666666666667	4861690\\
28.6833333333333	4862330\\
28.7	4863070\\
28.7166666666667	4863970\\
28.7333333333333	4864770\\
28.75	4865400\\
28.7666666666667	4866140\\
28.7833333333333	4866590\\
28.8	4866810\\
28.8166666666667	4867250\\
28.8333333333333	4867930\\
28.85	4868520\\
28.8666666666667	4869050\\
28.8833333333333	4869700\\
28.9	4870270\\
28.9166666666667	4870840\\
28.9333333333333	4871590\\
28.95	4872150\\
28.9666666666667	4872520\\
28.9833333333333	4873250\\
29	4873710\\
29.0166666666667	4874350\\
29.0333333333333	4874910\\
29.05	4875830\\
29.0666666666667	4876650\\
29.0833333333333	4877220\\
29.1	4878020\\
29.1166666666667	4878710\\
29.1333333333333	4879230\\
29.15	4879920\\
29.1666666666667	4880660\\
29.1833333333333	4881240\\
29.2	4881970\\
29.2166666666667	4882300\\
29.2333333333333	4883170\\
29.25	4883660\\
29.2666666666667	4884210\\
29.2833333333333	4884900\\
29.3	4885230\\
29.3166666666667	4885850\\
29.3333333333333	4886430\\
29.35	4886840\\
29.3666666666667	4887470\\
29.3833333333333	4888100\\
29.4	4888740\\
29.4166666666667	4889190\\
29.4333333333333	4889500\\
29.45	4889720\\
29.4666666666667	4890400\\
29.4833333333333	4890650\\
29.5	4891190\\
29.5166666666667	4891390\\
29.5333333333333	4892110\\
29.55	4892680\\
29.5666666666667	4892980\\
29.5833333333333	4893910\\
29.6	4894930\\
29.6166666666667	4895610\\
29.6333333333333	4896620\\
29.65	4896330\\
29.6666666666667	4897010\\
29.6833333333333	4897580\\
29.7	4898060\\
29.7166666666667	4898750\\
29.7333333333333	4899390\\
29.75	4899910\\
29.7666666666667	4900290\\
29.7833333333333	4900990\\
29.8	4901490\\
29.8166666666667	4901990\\
29.8333333333333	4902400\\
29.85	4903040\\
29.8666666666667	4903290\\
29.8833333333333	4904000\\
29.9	4904480\\
29.9166666666667	4905060\\
29.9333333333333	4905300\\
29.95	4905930\\
29.9666666666667	4906340\\
29.9833333333333	4906820\\
};

\addplot [color=mycolor5]
  table[row sep=crcr]{%
0	1986200\\
0.0166666666666667	1999790\\
0.0333333333333333	2011500\\
0.05	2021330\\
0.0666666666666667	2031660\\
0.0833333333333333	2043190\\
0.1	2051140\\
0.116666666666667	2059640\\
0.133333333333333	2067070\\
0.15	2075380\\
0.166666666666667	2086440\\
0.183333333333333	2095230\\
0.2	2104940\\
0.216666666666667	2113030\\
0.233333333333333	2121630\\
0.25	2129690\\
0.266666666666667	2137530\\
0.283333333333333	2143360\\
0.3	2150800\\
0.316666666666667	2155220\\
0.333333333333333	2159980\\
0.35	2165680\\
0.366666666666667	2171310\\
0.383333333333333	2178150\\
0.4	2184380\\
0.416666666666667	2190200\\
0.433333333333333	2196560\\
0.45	2203530\\
0.466666666666667	2210870\\
0.483333333333333	2218670\\
0.5	2226280\\
0.516666666666667	2233560\\
0.533333333333333	2240810\\
0.55	2248170\\
0.566666666666667	2255670\\
0.583333333333333	2262910\\
0.6	2269410\\
0.616666666666667	2275530\\
0.633333333333333	2281400\\
0.65	2287380\\
0.666666666666667	2292700\\
0.683333333333333	2298230\\
0.7	2304230\\
0.716666666666667	2309240\\
0.733333333333333	2314170\\
0.75	2318810\\
0.766666666666667	2323370\\
0.783333333333333	2327570\\
0.8	2331600\\
0.816666666666667	2335070\\
0.833333333333333	2338310\\
0.85	2341320\\
0.866666666666667	2344510\\
0.883333333333333	2347620\\
0.9	2350860\\
0.916666666666667	2354910\\
0.933333333333333	2358670\\
0.95	2362380\\
0.966666666666667	2366130\\
0.983333333333333	2369960\\
1	2373900\\
1.01666666666667	2377740\\
1.03333333333333	2381650\\
1.05	2385870\\
1.06666666666667	2390240\\
1.08333333333333	2394470\\
1.1	2398590\\
1.11666666666667	2403010\\
1.13333333333333	2407150\\
1.15	2411150\\
1.16666666666667	2415090\\
1.18333333333333	2419000\\
1.2	2422910\\
1.21666666666667	2426390\\
1.23333333333333	2428900\\
1.25	2428660\\
1.26666666666667	2430900\\
1.28333333333333	2433500\\
1.3	2438280\\
1.31666666666667	2442220\\
1.33333333333333	2445370\\
1.35	2448580\\
1.36666666666667	2452010\\
1.38333333333333	2455500\\
1.4	2458930\\
1.41666666666667	2462300\\
1.43333333333333	2465330\\
1.45	2468080\\
1.46666666666667	2470820\\
1.48333333333333	2473640\\
1.5	2476430\\
1.51666666666667	2479400\\
1.53333333333333	2482430\\
1.55	2485150\\
1.56666666666667	2488080\\
1.58333333333333	2491150\\
1.6	2494060\\
1.61666666666667	2496760\\
1.63333333333333	2499940\\
1.65	2502980\\
1.66666666666667	2505910\\
1.68333333333333	2509010\\
1.7	2512140\\
1.71666666666667	2514530\\
1.73333333333333	2517320\\
1.75	2518150\\
1.76666666666667	2517370\\
1.78333333333333	2520260\\
1.8	2523260\\
1.81666666666667	2526420\\
1.83333333333333	2526390\\
1.85	2525740\\
1.86666666666667	2528170\\
1.88333333333333	2530490\\
1.9	2533230\\
1.91666666666667	2535870\\
1.93333333333333	2537860\\
1.95	2540460\\
1.96666666666667	2542730\\
1.98333333333333	2545020\\
2	2547520\\
2.01666666666667	2550280\\
2.03333333333333	2552960\\
2.05	2555350\\
2.06666666666667	2557860\\
2.08333333333333	2560350\\
2.1	2563100\\
2.11666666666667	2565760\\
2.13333333333333	2568130\\
2.15	2570520\\
2.16666666666667	2435720\\
2.18333333333333	2715660\\
2.2	2729730\\
2.21666666666667	2734250\\
2.23333333333333	2738450\\
2.25	2742050\\
2.26666666666667	2745520\\
2.28333333333333	2748400\\
2.3	2750200\\
2.31666666666667	2751650\\
2.33333333333333	2753840\\
2.35	2755920\\
2.36666666666667	2757830\\
2.38333333333333	2760260\\
2.4	2762270\\
2.41666666666667	2767050\\
2.43333333333333	2773350\\
2.45	2775690\\
2.46666666666667	2777810\\
2.48333333333333	2780140\\
2.5	2782480\\
2.51666666666667	2784670\\
2.53333333333333	2786950\\
2.55	2789340\\
2.56666666666667	2791410\\
2.58333333333333	2794680\\
2.6	2796630\\
2.61666666666667	2798650\\
2.63333333333333	2800260\\
2.65	2801650\\
2.66666666666667	2803280\\
2.68333333333333	2804250\\
2.7	2805550\\
2.71666666666667	2806910\\
2.73333333333333	2807930\\
2.75	2808960\\
2.76666666666667	2810560\\
2.78333333333333	2812420\\
2.8	2814440\\
2.81666666666667	2816480\\
2.83333333333333	2818910\\
2.85	2821380\\
2.86666666666667	2823460\\
2.88333333333333	2825390\\
2.9	2827330\\
2.91666666666667	2829440\\
2.93333333333333	2831590\\
2.95	2834780\\
2.96666666666667	2837340\\
2.98333333333333	2839690\\
3	2841640\\
3.01666666666667	2843950\\
3.03333333333333	2846080\\
3.05	2849490\\
3.06666666666667	2852840\\
3.08333333333333	2854510\\
3.1	2856370\\
3.11666666666667	2858350\\
3.13333333333333	2860160\\
3.15	2862200\\
3.16666666666667	2864190\\
3.18333333333333	2866150\\
3.2	2868070\\
3.21666666666667	2870140\\
3.23333333333333	2872280\\
3.25	2874280\\
3.26666666666667	2876660\\
3.28333333333333	2879240\\
3.3	2882100\\
3.31666666666667	2884570\\
3.33333333333333	2887160\\
3.35	2889720\\
3.36666666666667	2892160\\
3.38333333333333	2894620\\
3.4	2897110\\
3.41666666666667	2899300\\
3.43333333333333	2901260\\
3.45	2903020\\
3.46666666666667	2904870\\
3.48333333333333	2906190\\
3.5	2907610\\
3.51666666666667	2906740\\
3.53333333333333	2908630\\
3.55	2910340\\
3.56666666666667	2912040\\
3.58333333333333	2913740\\
3.6	2915600\\
3.61666666666667	2917340\\
3.63333333333333	2919370\\
3.65	2921380\\
3.66666666666667	2923580\\
3.68333333333333	2925820\\
3.7	2927970\\
3.71666666666667	2929980\\
3.73333333333333	2932400\\
3.75	2934290\\
3.76666666666667	2935960\\
3.78333333333333	2937440\\
3.8	2939090\\
3.81666666666667	2940830\\
3.83333333333333	2942570\\
3.85	2944350\\
3.86666666666667	2946090\\
3.88333333333333	2947560\\
3.9	2949310\\
3.91666666666667	2951040\\
3.93333333333333	2952740\\
3.95	2954200\\
3.96666666666667	2955630\\
3.98333333333333	2956820\\
4	2958160\\
4.01666666666667	2959560\\
4.03333333333333	2961090\\
4.05	2962900\\
4.06666666666667	2964860\\
4.08333333333333	2966950\\
4.1	2969140\\
4.11666666666667	2971360\\
4.13333333333333	2973500\\
4.15	2975680\\
4.16666666666667	2977760\\
4.18333333333333	2979530\\
4.2	2981230\\
4.21666666666667	2983050\\
4.23333333333333	2984810\\
4.25	2986560\\
4.26666666666667	2988170\\
4.28333333333333	2989760\\
4.3	2991200\\
4.31666666666667	2992820\\
4.33333333333333	2994310\\
4.35	2995960\\
4.36666666666667	2997500\\
4.38333333333333	2999250\\
4.4	3000730\\
4.41666666666667	3002170\\
4.43333333333333	3003680\\
4.45	3005430\\
4.46666666666667	3007280\\
4.48333333333333	3008430\\
4.5	3008890\\
4.51666666666667	3012610\\
4.53333333333333	3015480\\
4.55	3018280\\
4.56666666666667	3020570\\
4.58333333333333	3022430\\
4.6	3024140\\
4.61666666666667	3025730\\
4.63333333333333	3027520\\
4.65	3028940\\
4.66666666666667	3030930\\
4.68333333333333	3032300\\
4.7	3033880\\
4.71666666666667	3035240\\
4.73333333333333	3036590\\
4.75	3037770\\
4.76666666666667	3039190\\
4.78333333333333	3040500\\
4.8	3042650\\
4.81666666666667	3044540\\
4.83333333333333	3045820\\
4.85	3047250\\
4.86666666666667	3048650\\
4.88333333333333	3049910\\
4.9	3050910\\
4.91666666666667	3051940\\
4.93333333333333	3053480\\
4.95	3055170\\
4.96666666666667	3056550\\
4.98333333333333	3057840\\
5	3058830\\
5.01666666666667	3059890\\
5.03333333333333	3061350\\
5.05	3063200\\
5.06666666666667	3064310\\
5.08333333333333	3065480\\
5.1	3066390\\
5.11666666666667	3067360\\
5.13333333333333	3068270\\
5.15	3069120\\
5.16666666666667	3069970\\
5.18333333333333	3070990\\
5.2	3071940\\
5.21666666666667	3073140\\
5.23333333333333	3074110\\
5.25	3075300\\
5.26666666666667	3076510\\
5.28333333333333	3078210\\
5.3	3079700\\
5.31666666666667	3081310\\
5.33333333333333	3082810\\
5.35	3084360\\
5.36666666666667	3086320\\
5.38333333333333	3088530\\
5.4	3090860\\
5.41666666666667	3092990\\
5.43333333333333	3094980\\
5.45	3096700\\
5.46666666666667	3098240\\
5.48333333333333	3099850\\
5.5	3101490\\
5.51666666666667	3102900\\
5.53333333333333	3104160\\
5.55	3105330\\
5.56666666666667	3106560\\
5.58333333333333	3107860\\
5.6	3108820\\
5.61666666666667	3109930\\
5.63333333333333	3111060\\
5.65	3112230\\
5.66666666666667	3113540\\
5.68333333333333	3114690\\
5.7	3115770\\
5.71666666666667	3117080\\
5.73333333333333	3118430\\
5.75	3119870\\
5.76666666666667	3121380\\
5.78333333333333	3123130\\
5.8	3125000\\
5.81666666666667	3126560\\
5.83333333333333	3128050\\
5.85	3129490\\
5.86666666666667	3131020\\
5.88333333333333	3132550\\
5.9	3134030\\
5.91666666666667	3135360\\
5.93333333333333	3136730\\
5.95	3137900\\
5.96666666666667	3138840\\
5.98333333333333	3140000\\
6	3141130\\
6.01666666666667	3142480\\
6.03333333333333	3143780\\
6.05	3145180\\
6.06666666666667	3146690\\
6.08333333333333	3148320\\
6.1	3149780\\
6.11666666666667	3151300\\
6.13333333333333	3152620\\
6.15	3154030\\
6.16666666666667	3155370\\
6.18333333333333	3157360\\
6.2	3158820\\
6.21666666666667	3160080\\
6.23333333333333	3161500\\
6.25	3162830\\
6.26666666666667	3164260\\
6.28333333333333	3165680\\
6.3	3167140\\
6.31666666666667	3168550\\
6.33333333333333	3170020\\
6.35	3171400\\
6.36666666666667	3172750\\
6.38333333333333	3174050\\
6.4	3175170\\
6.41666666666667	3176280\\
6.43333333333333	3177350\\
6.45	3178750\\
6.46666666666667	3179900\\
6.48333333333333	3181120\\
6.5	3182120\\
6.51666666666667	3183350\\
6.53333333333333	3184700\\
6.55	3186010\\
6.56666666666667	3187190\\
6.58333333333333	3188470\\
6.6	3189780\\
6.61666666666667	3191070\\
6.63333333333333	3192140\\
6.65	3193250\\
6.66666666666667	3194400\\
6.68333333333333	3195570\\
6.7	3196840\\
6.71666666666667	3198070\\
6.73333333333333	3199390\\
6.75	3200610\\
6.76666666666667	3201890\\
6.78333333333333	3203000\\
6.8	3204140\\
6.81666666666667	3205260\\
6.83333333333333	3206360\\
6.85	3207580\\
6.86666666666667	3209010\\
6.88333333333333	3210400\\
6.9	3211660\\
6.91666666666667	3212840\\
6.93333333333333	3213940\\
6.95	3214930\\
6.96666666666667	3215960\\
6.98333333333333	3216900\\
7	3218060\\
7.01666666666667	3219120\\
7.03333333333333	3220180\\
7.05	3221270\\
7.06666666666667	3222600\\
7.08333333333333	3224070\\
7.1	3226150\\
7.11666666666667	3227530\\
7.13333333333333	3228840\\
7.15	3230190\\
7.16666666666667	3231610\\
7.18333333333333	3232930\\
7.2	3234350\\
7.21666666666667	3235710\\
7.23333333333333	3237010\\
7.25	3238020\\
7.26666666666667	3239120\\
7.28333333333333	3240470\\
7.3	3241590\\
7.31666666666667	3242660\\
7.33333333333333	3243810\\
7.35	3245050\\
7.36666666666667	3246280\\
7.38333333333333	3247290\\
7.4	3248510\\
7.41666666666667	3249720\\
7.43333333333333	3251020\\
7.45	3252380\\
7.46666666666667	3253690\\
7.48333333333333	3254800\\
7.5	3255790\\
7.51666666666667	3256730\\
7.53333333333333	3251260\\
7.55	3256000\\
7.56666666666667	3258430\\
7.58333333333333	3259070\\
7.6	3259250\\
7.61666666666667	3261370\\
7.63333333333333	3262330\\
7.65	3263060\\
7.66666666666667	3264390\\
7.68333333333333	3266120\\
7.7	3267180\\
7.71666666666667	3268610\\
7.73333333333333	3270380\\
7.75	3271850\\
7.76666666666667	3273010\\
7.78333333333333	3274020\\
7.8	3274820\\
7.81666666666667	3275350\\
7.83333333333333	3276400\\
7.85	3276770\\
7.86666666666667	3279630\\
7.88333333333333	3281010\\
7.9	3283770\\
7.91666666666667	3284730\\
7.93333333333333	3285430\\
7.95	3286530\\
7.96666666666667	3287610\\
7.98333333333333	3288790\\
8	3289950\\
8.01666666666667	3290930\\
8.03333333333333	3291980\\
8.05	3293310\\
8.06666666666667	3294320\\
8.08333333333333	3295520\\
8.1	3296720\\
8.11666666666667	3297780\\
8.13333333333333	3299080\\
8.15	3300360\\
8.16666666666667	3301910\\
8.18333333333333	3303140\\
8.2	3304500\\
8.21666666666667	3305800\\
8.23333333333333	3306930\\
8.25	3308100\\
8.26666666666667	3309340\\
8.28333333333333	3310590\\
8.3	3311890\\
8.31666666666667	3312780\\
8.33333333333333	3313970\\
8.35	3314970\\
8.36666666666667	3316050\\
8.38333333333333	3316940\\
8.4	3317970\\
8.41666666666667	3318690\\
8.43333333333333	3319330\\
8.45	3319930\\
8.46666666666667	3320700\\
8.48333333333333	3321450\\
8.5	3322120\\
8.51666666666667	3323020\\
8.53333333333333	3323860\\
8.55	3324580\\
8.56666666666667	3325400\\
8.58333333333333	3326340\\
8.6	3327100\\
8.61666666666667	3327900\\
8.63333333333333	3328550\\
8.65	3329330\\
8.66666666666667	3330160\\
8.68333333333333	3331160\\
8.7	3332070\\
8.71666666666667	3333160\\
8.73333333333333	3334420\\
8.75	3335800\\
8.76666666666667	3336950\\
8.78333333333333	3338110\\
8.8	3339580\\
8.81666666666667	3340680\\
8.83333333333333	3341850\\
8.85	3342900\\
8.86666666666667	3343800\\
8.88333333333333	3344600\\
8.9	3344730\\
8.91666666666667	3345550\\
8.93333333333333	3346580\\
8.95	3347470\\
8.96666666666667	3348370\\
8.98333333333333	3349280\\
9	3350270\\
9.01666666666667	3351120\\
9.03333333333333	3351980\\
9.05	3352840\\
9.06666666666667	3353850\\
9.08333333333333	3354670\\
9.1	3355490\\
9.11666666666667	3356270\\
9.13333333333333	3356760\\
9.15	3357750\\
9.16666666666667	3358550\\
9.18333333333333	3359320\\
9.2	3360340\\
9.21666666666667	3361160\\
9.23333333333333	3362040\\
9.25	3362960\\
9.26666666666667	3363830\\
9.28333333333333	3364700\\
9.3	3365630\\
9.31666666666667	3366570\\
9.33333333333333	3367490\\
9.35	3368460\\
9.36666666666667	3369420\\
9.38333333333333	3370230\\
9.4	3371240\\
9.41666666666667	3372110\\
9.43333333333333	3373060\\
9.45	3373920\\
9.46666666666667	3374810\\
9.48333333333333	3375750\\
9.5	3376730\\
9.51666666666667	3377690\\
9.53333333333333	3378620\\
9.55	3379580\\
9.56666666666667	3380670\\
9.58333333333333	3381610\\
9.6	3382530\\
9.61666666666667	3383400\\
9.63333333333333	3384150\\
9.65	3385000\\
9.66666666666667	3385830\\
9.68333333333333	3386610\\
9.7	3387380\\
9.71666666666667	3388030\\
9.73333333333333	3388720\\
9.75	3389510\\
9.76666666666667	3390400\\
9.78333333333333	3391470\\
9.8	3392290\\
9.81666666666667	3393100\\
9.83333333333333	3393980\\
9.85	3394830\\
9.86666666666667	3395820\\
9.88333333333333	3396800\\
9.9	3397830\\
9.91666666666667	3398670\\
9.93333333333333	3399860\\
9.95	3400700\\
9.96666666666667	3401700\\
9.98333333333333	3402640\\
10	3403690\\
10.0166666666667	3404810\\
10.0333333333333	3405760\\
10.05	3406690\\
10.0666666666667	3407720\\
10.0833333333333	3408610\\
10.1	3409690\\
10.1166666666667	3410900\\
10.1333333333333	3412140\\
10.15	3413360\\
10.1666666666667	3414620\\
10.1833333333333	3415750\\
10.2	3416950\\
10.2166666666667	3418110\\
10.2333333333333	3419400\\
10.25	3420300\\
10.2666666666667	3421280\\
10.2833333333333	3422230\\
10.3	3423290\\
10.3166666666667	3424510\\
10.3333333333333	3425850\\
10.35	3426810\\
10.3666666666667	3427890\\
10.3833333333333	3428890\\
10.4	3429920\\
10.4166666666667	3431030\\
10.4333333333333	3432220\\
10.45	3433370\\
10.4666666666667	3434510\\
10.4833333333333	3435520\\
10.5	3436490\\
10.5166666666667	3437580\\
10.5333333333333	3438810\\
10.55	3439560\\
10.5666666666667	3440640\\
10.5833333333333	3441510\\
10.6	3442340\\
10.6166666666667	3443240\\
10.6333333333333	3444010\\
10.65	3445000\\
10.6666666666667	3445690\\
10.6833333333333	3446650\\
10.7	3447530\\
10.7166666666667	3448680\\
10.7333333333333	3449770\\
10.75	3450780\\
10.7666666666667	3451920\\
10.7833333333333	3453220\\
10.8	3454330\\
10.8166666666667	3455380\\
10.8333333333333	3456430\\
10.85	3457510\\
10.8666666666667	3458660\\
10.8833333333333	3459400\\
10.9	3460280\\
10.9166666666667	3461080\\
10.9333333333333	3461830\\
10.95	3462480\\
10.9666666666667	3463320\\
10.9833333333333	3464150\\
11	3464970\\
11.0166666666667	3465790\\
11.0333333333333	3466470\\
11.05	3467290\\
11.0666666666667	3468060\\
11.0833333333333	3468850\\
11.1	3469560\\
11.1166666666667	3470330\\
11.1333333333333	3471250\\
11.15	3472050\\
11.1666666666667	3472930\\
11.1833333333333	3473960\\
11.2	3474920\\
11.2166666666667	3475940\\
11.2333333333333	3476820\\
11.25	3477860\\
11.2666666666667	3478760\\
11.2833333333333	3479690\\
11.3	3480730\\
11.3166666666667	3481680\\
11.3333333333333	3482480\\
11.35	3483400\\
11.3666666666667	3484200\\
11.3833333333333	3484990\\
11.4	3485720\\
11.4166666666667	3486400\\
11.4333333333333	3487560\\
11.45	3488410\\
11.4666666666667	3489350\\
11.4833333333333	3490240\\
11.5	3491180\\
11.5166666666667	3492110\\
11.5333333333333	3492990\\
11.55	3493970\\
11.5666666666667	3494860\\
11.5833333333333	3495900\\
11.6	3496830\\
11.6166666666667	3497520\\
11.6333333333333	3498300\\
11.65	3498900\\
11.6666666666667	3499150\\
11.6833333333333	3499640\\
11.7	3500260\\
11.7166666666667	3500810\\
11.7333333333333	3501490\\
11.75	3502210\\
11.7666666666667	3503060\\
11.7833333333333	3503680\\
11.8	3504350\\
11.8166666666667	3505100\\
11.8333333333333	3505920\\
11.85	3506800\\
11.8666666666667	3507640\\
11.8833333333333	3508380\\
11.9	3509250\\
11.9166666666667	3510180\\
11.9333333333333	3511050\\
11.95	3512030\\
11.9666666666667	3512960\\
11.9833333333333	3513930\\
12	3514760\\
12.0166666666667	3515810\\
12.0333333333333	3516780\\
12.05	3517700\\
12.0666666666667	3518550\\
12.0833333333333	3519470\\
12.1	3520460\\
12.1166666666667	3521280\\
12.1333333333333	3522020\\
12.15	3522790\\
12.1666666666667	3523750\\
12.1833333333333	3524570\\
12.2	3525420\\
12.2166666666667	3526190\\
12.2333333333333	3526980\\
12.25	3527730\\
12.2666666666667	3528620\\
12.2833333333333	3529360\\
12.3	3530040\\
12.3166666666667	3531020\\
12.3333333333333	3531860\\
12.35	3532710\\
12.3666666666667	3533530\\
12.3833333333333	3534510\\
12.4	3535350\\
12.4166666666667	3536180\\
12.4333333333333	3536970\\
12.45	3537760\\
12.4666666666667	3538560\\
12.4833333333333	3539370\\
12.5	3540280\\
12.5166666666667	3541220\\
12.5333333333333	3542160\\
12.55	3542920\\
12.5666666666667	3543600\\
12.5833333333333	3544040\\
12.6	3545070\\
12.6166666666667	3546190\\
12.6333333333333	3546690\\
12.65	3547360\\
12.6666666666667	3548110\\
12.6833333333333	3548850\\
12.7	3549620\\
12.7166666666667	3550510\\
12.7333333333333	3551210\\
12.75	3551920\\
12.7666666666667	3552760\\
12.7833333333333	3553520\\
12.8	3554230\\
12.8166666666667	3555070\\
12.8333333333333	3555740\\
12.85	3556710\\
12.8666666666667	3557540\\
12.8833333333333	3558300\\
12.9	3559210\\
12.9166666666667	3559890\\
12.9333333333333	3560750\\
12.95	3561380\\
12.9666666666667	3562190\\
12.9833333333333	3562900\\
13	3563650\\
13.0166666666667	3564390\\
13.0333333333333	3565080\\
13.05	3565730\\
13.0666666666667	3566300\\
13.0833333333333	3566860\\
13.1	3567500\\
13.1166666666667	3568320\\
13.1333333333333	3569010\\
13.15	3569750\\
13.1666666666667	3570530\\
13.1833333333333	3571360\\
13.2	3572040\\
13.2166666666667	3572600\\
13.2333333333333	3573540\\
13.25	3574460\\
13.2666666666667	3575010\\
13.2833333333333	3575780\\
13.3	3576690\\
13.3166666666667	3577910\\
13.3333333333333	3578720\\
13.35	3579660\\
13.3666666666667	3580560\\
13.3833333333333	3581360\\
13.4	3582220\\
13.4166666666667	3583270\\
13.4333333333333	3584150\\
13.45	3585000\\
13.4666666666667	3586040\\
13.4833333333333	3587000\\
13.5	3587900\\
13.5166666666667	3588760\\
13.5333333333333	3589540\\
13.55	3590390\\
13.5666666666667	3591200\\
13.5833333333333	3592090\\
13.6	3592920\\
13.6166666666667	3593930\\
13.6333333333333	3594690\\
13.65	3595510\\
13.6666666666667	3596470\\
13.6833333333333	3596920\\
13.7	3597630\\
13.7166666666667	3598530\\
13.7333333333333	3599460\\
13.75	3600170\\
13.7666666666667	3600980\\
13.7833333333333	3601850\\
13.8	3602170\\
13.8166666666667	3598980\\
13.8333333333333	3595170\\
13.85	3595790\\
13.8666666666667	3596830\\
13.8833333333333	3597880\\
13.9	3599030\\
13.9166666666667	3599820\\
13.9333333333333	3600480\\
13.95	3601580\\
13.9666666666667	3602220\\
13.9833333333333	3603080\\
14	3603940\\
14.0166666666667	3604590\\
14.0333333333333	3605400\\
14.05	3606280\\
14.0666666666667	3607100\\
14.0833333333333	3607870\\
14.1	3608450\\
14.1166666666667	3609280\\
14.1333333333333	3609960\\
14.15	3610680\\
14.1666666666667	3611480\\
14.1833333333333	3612290\\
14.2	3612900\\
14.2166666666667	3613880\\
14.2333333333333	3614430\\
14.25	3615250\\
14.2666666666667	3615950\\
14.2833333333333	3616680\\
14.3	3617270\\
14.3166666666667	3618080\\
14.3333333333333	3618880\\
14.35	3619750\\
14.3666666666667	3620590\\
14.3833333333333	3621460\\
14.4	3622280\\
14.4166666666667	3623090\\
14.4333333333333	3623790\\
14.45	3624510\\
14.4666666666667	3625320\\
14.4833333333333	3625980\\
14.5	3626920\\
14.5166666666667	3627820\\
14.5333333333333	3628690\\
14.55	3629610\\
14.5666666666667	3630410\\
14.5833333333333	3631310\\
14.6	3632050\\
14.6166666666667	3632970\\
14.6333333333333	3633820\\
14.65	3634650\\
14.6666666666667	3635570\\
14.6833333333333	3636480\\
14.7	3635810\\
14.7166666666667	3638030\\
14.7333333333333	3638830\\
14.75	3639520\\
14.7666666666667	3640220\\
14.7833333333333	3641120\\
14.8	3641930\\
14.8166666666667	3642670\\
14.8333333333333	3643480\\
14.85	3644200\\
14.8666666666667	3644980\\
14.8833333333333	3645800\\
14.9	3646740\\
14.9166666666667	3647440\\
14.9333333333333	3648340\\
14.95	3649280\\
14.9666666666667	3650220\\
14.9833333333333	3650970\\
15	3651970\\
15.0166666666667	3652950\\
15.0333333333333	3653850\\
15.05	3654820\\
15.0666666666667	3655740\\
15.0833333333333	3656880\\
15.1	3657950\\
15.1166666666667	3658930\\
15.1333333333333	3659920\\
15.15	3660980\\
15.1666666666667	3661940\\
15.1833333333333	3662810\\
15.2	3663960\\
15.2166666666667	3664830\\
15.2333333333333	3665890\\
15.25	3667040\\
15.2666666666667	3668020\\
15.2833333333333	3668960\\
15.3	3670070\\
15.3166666666667	3671220\\
15.3333333333333	3672250\\
15.35	3673250\\
15.3666666666667	3674070\\
15.3833333333333	3675250\\
15.4	3676340\\
15.4166666666667	3677430\\
15.4333333333333	3678300\\
15.45	3679420\\
15.4666666666667	3680520\\
15.4833333333333	3681290\\
15.5	3682310\\
15.5166666666667	3683320\\
15.5333333333333	3684200\\
15.55	3685310\\
15.5666666666667	3686310\\
15.5833333333333	3687310\\
15.6	3688290\\
15.6166666666667	3689330\\
15.6333333333333	3690310\\
15.65	3691160\\
15.6666666666667	3692110\\
15.6833333333333	3693020\\
15.7	3693950\\
15.7166666666667	3694780\\
15.7333333333333	3695590\\
15.75	3696470\\
15.7666666666667	3697370\\
15.7833333333333	3698240\\
15.8	3698980\\
15.8166666666667	3699790\\
15.8333333333333	3700540\\
15.85	3701390\\
15.8666666666667	3702310\\
15.8833333333333	3703370\\
15.9	3704330\\
15.9166666666667	3705170\\
15.9333333333333	3706100\\
15.95	3706940\\
15.9666666666667	3707790\\
15.9833333333333	3708660\\
16	3709370\\
16.0166666666667	3710280\\
16.0333333333333	3711040\\
16.05	3711760\\
16.0666666666667	3712820\\
16.0833333333333	3713810\\
16.1	3714590\\
16.1166666666667	3715470\\
16.1333333333333	3716390\\
16.15	3717440\\
16.1666666666667	3718020\\
16.1833333333333	3718790\\
16.2	3719680\\
16.2166666666667	3720650\\
16.2333333333333	3721540\\
16.25	3722280\\
16.2666666666667	3723210\\
16.2833333333333	3724130\\
16.3	3725030\\
16.3166666666667	3725790\\
16.3333333333333	3726650\\
16.35	3727590\\
16.3666666666667	3728420\\
16.3833333333333	3729210\\
16.4	3730040\\
16.4166666666667	3730840\\
16.4333333333333	3731640\\
16.45	3732430\\
16.4666666666667	3733210\\
16.4833333333333	3734050\\
16.5	3734870\\
16.5166666666667	3735580\\
16.5333333333333	3736430\\
16.55	3737270\\
16.5666666666667	3738090\\
16.5833333333333	3738830\\
16.6	3739660\\
16.6166666666667	3740300\\
16.6333333333333	3740850\\
16.65	3741750\\
16.6666666666667	3742680\\
16.6833333333333	3743600\\
16.7	3744280\\
16.7166666666667	3745110\\
16.7333333333333	3745980\\
16.75	3746830\\
16.7666666666667	3747630\\
16.7833333333333	3748640\\
16.8	3749660\\
16.8166666666667	3750590\\
16.8333333333333	3751590\\
16.85	3752560\\
16.8666666666667	3753520\\
16.8833333333333	3754400\\
16.9	3755220\\
16.9166666666667	3756180\\
16.9333333333333	3757040\\
16.95	3757910\\
16.9666666666667	3758830\\
16.9833333333333	3759750\\
17	3760700\\
17.0166666666667	3761650\\
17.0333333333333	3762400\\
17.05	3763150\\
17.0666666666667	3764080\\
17.0833333333333	3764790\\
17.1	3765450\\
17.1166666666667	3766230\\
17.1333333333333	3766880\\
17.15	3767550\\
17.1666666666667	3768090\\
17.1833333333333	3768550\\
17.2	3768200\\
17.2166666666667	3767350\\
17.2333333333333	3768550\\
17.25	3769320\\
17.2666666666667	3770110\\
17.2833333333333	3770970\\
17.3	3771590\\
17.3166666666667	3772830\\
17.3333333333333	3773940\\
17.35	3774690\\
17.3666666666667	3775470\\
17.3833333333333	3776150\\
17.4	3776610\\
17.4166666666667	3777360\\
17.4333333333333	3777640\\
17.45	3778140\\
17.4666666666667	3778770\\
17.4833333333333	3779140\\
17.5	3779810\\
17.5166666666667	3780420\\
17.5333333333333	3781030\\
17.55	3781710\\
17.5666666666667	3782400\\
17.5833333333333	3783330\\
17.6	3784180\\
17.6166666666667	3785270\\
17.6333333333333	3786360\\
17.65	3787520\\
17.6666666666667	3788550\\
17.6833333333333	3789480\\
17.7	3790270\\
17.7166666666667	3791030\\
17.7333333333333	3791780\\
17.75	3792420\\
17.7666666666667	3793230\\
17.7833333333333	3793820\\
17.8	3794420\\
17.8166666666667	3794970\\
17.8333333333333	3795740\\
17.85	3796190\\
17.8666666666667	3796800\\
17.8833333333333	3797390\\
17.9	3797970\\
17.9166666666667	3798280\\
17.9333333333333	3798350\\
17.95	3798900\\
17.9666666666667	3799590\\
17.9833333333333	3800380\\
18	3801010\\
18.0166666666667	3801620\\
18.0333333333333	3802350\\
18.05	3803220\\
18.0666666666667	3803960\\
18.0833333333333	3804760\\
18.1	3805830\\
18.1166666666667	3806640\\
18.1333333333333	3807440\\
18.15	3808140\\
18.1666666666667	3808850\\
18.1833333333333	3809640\\
18.2	3810410\\
18.2166666666667	3810930\\
18.2333333333333	3811430\\
18.25	3811990\\
18.2666666666667	3812470\\
18.2833333333333	3812960\\
18.3	3813520\\
18.3166666666667	3814170\\
18.3333333333333	3814570\\
18.35	3815110\\
18.3666666666667	3815730\\
18.3833333333333	3816260\\
18.4	3816980\\
18.4166666666667	3817430\\
18.4333333333333	3818110\\
18.45	3818610\\
18.4666666666667	3819240\\
18.4833333333333	3819890\\
18.5	3820720\\
18.5166666666667	3821480\\
18.5333333333333	3822300\\
18.55	3822740\\
18.5666666666667	3823390\\
18.5833333333333	3824080\\
18.6	3824740\\
18.6166666666667	3825370\\
18.6333333333333	3826030\\
18.65	3826780\\
18.6666666666667	3827440\\
18.6833333333333	3828130\\
18.7	3828840\\
18.7166666666667	3829460\\
18.7333333333333	3830200\\
18.75	3831020\\
18.7666666666667	3831860\\
18.7833333333333	3832560\\
18.8	3833510\\
18.8166666666667	3834280\\
18.8333333333333	3834900\\
18.85	3835650\\
18.8666666666667	3836360\\
18.8833333333333	3837160\\
18.9	3837930\\
18.9166666666667	3838470\\
18.9333333333333	3839410\\
18.95	3839940\\
18.9666666666667	3840690\\
18.9833333333333	3841220\\
19	3841870\\
19.0166666666667	3842470\\
19.0333333333333	3843060\\
19.05	3843640\\
19.0666666666667	3844080\\
19.0833333333333	3844780\\
19.1	3845140\\
19.1166666666667	3845760\\
19.1333333333333	3846320\\
19.15	3846870\\
19.1666666666667	3847450\\
19.1833333333333	3848110\\
19.2	3848830\\
19.2166666666667	3849620\\
19.2333333333333	3850310\\
19.25	3851230\\
19.2666666666667	3852330\\
19.2833333333333	3853180\\
19.3	3853990\\
19.3166666666667	3854810\\
19.3333333333333	3855630\\
19.35	3856360\\
19.3666666666667	3857170\\
19.3833333333333	3857980\\
19.4	3858560\\
19.4166666666667	3859430\\
19.4333333333333	3859990\\
19.45	3860670\\
19.4666666666667	3861370\\
19.4833333333333	3861950\\
19.5	3862690\\
19.5166666666667	3863650\\
19.5333333333333	3864110\\
19.55	3864970\\
19.5666666666667	3865500\\
19.5833333333333	3866260\\
19.6	3866770\\
19.6166666666667	3867510\\
19.6333333333333	3868120\\
19.65	3868840\\
19.6666666666667	3869560\\
19.6833333333333	3870350\\
19.7	3870860\\
19.7166666666667	3871650\\
19.7333333333333	3872260\\
19.75	3873190\\
19.7666666666667	3873730\\
19.7833333333333	3874650\\
19.8	3875230\\
19.8166666666667	3875850\\
19.8333333333333	3876780\\
19.85	3877850\\
19.8666666666667	3878840\\
19.8833333333333	3880150\\
19.9	3881160\\
19.9166666666667	3881820\\
19.9333333333333	3882360\\
19.95	3883100\\
19.9666666666667	3883810\\
19.9833333333333	3884510\\
20	3885430\\
20.0166666666667	3886220\\
20.0333333333333	3886940\\
20.05	3887870\\
20.0666666666667	3888230\\
20.0833333333333	3889080\\
20.1	3890000\\
20.1166666666667	3890770\\
20.1333333333333	3891280\\
20.15	3892070\\
20.1666666666667	3892690\\
20.1833333333333	3893180\\
20.2	3893540\\
20.2166666666667	3893980\\
20.2333333333333	3894440\\
20.25	3894800\\
20.2666666666667	3895300\\
20.2833333333333	3895560\\
20.3	3895990\\
20.3166666666667	3896520\\
20.3333333333333	3897070\\
20.35	3897590\\
20.3666666666667	3898160\\
20.3833333333333	3898950\\
20.4	3899890\\
20.4166666666667	3900950\\
20.4333333333333	3902260\\
20.45	3903280\\
20.4666666666667	3904030\\
20.4833333333333	3904980\\
20.5	3905810\\
20.5166666666667	3906630\\
20.5333333333333	3907250\\
20.55	3907850\\
20.5666666666667	3908720\\
20.5833333333333	3909440\\
20.6	3909990\\
20.6166666666667	3910770\\
20.6333333333333	3911400\\
20.65	3912050\\
20.6666666666667	3912740\\
20.6833333333333	3913210\\
20.7	3913960\\
20.7166666666667	3914380\\
20.7333333333333	3915010\\
20.75	3915450\\
20.7666666666667	3916090\\
20.7833333333333	3916570\\
20.8	3916980\\
20.8166666666667	3917580\\
20.8333333333333	3918160\\
20.85	3918710\\
20.8666666666667	3919240\\
20.8833333333333	3919800\\
20.9	3920310\\
20.9166666666667	3921020\\
20.9333333333333	3921510\\
20.95	3922210\\
20.9666666666667	3922970\\
20.9833333333333	3923620\\
21	3924220\\
21.0166666666667	3924650\\
21.0333333333333	3925370\\
21.05	3926050\\
21.0666666666667	3926340\\
21.0833333333333	3926960\\
21.1	3927150\\
21.1166666666667	3927250\\
21.1333333333333	3927920\\
21.15	3928590\\
21.1666666666667	3929190\\
21.1833333333333	3929900\\
21.2	3930480\\
21.2166666666667	3931160\\
21.2333333333333	3931840\\
21.25	3932430\\
21.2666666666667	3933030\\
21.2833333333333	3933700\\
21.3	3934300\\
21.3166666666667	3934870\\
21.3333333333333	3935650\\
21.35	3936340\\
21.3666666666667	3937140\\
21.3833333333333	3937640\\
21.4	3938550\\
21.4166666666667	3939200\\
21.4333333333333	3939960\\
21.45	3940680\\
21.4666666666667	3941580\\
21.4833333333333	3942190\\
21.5	3942830\\
21.5166666666667	3943540\\
21.5333333333333	3944080\\
21.55	3944820\\
21.5666666666667	3945500\\
21.5833333333333	3946030\\
21.6	3946760\\
21.6166666666667	3947400\\
21.6333333333333	3948070\\
21.65	3948700\\
21.6666666666667	3949210\\
21.6833333333333	3949780\\
21.7	3949960\\
21.7166666666667	3950890\\
21.7333333333333	3951600\\
21.75	3952610\\
21.7666666666667	3953240\\
21.7833333333333	3953880\\
21.8	3954240\\
21.8166666666667	3954690\\
21.8333333333333	3955510\\
21.85	3956060\\
21.8666666666667	3956960\\
21.8833333333333	3957470\\
21.9	3958250\\
21.9166666666667	3958880\\
21.9333333333333	3959400\\
21.95	3960170\\
21.9666666666667	3960790\\
21.9833333333333	3961370\\
22	3961800\\
22.0166666666667	3962350\\
22.0333333333333	3962730\\
22.05	3963150\\
22.0666666666667	3963430\\
22.0833333333333	3963790\\
22.1	3933340\\
22.1166666666667	3473830\\
22.1333333333333	3568520\\
22.15	2377730\\
22.1666666666667	3035540\\
22.1833333333333	3039390\\
22.2	3054600\\
22.2166666666667	3072540\\
22.2333333333333	3090140\\
22.25	3106850\\
22.2666666666667	3126480\\
22.2833333333333	3146300\\
22.3	3165900\\
22.3166666666667	3185480\\
22.3333333333333	3204140\\
22.35	3221620\\
22.3666666666667	3237880\\
22.3833333333333	3253460\\
22.4	3267040\\
22.4166666666667	3275450\\
22.4333333333333	3287240\\
22.45	3296990\\
22.4666666666667	3305850\\
22.4833333333333	3314320\\
22.5	3322020\\
22.5166666666667	3329430\\
22.5333333333333	3335960\\
22.55	3342270\\
22.5666666666667	3348260\\
22.5833333333333	3353710\\
22.6	3359150\\
22.6166666666667	3364240\\
22.6333333333333	3369270\\
22.65	3374010\\
22.6666666666667	3378280\\
22.6833333333333	3382440\\
22.7	3386130\\
22.7166666666667	3389550\\
22.7333333333333	3393290\\
22.75	3396600\\
22.7666666666667	3399770\\
22.7833333333333	3403090\\
22.8	3406000\\
22.8166666666667	3408670\\
22.8333333333333	3411060\\
22.85	3413490\\
22.8666666666667	3415790\\
22.8833333333333	3419100\\
22.9	3421070\\
22.9166666666667	3423180\\
22.9333333333333	3425030\\
22.95	3426780\\
22.9666666666667	3428570\\
22.9833333333333	3430550\\
23	3432030\\
23.0166666666667	3433170\\
23.0333333333333	3434390\\
23.05	3434980\\
23.0666666666667	3436130\\
23.0833333333333	3437910\\
23.1	3438960\\
23.1166666666667	3440300\\
23.1333333333333	3441790\\
23.15	3442930\\
23.1666666666667	3444240\\
23.1833333333333	3445810\\
23.2	3447080\\
23.2166666666667	3448550\\
23.2333333333333	3406780\\
23.25	3146580\\
23.2666666666667	1824620\\
23.2833333333333	2894760\\
23.3	3105410\\
23.3166666666667	3200920\\
23.3333333333333	3275860\\
23.35	3478800\\
23.3666666666667	3559880\\
23.3833333333333	3614370\\
23.4	3659220\\
23.4166666666667	3695760\\
23.4333333333333	3730330\\
23.45	3761290\\
23.4666666666667	3787200\\
23.4833333333333	3812170\\
23.5	3834560\\
23.5166666666667	3855240\\
23.5333333333333	3874150\\
23.55	3892500\\
23.5666666666667	3909930\\
23.5833333333333	3925770\\
23.6	3941540\\
23.6166666666667	3956590\\
23.6333333333333	3970870\\
23.65	3984660\\
23.6666666666667	3997070\\
23.6833333333333	4009660\\
23.7	4021970\\
23.7166666666667	4033480\\
23.7333333333333	4045190\\
23.75	4056350\\
23.7666666666667	4067670\\
23.7833333333333	4079780\\
23.8	4092430\\
23.8166666666667	4103210\\
23.8333333333333	4113270\\
23.85	4123190\\
23.8666666666667	4132810\\
23.8833333333333	4143410\\
23.9	4153160\\
23.9166666666667	4162280\\
23.9333333333333	4171020\\
23.95	4179710\\
23.9666666666667	4188260\\
23.9833333333333	4196470\\
24	4204350\\
24.0166666666667	4211850\\
24.0333333333333	4219190\\
24.05	4227170\\
24.0666666666667	4234780\\
24.0833333333333	4241320\\
24.1	4248760\\
24.1166666666667	4256470\\
24.1333333333333	4262930\\
24.15	4269380\\
24.1666666666667	4276100\\
24.1833333333333	4283820\\
24.2	4289900\\
24.2166666666667	4295990\\
24.2333333333333	4302010\\
24.25	4308340\\
24.2666666666667	4314300\\
24.2833333333333	4320450\\
24.3	4326210\\
24.3166666666667	4331920\\
24.3333333333333	4337440\\
24.35	4342670\\
24.3666666666667	4347860\\
24.3833333333333	4353130\\
24.4	4358750\\
24.4166666666667	4363880\\
24.4333333333333	4369340\\
24.45	4373810\\
24.4666666666667	4379050\\
24.4833333333333	4382950\\
24.5	4387410\\
24.5166666666667	4392330\\
24.5333333333333	4396940\\
24.55	4401880\\
24.5666666666667	4406410\\
24.5833333333333	4410920\\
24.6	4415340\\
24.6166666666667	4419580\\
24.6333333333333	4423910\\
24.65	4428150\\
24.6666666666667	4432510\\
24.6833333333333	4436560\\
24.7	4440470\\
24.7166666666667	4444610\\
24.7333333333333	4448390\\
24.75	4452500\\
24.7666666666667	4456570\\
24.7833333333333	4460800\\
24.8	4465140\\
24.8166666666667	4469040\\
24.8333333333333	4472730\\
24.85	4476710\\
24.8666666666667	4480350\\
24.8833333333333	4484300\\
24.9	4487910\\
24.9166666666667	4491440\\
24.9333333333333	4495250\\
24.95	4498740\\
24.9666666666667	4501840\\
24.9833333333333	4505190\\
25	4508540\\
25.0166666666667	4512040\\
25.0333333333333	4515350\\
25.05	4518870\\
25.0666666666667	4522070\\
25.0833333333333	4525320\\
25.1	4528670\\
25.1166666666667	4531720\\
25.1333333333333	4534600\\
25.15	4537710\\
25.1666666666667	4540740\\
25.1833333333333	4543850\\
25.2	4546840\\
25.2166666666667	4553580\\
25.2333333333333	4557030\\
25.25	4559730\\
25.2666666666667	4562070\\
25.2833333333333	4564340\\
25.3	4567020\\
25.3166666666667	4570140\\
25.3333333333333	4576110\\
25.35	4578500\\
25.3666666666667	4580840\\
25.3833333333333	4583500\\
25.4	4586210\\
25.4166666666667	4589070\\
25.4333333333333	4591850\\
25.45	4594420\\
25.4666666666667	4597090\\
25.4833333333333	4599600\\
25.5	4602300\\
25.5166666666667	4604920\\
25.5333333333333	4607320\\
25.55	4610050\\
25.5666666666667	4612580\\
25.5833333333333	4614880\\
25.6	4617480\\
25.6166666666667	4620270\\
25.6333333333333	4622730\\
25.65	4625070\\
25.6666666666667	4627390\\
25.6833333333333	4629780\\
25.7	4632340\\
25.7166666666667	4634630\\
25.7333333333333	4636980\\
25.75	4639240\\
25.7666666666667	4641620\\
25.7833333333333	4643310\\
25.8	4645550\\
25.8166666666667	4647720\\
25.8333333333333	4649640\\
25.85	4651850\\
25.8666666666667	4653860\\
25.8833333333333	4656120\\
25.9	4658700\\
25.9166666666667	4660510\\
25.9333333333333	4662820\\
25.95	4664830\\
25.9666666666667	4666930\\
25.9833333333333	4668820\\
26	4670880\\
26.0166666666667	4672880\\
26.0333333333333	4675100\\
26.05	4677230\\
26.0666666666667	4679220\\
26.0833333333333	4681140\\
26.1	4683110\\
26.1166666666667	4685250\\
26.1333333333333	4687650\\
26.15	4689830\\
26.1666666666667	4692000\\
26.1833333333333	4693620\\
26.2	4695150\\
26.2166666666667	4696860\\
26.2333333333333	4698380\\
26.25	4699060\\
26.2666666666667	4700330\\
26.2833333333333	4702260\\
26.3	4703600\\
26.3166666666667	4705860\\
26.3333333333333	4707400\\
26.35	4708570\\
26.3666666666667	4709840\\
26.3833333333333	4710420\\
26.4	4712120\\
26.4166666666667	4713600\\
26.4333333333333	4715250\\
26.45	4717630\\
26.4666666666667	4719320\\
26.4833333333333	4720860\\
26.5	4722720\\
26.5166666666667	4724310\\
26.5333333333333	4725820\\
26.55	4727350\\
26.5666666666667	4729290\\
26.5833333333333	4731180\\
26.6	4732080\\
26.6166666666667	4734000\\
26.6333333333333	4735010\\
26.65	4736610\\
26.6666666666667	4737980\\
26.6833333333333	4739430\\
26.7	4739990\\
26.7166666666667	4741520\\
26.7333333333333	4742740\\
26.75	4744110\\
26.7666666666667	4745470\\
26.7833333333333	4746870\\
26.8	4748260\\
26.8166666666667	4749860\\
26.8333333333333	4751390\\
26.85	4752690\\
26.8666666666667	4754170\\
26.8833333333333	4755550\\
26.9	4756570\\
26.9166666666667	4757850\\
26.9333333333333	4759320\\
26.95	4761160\\
26.9666666666667	4762480\\
26.9833333333333	4764220\\
27	4765290\\
27.0166666666667	4766540\\
27.0333333333333	4767960\\
27.05	4769210\\
27.0666666666667	4770670\\
27.0833333333333	4771590\\
27.1	4772900\\
27.1166666666667	4774070\\
27.1333333333333	4774450\\
27.15	4775680\\
27.1666666666667	4776610\\
27.1833333333333	4777690\\
27.2	4778750\\
27.2166666666667	4779790\\
27.2333333333333	4781190\\
27.25	4782560\\
27.2666666666667	4783720\\
27.2833333333333	4784750\\
27.3	4785880\\
27.3166666666667	4787280\\
27.3333333333333	4788330\\
27.35	4789490\\
27.3666666666667	4790690\\
27.3833333333333	4792000\\
27.4	4792910\\
27.4166666666667	4793820\\
27.4333333333333	4794660\\
27.45	4795970\\
27.4666666666667	4797040\\
27.4833333333333	4798000\\
27.5	4799230\\
27.5166666666667	4800530\\
27.5333333333333	4801720\\
27.55	4803020\\
27.5666666666667	4804090\\
27.5833333333333	4804900\\
27.6	4806240\\
27.6166666666667	4807050\\
27.6333333333333	4808060\\
27.65	4809070\\
27.6666666666667	4809960\\
27.6833333333333	4810850\\
27.7	4811750\\
27.7166666666667	4812660\\
27.7333333333333	4813600\\
27.75	4814270\\
27.7666666666667	4814990\\
27.7833333333333	4815930\\
27.8	4816740\\
27.8166666666667	4817200\\
27.8333333333333	4818300\\
27.85	4819260\\
27.8666666666667	4819770\\
27.8833333333333	4820670\\
27.9	4821770\\
27.9166666666667	4822720\\
27.9333333333333	4823770\\
27.95	4824980\\
27.9666666666667	4825960\\
27.9833333333333	4827010\\
28	4828300\\
28.0166666666667	4829020\\
28.0333333333333	4830010\\
28.05	4830850\\
28.0666666666667	4831550\\
28.0833333333333	4832690\\
28.1	4833450\\
28.1166666666667	4834520\\
28.1333333333333	4835480\\
28.15	4836460\\
28.1666666666667	4837690\\
28.1833333333333	4838220\\
28.2	4839440\\
28.2166666666667	4840230\\
28.2333333333333	4841070\\
28.25	4841980\\
28.2666666666667	4842780\\
28.2833333333333	4843940\\
28.3	4844000\\
28.3166666666667	4844690\\
28.3333333333333	4845660\\
28.35	4846400\\
28.3666666666667	4846980\\
28.3833333333333	4847920\\
28.4	4848680\\
28.4166666666667	4849450\\
28.4333333333333	4850310\\
28.45	4851090\\
28.4666666666667	4851660\\
28.4833333333333	4852620\\
28.5	4853350\\
28.5166666666667	4854100\\
28.5333333333333	4855060\\
28.55	4855790\\
28.5666666666667	4856400\\
28.5833333333333	4857140\\
28.6	4857700\\
28.6166666666667	4858390\\
28.6333333333333	4859150\\
28.65	4860730\\
28.6666666666667	4861690\\
28.6833333333333	4862330\\
28.7	4863070\\
28.7166666666667	4863970\\
28.7333333333333	4864770\\
28.75	4865400\\
28.7666666666667	4866140\\
28.7833333333333	4866590\\
28.8	4866810\\
28.8166666666667	4867250\\
28.8333333333333	4867930\\
28.85	4868520\\
28.8666666666667	4869050\\
28.8833333333333	4869700\\
28.9	4870270\\
28.9166666666667	4870840\\
28.9333333333333	4871590\\
28.95	4872150\\
28.9666666666667	4872520\\
28.9833333333333	4873250\\
29	4873710\\
29.0166666666667	4874350\\
29.0333333333333	4874910\\
29.05	4875830\\
29.0666666666667	4876650\\
29.0833333333333	4877220\\
29.1	4878020\\
29.1166666666667	4878710\\
29.1333333333333	4879230\\
29.15	4879920\\
29.1666666666667	4880660\\
29.1833333333333	4881240\\
29.2	4881970\\
29.2166666666667	4882300\\
29.2333333333333	4883170\\
29.25	4883660\\
29.2666666666667	4884210\\
29.2833333333333	4884900\\
29.3	4885230\\
29.3166666666667	4885850\\
29.3333333333333	4886430\\
29.35	4886840\\
29.3666666666667	4887470\\
29.3833333333333	4888100\\
29.4	4888740\\
29.4166666666667	4889190\\
29.4333333333333	4889500\\
29.45	4889720\\
29.4666666666667	4890400\\
29.4833333333333	4890650\\
29.5	4891190\\
29.5166666666667	4891390\\
29.5333333333333	4892110\\
29.55	4892680\\
29.5666666666667	4892980\\
29.5833333333333	4893910\\
29.6	4894930\\
29.6166666666667	4895610\\
29.6333333333333	4896620\\
29.65	4896330\\
29.6666666666667	4897010\\
29.6833333333333	4897580\\
29.7	4898060\\
29.7166666666667	4898750\\
29.7333333333333	4899390\\
29.75	4899910\\
29.7666666666667	4900290\\
29.7833333333333	4900990\\
29.8	4901490\\
29.8166666666667	4901990\\
29.8333333333333	4902400\\
29.85	4903040\\
29.8666666666667	4903290\\
29.8833333333333	4904000\\
29.9	4904480\\
29.9166666666667	4905060\\
29.9333333333333	4905300\\
29.95	4905930\\
29.9666666666667	4906340\\
29.9833333333333	4906820\\
};

\addplot [color=mycolor6]
  table[row sep=crcr]{%
0	13462900\\
0.0166666666666667	13455300\\
0.0333333333333333	13512200\\
0.05	13545600\\
0.0666666666666667	13612000\\
0.0833333333333333	13613500\\
0.1	13625900\\
0.116666666666667	13671200\\
0.133333333333333	13741900\\
0.15	13763900\\
0.166666666666667	13902200\\
0.183333333333333	14077600\\
0.2	14247000\\
0.216666666666667	14375800\\
0.233333333333333	14486200\\
0.25	14596400\\
0.266666666666667	14699500\\
0.283333333333333	14807000\\
0.3	14928600\\
0.316666666666667	15054300\\
0.333333333333333	15149500\\
0.35	15236000\\
0.366666666666667	15330100\\
0.383333333333333	15415900\\
0.4	15519300\\
0.416666666666667	15608000\\
0.433333333333333	15707800\\
0.45	15785500\\
0.466666666666667	15894300\\
0.483333333333333	16052300\\
0.5	16193300\\
0.516666666666667	16329800\\
0.533333333333333	16417500\\
0.55	16499000\\
0.566666666666667	16613100\\
0.583333333333333	16742800\\
0.6	16896100\\
0.616666666666667	17067800\\
0.633333333333333	17262300\\
0.65	17460900\\
0.666666666666667	17667400\\
0.683333333333333	17881600\\
0.7	18111100\\
0.716666666666667	18358300\\
0.733333333333333	18608500\\
0.75	18840400\\
0.766666666666667	19048500\\
0.783333333333333	19268200\\
0.8	19467500\\
0.816666666666667	19629200\\
0.833333333333333	19800300\\
0.85	19980300\\
0.866666666666667	20173700\\
0.883333333333333	20399700\\
0.9	20636000\\
0.916666666666667	20904200\\
0.933333333333333	21205400\\
0.95	21541800\\
0.966666666666667	21928900\\
0.983333333333333	22366400\\
1	22858600\\
1.01666666666667	23379800\\
1.03333333333333	23920000\\
1.05	24509900\\
1.06666666666667	25186900\\
1.08333333333333	25860600\\
1.1	26508300\\
1.11666666666667	27170900\\
1.13333333333333	27840000\\
1.15	28528900\\
1.16666666666667	29190300\\
1.18333333333333	29802400\\
1.2	30326600\\
1.21666666666667	30854300\\
1.23333333333333	31351000\\
1.25	31839100\\
1.26666666666667	32313500\\
1.28333333333333	32752000\\
1.3	33165100\\
1.31666666666667	33567900\\
1.33333333333333	33904100\\
1.35	34248600\\
1.36666666666667	34619800\\
1.38333333333333	34977100\\
1.4	35301900\\
1.41666666666667	35610400\\
1.43333333333333	35956400\\
1.45	36327200\\
1.46666666666667	36712800\\
1.48333333333333	37085700\\
1.5	37480900\\
1.51666666666667	37889700\\
1.53333333333333	38325100\\
1.55	38718700\\
1.56666666666667	39118400\\
1.58333333333333	39500100\\
1.6	39850500\\
1.61666666666667	40147800\\
1.63333333333333	40375800\\
1.65	40638200\\
1.66666666666667	40899500\\
1.68333333333333	41159600\\
1.7	41371500\\
1.71666666666667	41638600\\
1.73333333333333	42011400\\
1.75	42381200\\
1.76666666666667	42707100\\
1.78333333333333	43020800\\
1.8	43348400\\
1.81666666666667	43726000\\
1.83333333333333	44178800\\
1.85	44635900\\
1.86666666666667	45092400\\
1.88333333333333	45516900\\
1.9	45908300\\
1.91666666666667	46349800\\
1.93333333333333	46764600\\
1.95	47151800\\
1.96666666666667	47455300\\
1.98333333333333	47775900\\
2	48159200\\
2.01666666666667	48493000\\
2.03333333333333	48763500\\
2.05	49051500\\
2.06666666666667	49328900\\
2.08333333333333	49641200\\
2.1	49964200\\
2.11666666666667	50264800\\
2.13333333333333	50445000\\
2.15	50612800\\
2.16666666666667	50736500\\
2.18333333333333	50775900\\
2.2	50723800\\
2.21666666666667	50733800\\
2.23333333333333	50930700\\
2.25	51315800\\
2.26666666666667	52269200\\
2.28333333333333	52918200\\
2.3	53313900\\
2.31666666666667	53694500\\
2.33333333333333	54133400\\
2.35	54549600\\
2.36666666666667	54836100\\
2.38333333333333	54968600\\
2.4	55057500\\
2.41666666666667	54885300\\
2.43333333333333	54670600\\
2.45	54428800\\
2.46666666666667	54163700\\
2.48333333333333	53927200\\
2.5	53721000\\
2.51666666666667	53566700\\
2.53333333333333	53516900\\
2.55	53515300\\
2.56666666666667	53613100\\
2.58333333333333	53798100\\
2.6	54126600\\
2.61666666666667	54458700\\
2.63333333333333	54798100\\
2.65	55151400\\
2.66666666666667	55491200\\
2.68333333333333	55777800\\
2.7	56154100\\
2.71666666666667	56641400\\
2.73333333333333	57176700\\
2.75	57753000\\
2.76666666666667	58328800\\
2.78333333333333	58914300\\
2.8	59466500\\
2.81666666666667	59812000\\
2.83333333333333	59762800\\
2.85	59546400\\
2.86666666666667	59268600\\
2.88333333333333	59182800\\
2.9	59077400\\
2.91666666666667	58997900\\
2.93333333333333	58950900\\
2.95	59005700\\
2.96666666666667	59141300\\
2.98333333333333	59333700\\
3	59477800\\
3.01666666666667	59497100\\
3.03333333333333	59382300\\
3.05	59334000\\
3.06666666666667	59369300\\
3.08333333333333	59410900\\
3.1	59430200\\
3.11666666666667	59707400\\
3.13333333333333	60039900\\
3.15	60466800\\
3.16666666666667	61023900\\
3.18333333333333	61594500\\
3.2	62079400\\
3.21666666666667	62469700\\
3.23333333333333	62824400\\
3.25	63174100\\
3.26666666666667	63428900\\
3.28333333333333	63594900\\
3.3	63806900\\
3.31666666666667	63914000\\
3.33333333333333	63912300\\
3.35	63964700\\
3.36666666666667	64339400\\
3.38333333333333	64727100\\
3.4	65066200\\
3.41666666666667	65323900\\
3.43333333333333	65367500\\
3.45	65484100\\
3.46666666666667	65395800\\
3.48333333333333	65318000\\
3.5	65312600\\
3.51666666666667	65339600\\
3.53333333333333	65409400\\
3.55	65529700\\
3.56666666666667	65718300\\
3.58333333333333	65926200\\
3.6	66135800\\
3.61666666666667	66262100\\
3.63333333333333	66456100\\
3.65	66635900\\
3.66666666666667	66690500\\
3.68333333333333	66680100\\
3.7	66796600\\
3.71666666666667	66969200\\
3.73333333333333	67028700\\
3.75	66996000\\
3.76666666666667	66959200\\
3.78333333333333	66933100\\
3.8	66771300\\
3.81666666666667	66646700\\
3.83333333333333	66545900\\
3.85	66475100\\
3.86666666666667	66315200\\
3.88333333333333	66144200\\
3.9	66142400\\
3.91666666666667	66213700\\
3.93333333333333	66397900\\
3.95	66632200\\
3.96666666666667	66785600\\
3.98333333333333	66884900\\
4	67041900\\
4.01666666666667	67333100\\
4.03333333333333	67734500\\
4.05	68130600\\
4.06666666666667	68632700\\
4.08333333333333	69067700\\
4.1	69341400\\
4.11666666666667	69675100\\
4.13333333333333	69894900\\
4.15	70118600\\
4.16666666666667	70182500\\
4.18333333333333	70169900\\
4.2	70086100\\
4.21666666666667	69972100\\
4.23333333333333	69902800\\
4.25	69735100\\
4.26666666666667	69642400\\
4.28333333333333	69598300\\
4.3	69632100\\
4.31666666666667	69539900\\
4.33333333333333	69445000\\
4.35	69328800\\
4.36666666666667	69230200\\
4.38333333333333	69181600\\
4.4	69175700\\
4.41666666666667	69226600\\
4.43333333333333	69358300\\
4.45	69464300\\
4.46666666666667	69638900\\
4.48333333333333	69915200\\
4.5	70217500\\
4.51666666666667	70667600\\
4.53333333333333	71456800\\
4.55	72254900\\
4.56666666666667	73100900\\
4.58333333333333	73646200\\
4.6	74180500\\
4.61666666666667	74514100\\
4.63333333333333	74714300\\
4.65	74787700\\
4.66666666666667	74915500\\
4.68333333333333	75089900\\
4.7	75305600\\
4.71666666666667	75518500\\
4.73333333333333	75626100\\
4.75	75782300\\
4.76666666666667	75831100\\
4.78333333333333	75955000\\
4.8	76115700\\
4.81666666666667	76144900\\
4.83333333333333	76201800\\
4.85	76287700\\
4.86666666666667	76298500\\
4.88333333333333	76363700\\
4.9	76428700\\
4.91666666666667	76491500\\
4.93333333333333	76419500\\
4.95	76273700\\
4.96666666666667	76176300\\
4.98333333333333	75989100\\
5	75910300\\
5.01666666666667	76018600\\
5.03333333333333	76231800\\
5.05	76459800\\
5.06666666666667	76653300\\
5.08333333333333	76818700\\
5.1	76979300\\
5.11666666666667	77336100\\
5.13333333333333	77768800\\
5.15	78009400\\
5.16666666666667	78323400\\
5.18333333333333	78969200\\
5.2	79671600\\
5.21666666666667	80164900\\
5.23333333333333	80359400\\
5.25	80231600\\
5.26666666666667	80233300\\
5.28333333333333	80813200\\
5.3	81558800\\
5.31666666666667	82067600\\
5.33333333333333	82548700\\
5.35	82971300\\
5.36666666666667	83210800\\
5.38333333333333	83202000\\
5.4	83077300\\
5.41666666666667	82931700\\
5.43333333333333	82834600\\
5.45	82916300\\
5.46666666666667	83220000\\
5.48333333333333	83335500\\
5.5	83268300\\
5.51666666666667	83094300\\
5.53333333333333	82951300\\
5.55	82699900\\
5.56666666666667	82533400\\
5.58333333333333	82440500\\
5.6	82419700\\
5.61666666666667	82265700\\
5.63333333333333	82204100\\
5.65	82233100\\
5.66666666666667	82237400\\
5.68333333333333	82151100\\
5.7	82256600\\
5.71666666666667	82324100\\
5.73333333333333	82394600\\
5.75	82482600\\
5.76666666666667	82608000\\
5.78333333333333	82730400\\
5.8	82797200\\
5.81666666666667	82872700\\
5.83333333333333	83042900\\
5.85	83199200\\
5.86666666666667	83462900\\
5.88333333333333	83731500\\
5.9	83963600\\
5.91666666666667	84110400\\
5.93333333333333	84122500\\
5.95	84074800\\
5.96666666666667	84092900\\
5.98333333333333	84214700\\
6	84476900\\
6.01666666666667	84708200\\
6.03333333333333	84999000\\
6.05	84887500\\
6.06666666666667	84777600\\
6.08333333333333	84546900\\
6.1	84420700\\
6.11666666666667	84247700\\
6.13333333333333	83993200\\
6.15	83920700\\
6.16666666666667	84182600\\
6.18333333333333	84461000\\
6.2	84603700\\
6.21666666666667	84585900\\
6.23333333333333	84618000\\
6.25	84746300\\
6.26666666666667	85007600\\
6.28333333333333	84692500\\
6.3	84674300\\
6.31666666666667	84756000\\
6.33333333333333	84890700\\
6.35	85177700\\
6.36666666666667	85441900\\
6.38333333333333	85674100\\
6.4	85932200\\
6.41666666666667	86111100\\
6.43333333333333	86372500\\
6.45	86573800\\
6.46666666666667	86966200\\
6.48333333333333	87289700\\
6.5	87527400\\
6.51666666666667	87585900\\
6.53333333333333	87602600\\
6.55	87367100\\
6.56666666666667	87164400\\
6.58333333333333	86842200\\
6.6	86746200\\
6.61666666666667	86915300\\
6.63333333333333	86988700\\
6.65	87109800\\
6.66666666666667	87298900\\
6.68333333333333	87568300\\
6.7	87685900\\
6.71666666666667	87643200\\
6.73333333333333	87692600\\
6.75	87877300\\
6.76666666666667	88218800\\
6.78333333333333	88451000\\
6.8	88771100\\
6.81666666666667	89080700\\
6.83333333333333	89306300\\
6.85	89399900\\
6.86666666666667	89535000\\
6.88333333333333	89602000\\
6.9	89848400\\
6.91666666666667	90429200\\
6.93333333333333	90748800\\
6.95	90956600\\
6.96666666666667	90988200\\
6.98333333333333	91061100\\
7	91016200\\
7.01666666666667	90794200\\
7.03333333333333	90745200\\
7.05	90781900\\
7.06666666666667	90865700\\
7.08333333333333	91419100\\
7.1	92156800\\
7.11666666666667	92891900\\
7.13333333333333	93608300\\
7.15	94439600\\
7.16666666666667	94826600\\
7.18333333333333	95139500\\
7.2	95480000\\
7.21666666666667	95578300\\
7.23333333333333	95517600\\
7.25	95228800\\
7.26666666666667	95120000\\
7.28333333333333	94902100\\
7.3	94865400\\
7.31666666666667	94716200\\
7.33333333333333	94657900\\
7.35	94710300\\
7.36666666666667	94778900\\
7.38333333333333	94974100\\
7.4	95219800\\
7.41666666666667	95654600\\
7.43333333333333	95912300\\
7.45	96278200\\
7.46666666666667	96623400\\
7.48333333333333	96891700\\
7.5	96925000\\
7.51666666666667	96640800\\
7.53333333333333	96386800\\
7.55	96399400\\
7.56666666666667	96560000\\
7.58333333333333	96812700\\
7.6	96979900\\
7.61666666666667	97156100\\
7.63333333333333	97198000\\
7.65	97219500\\
7.66666666666667	97372700\\
7.68333333333333	97667300\\
7.7	98232400\\
7.71666666666667	98738600\\
7.73333333333333	99064600\\
7.75	99144900\\
7.76666666666667	99235300\\
7.78333333333333	99406300\\
7.8	99370400\\
7.81666666666667	99385300\\
7.83333333333333	99353200\\
7.85	99613600\\
7.86666666666667	99198800\\
7.88333333333333	99285500\\
7.9	99285400\\
7.91666666666667	99315500\\
7.93333333333333	99174400\\
7.95	99081200\\
7.96666666666667	99033500\\
7.98333333333333	98999100\\
8	98958700\\
8.01666666666667	98746100\\
8.03333333333333	98602000\\
8.05	98598300\\
8.06666666666667	98615300\\
8.08333333333333	98808000\\
8.1	98836300\\
8.11666666666667	98891000\\
8.13333333333333	99022100\\
8.15	99204100\\
8.16666666666667	99300200\\
8.18333333333333	99484700\\
8.2	99727200\\
8.21666666666667	99863600\\
8.23333333333333	99903600\\
8.25	99889900\\
8.26666666666667	99793400\\
8.28333333333333	99576900\\
8.3	99692400\\
8.31666666666667	99831500\\
8.33333333333333	100158000\\
8.35	100488000\\
8.36666666666667	100655000\\
8.38333333333333	100923000\\
8.4	101303000\\
8.41666666666667	101578000\\
8.43333333333333	101739000\\
8.45	102053000\\
8.46666666666667	102336000\\
8.48333333333333	102976000\\
8.5	103577000\\
8.51666666666667	104135000\\
8.53333333333333	104460000\\
8.55	104667000\\
8.56666666666667	104911000\\
8.58333333333333	105076000\\
8.6	105172000\\
8.61666666666667	105261000\\
8.63333333333333	105127000\\
8.65	105013000\\
8.66666666666667	104966000\\
8.68333333333333	104880000\\
8.7	104777000\\
8.71666666666667	104641000\\
8.73333333333333	104431000\\
8.75	104331000\\
8.76666666666667	104428000\\
8.78333333333333	104503000\\
8.8	104420000\\
8.81666666666667	104602000\\
8.83333333333333	104721000\\
8.85	104761000\\
8.86666666666667	104905000\\
8.88333333333333	105077000\\
8.9	105017000\\
8.91666666666667	105117000\\
8.93333333333333	105067000\\
8.95	105191000\\
8.96666666666667	105208000\\
8.98333333333333	105364000\\
9	105439000\\
9.01666666666667	105515000\\
9.03333333333333	105766000\\
9.05	105780000\\
9.06666666666667	105974000\\
9.08333333333333	106094000\\
9.1	106134000\\
9.11666666666667	106179000\\
9.13333333333333	106264000\\
9.15	106375000\\
9.16666666666667	106163000\\
9.18333333333333	106230000\\
9.2	106331000\\
9.21666666666667	106278000\\
9.23333333333333	106310000\\
9.25	106285000\\
9.26666666666667	106465000\\
9.28333333333333	106531000\\
9.3	106365000\\
9.31666666666667	106423000\\
9.33333333333333	106323000\\
9.35	106249000\\
9.36666666666667	106126000\\
9.38333333333333	106192000\\
9.4	106039000\\
9.41666666666667	106178000\\
9.43333333333333	106067000\\
9.45	106041000\\
9.46666666666667	106058000\\
9.48333333333333	105950000\\
9.5	106057000\\
9.51666666666667	106012000\\
9.53333333333333	105967000\\
9.55	105973000\\
9.56666666666667	106073000\\
9.58333333333333	106052000\\
9.6	106053000\\
9.61666666666667	106109000\\
9.63333333333333	105968000\\
9.65	105823000\\
9.66666666666667	105746000\\
9.68333333333333	105770000\\
9.7	105772000\\
9.71666666666667	105971000\\
9.73333333333333	105970000\\
9.75	106026000\\
9.76666666666667	105964000\\
9.78333333333333	106152000\\
9.8	105900000\\
9.81666666666667	106040000\\
9.83333333333333	106168000\\
9.85	106259000\\
9.86666666666667	106339000\\
9.88333333333333	106287000\\
9.9	106173000\\
9.91666666666667	106223000\\
9.93333333333333	106156000\\
9.95	106106000\\
9.96666666666667	106330000\\
9.98333333333333	106311000\\
10	106249000\\
10.0166666666667	106321000\\
10.0333333333333	106222000\\
10.05	106260000\\
10.0666666666667	106215000\\
10.0833333333333	106252000\\
10.1	106224000\\
10.1166666666667	106229000\\
10.1333333333333	106348000\\
10.15	106382000\\
10.1666666666667	106443000\\
10.1833333333333	106487000\\
10.2	106514000\\
10.2166666666667	106439000\\
10.2333333333333	106698000\\
10.25	106676000\\
10.2666666666667	106782000\\
10.2833333333333	106960000\\
10.3	107133000\\
10.3166666666667	106888000\\
10.3333333333333	107207000\\
10.35	107219000\\
10.3666666666667	107139000\\
10.3833333333333	107229000\\
10.4	107015000\\
10.4166666666667	107122000\\
10.4333333333333	107011000\\
10.45	106868000\\
10.4666666666667	106616000\\
10.4833333333333	106561000\\
10.5	106445000\\
10.5166666666667	106575000\\
10.5333333333333	106306000\\
10.55	106469000\\
10.5666666666667	106247000\\
10.5833333333333	106218000\\
10.6	106246000\\
10.6166666666667	106354000\\
10.6333333333333	106314000\\
10.65	106368000\\
10.6666666666667	106360000\\
10.6833333333333	106483000\\
10.7	106871000\\
10.7166666666667	106883000\\
10.7333333333333	106939000\\
10.75	107025000\\
10.7666666666667	106975000\\
10.7833333333333	106945000\\
10.8	106870000\\
10.8166666666667	106758000\\
10.8333333333333	106964000\\
10.85	106886000\\
10.8666666666667	106678000\\
10.8833333333333	106838000\\
10.9	106776000\\
10.9166666666667	106554000\\
10.9333333333333	106541000\\
10.95	106427000\\
10.9666666666667	106403000\\
10.9833333333333	106319000\\
11	106368000\\
11.0166666666667	106238000\\
11.0333333333333	106194000\\
11.05	106387000\\
11.0666666666667	106275000\\
11.0833333333333	106205000\\
11.1	106338000\\
11.1166666666667	106232000\\
11.1333333333333	106252000\\
11.15	106439000\\
11.1666666666667	106572000\\
11.1833333333333	106630000\\
11.2	106675000\\
11.2166666666667	106835000\\
11.2333333333333	106905000\\
11.25	106985000\\
11.2666666666667	106777000\\
11.2833333333333	107030000\\
11.3	106951000\\
11.3166666666667	107090000\\
11.3333333333333	107038000\\
11.35	106854000\\
11.3666666666667	106877000\\
11.3833333333333	106798000\\
11.4	106839000\\
11.4166666666667	106752000\\
11.4333333333333	106577000\\
11.45	106543000\\
11.4666666666667	106422000\\
11.4833333333333	106295000\\
11.5	106127000\\
11.5166666666667	106028000\\
11.5333333333333	105967000\\
11.55	105998000\\
11.5666666666667	106102000\\
11.5833333333333	106136000\\
11.6	106102000\\
11.6166666666667	106460000\\
11.6333333333333	106361000\\
11.65	106526000\\
11.6666666666667	106750000\\
11.6833333333333	106631000\\
11.7	106828000\\
11.7166666666667	106767000\\
11.7333333333333	106716000\\
11.75	106561000\\
11.7666666666667	106472000\\
11.7833333333333	106445000\\
11.8	106421000\\
11.8166666666667	106345000\\
11.8333333333333	106216000\\
11.85	106071000\\
11.8666666666667	105927000\\
11.8833333333333	105842000\\
11.9	105972000\\
11.9166666666667	105753000\\
11.9333333333333	105879000\\
11.95	105808000\\
11.9666666666667	106397000\\
11.9833333333333	106471000\\
12	106264000\\
12.0166666666667	106518000\\
12.0333333333333	106763000\\
12.05	106663000\\
12.0666666666667	106760000\\
12.0833333333333	106629000\\
12.1	106497000\\
12.1166666666667	106551000\\
12.1333333333333	106263000\\
12.15	106399000\\
12.1666666666667	106055000\\
12.1833333333333	105896000\\
12.2	105803000\\
12.2166666666667	105561000\\
12.2333333333333	105715000\\
12.25	105667000\\
12.2666666666667	105654000\\
12.2833333333333	105648000\\
12.3	105510000\\
12.3166666666667	105744000\\
12.3333333333333	105888000\\
12.35	105847000\\
12.3666666666667	106044000\\
12.3833333333333	106272000\\
12.4	106274000\\
12.4166666666667	106325000\\
12.4333333333333	106170000\\
12.45	106021000\\
12.4666666666667	106103000\\
12.4833333333333	105984000\\
12.5	105715000\\
12.5166666666667	105536000\\
12.5333333333333	105522000\\
12.55	105465000\\
12.5666666666667	105108000\\
12.5833333333333	104945000\\
12.6	104882000\\
12.6166666666667	104878000\\
12.6333333333333	104781000\\
12.65	104847000\\
12.6666666666667	104954000\\
12.6833333333333	105068000\\
12.7	105312000\\
12.7166666666667	105318000\\
12.7333333333333	105307000\\
12.75	105029000\\
12.7666666666667	105340000\\
12.7833333333333	105164000\\
12.8	104986000\\
12.8166666666667	105039000\\
12.8333333333333	105170000\\
12.85	105294000\\
12.8666666666667	105217000\\
12.8833333333333	105555000\\
12.9	105433000\\
12.9166666666667	105361000\\
12.9333333333333	105359000\\
12.95	105390000\\
12.9666666666667	105104000\\
12.9833333333333	105377000\\
13	105262000\\
13.0166666666667	105349000\\
13.0333333333333	105249000\\
13.05	105348000\\
13.0666666666667	105375000\\
13.0833333333333	105302000\\
13.1	105351000\\
13.1166666666667	105344000\\
13.1333333333333	105127000\\
13.15	105368000\\
13.1666666666667	104974000\\
13.1833333333333	104926000\\
13.2	104706000\\
13.2166666666667	104445000\\
13.2333333333333	104110000\\
13.25	104453000\\
13.2666666666667	104473000\\
13.2833333333333	104183000\\
13.3	104345000\\
13.3166666666667	104428000\\
13.3333333333333	104385000\\
13.35	103991000\\
13.3666666666667	103682000\\
13.3833333333333	103554000\\
13.4	103229000\\
13.4166666666667	103173000\\
13.4333333333333	102970000\\
13.45	102797000\\
13.4666666666667	102559000\\
13.4833333333333	102482000\\
13.5	102082000\\
13.5166666666667	102085000\\
13.5333333333333	102171000\\
13.55	102160000\\
13.5666666666667	102307000\\
13.5833333333333	102248000\\
13.6	102425000\\
13.6166666666667	102230000\\
13.6333333333333	100971000\\
13.65	101218000\\
13.6666666666667	101494000\\
13.6833333333333	101516000\\
13.7	101536000\\
13.7166666666667	101466000\\
13.7333333333333	101672000\\
13.75	101770000\\
13.7666666666667	103278000\\
13.7833333333333	104146000\\
13.8	104224000\\
13.8166666666667	104300000\\
13.8333333333333	103843000\\
13.85	103441000\\
13.8666666666667	103077000\\
13.8833333333333	103278000\\
13.9	103381000\\
13.9166666666667	103643000\\
13.9333333333333	103777000\\
13.95	103341000\\
13.9666666666667	103250000\\
13.9833333333333	102901000\\
14	103257000\\
14.0166666666667	101397000\\
14.0333333333333	98968000\\
14.05	99040200\\
14.0666666666667	101441000\\
14.0833333333333	103524000\\
14.1	104520000\\
14.1166666666667	102813000\\
14.1333333333333	103929000\\
14.15	104541000\\
14.1666666666667	99946100\\
14.1833333333333	97685800\\
14.2	98000700\\
14.2166666666667	94636500\\
14.2333333333333	95855200\\
14.25	97537700\\
14.2666666666667	98410100\\
14.2833333333333	96746600\\
14.3	98075000\\
14.3166666666667	99785200\\
14.3333333333333	101243000\\
14.35	102565000\\
14.3666666666667	99595100\\
14.3833333333333	88974500\\
14.4	81313700\\
14.4166666666667	78527600\\
14.4333333333333	78736600\\
14.45	83072700\\
14.4666666666667	88905300\\
14.4833333333333	91999900\\
14.5	95759200\\
14.5166666666667	95597900\\
14.5333333333333	91644400\\
14.55	88802400\\
14.5666666666667	84800400\\
14.5833333333333	86889100\\
14.6	91020700\\
14.6166666666667	94754600\\
14.6333333333333	98564600\\
14.65	100322000\\
14.6666666666667	101078000\\
14.6833333333333	102420000\\
14.7	103077000\\
14.7166666666667	99433700\\
14.7333333333333	97087300\\
14.75	95179000\\
14.7666666666667	93259100\\
14.7833333333333	92438000\\
14.8	91532100\\
14.8166666666667	88434800\\
14.8333333333333	84467500\\
14.85	78898500\\
14.8666666666667	77020500\\
14.8833333333333	76699900\\
14.9	79092000\\
14.9166666666667	82147300\\
14.9333333333333	84339000\\
14.95	85915200\\
14.9666666666667	87155600\\
14.9833333333333	88257500\\
15	89215800\\
15.0166666666667	90523300\\
15.0333333333333	91501100\\
15.05	88397200\\
15.0666666666667	87726500\\
15.0833333333333	88774700\\
15.1	89654500\\
15.1166666666667	90618700\\
15.1333333333333	90611200\\
15.15	89658500\\
15.1666666666667	90795100\\
15.1833333333333	91429100\\
15.2	90555100\\
15.2166666666667	90730500\\
15.2333333333333	91734900\\
15.25	92848500\\
15.2666666666667	94536800\\
15.2833333333333	92376200\\
15.3	94048600\\
15.3166666666667	94026100\\
15.3333333333333	93642600\\
15.35	93042900\\
15.3666666666667	92279400\\
15.3833333333333	91224900\\
15.4	89624100\\
15.4166666666667	88763000\\
15.4333333333333	89035900\\
15.45	90035800\\
15.4666666666667	91449300\\
15.4833333333333	92354500\\
15.5	92872900\\
15.5166666666667	93415500\\
15.5333333333333	94338500\\
15.55	94670600\\
15.5666666666667	89879700\\
15.5833333333333	86326400\\
15.6	85706600\\
15.6166666666667	86229900\\
15.6333333333333	85988400\\
15.65	85193500\\
15.6666666666667	84545700\\
15.6833333333333	84838900\\
15.7	85551300\\
15.7166666666667	85511800\\
15.7333333333333	84791800\\
15.75	85042700\\
15.7666666666667	84694000\\
15.7833333333333	84253700\\
15.8	84587500\\
15.8166666666667	84580400\\
15.8333333333333	84756000\\
15.85	85040800\\
15.8666666666667	85562400\\
15.8833333333333	85402200\\
15.9	84726300\\
15.9166666666667	84156300\\
15.9333333333333	84064300\\
15.95	85248000\\
15.9666666666667	83791100\\
15.9833333333333	87838600\\
16	79062900\\
16.0166666666667	68696700\\
16.0333333333333	68946800\\
16.05	71446500\\
16.0666666666667	71147100\\
16.0833333333333	72789700\\
16.1	74141500\\
16.1166666666667	76731100\\
16.1333333333333	79528200\\
16.15	80847200\\
16.1666666666667	81095800\\
16.1833333333333	81819900\\
16.2	82789700\\
16.2166666666667	84077500\\
16.2333333333333	85829100\\
16.25	86248500\\
16.2666666666667	85986000\\
16.2833333333333	81867000\\
16.3	80874500\\
16.3166666666667	84538100\\
16.3333333333333	85205500\\
16.35	87540500\\
16.3666666666667	87279400\\
16.3833333333333	88840500\\
16.4	87548300\\
16.4166666666667	87142400\\
16.4333333333333	86088300\\
16.45	85275500\\
16.4666666666667	82940500\\
16.4833333333333	81873100\\
16.5	82186200\\
16.5166666666667	79522500\\
16.5333333333333	74750400\\
16.55	74122900\\
16.5666666666667	74073400\\
16.5833333333333	74256100\\
16.6	75127000\\
16.6166666666667	76121900\\
16.6333333333333	76482900\\
16.65	76652600\\
16.6666666666667	76139800\\
16.6833333333333	75943600\\
16.7	75647600\\
16.7166666666667	75976800\\
16.7333333333333	75518800\\
16.75	76133100\\
16.7666666666667	76765900\\
16.7833333333333	76969600\\
16.8	77212700\\
16.8166666666667	77554000\\
16.8333333333333	77646800\\
16.85	77248900\\
16.8666666666667	77938000\\
16.8833333333333	78582700\\
16.9	78793600\\
16.9166666666667	78677700\\
16.9333333333333	78474400\\
16.95	78399100\\
16.9666666666667	78750000\\
16.9833333333333	78895400\\
17	79561100\\
17.0166666666667	79910000\\
17.0333333333333	80677800\\
17.05	81064200\\
17.0666666666667	81629700\\
17.0833333333333	82110400\\
17.1	82538800\\
17.1166666666667	82858800\\
17.1333333333333	83405800\\
17.15	83627800\\
17.1666666666667	83992400\\
17.1833333333333	84556600\\
17.2	84684100\\
17.2166666666667	84533500\\
17.2333333333333	84056500\\
17.25	83721000\\
17.2666666666667	83513500\\
17.2833333333333	82843800\\
17.3	83202800\\
17.3166666666667	82947800\\
17.3333333333333	82901800\\
17.35	83009800\\
17.3666666666667	83032400\\
17.3833333333333	83441800\\
17.4	83088000\\
17.4166666666667	83100300\\
17.4333333333333	83218100\\
17.45	82941100\\
17.4666666666667	82343500\\
17.4833333333333	81679500\\
17.5	81041900\\
17.5166666666667	79941000\\
17.5333333333333	78308900\\
17.55	76200700\\
17.5666666666667	73976900\\
17.5833333333333	72561600\\
17.6	72092900\\
17.6166666666667	72401000\\
17.6333333333333	73339300\\
17.65	74320000\\
17.6666666666667	74563400\\
17.6833333333333	75025100\\
17.7	75193900\\
17.7166666666667	75455800\\
17.7333333333333	75939400\\
17.75	76525800\\
17.7666666666667	77564600\\
17.7833333333333	78473100\\
17.8	79320400\\
17.8166666666667	79930000\\
17.8333333333333	80298100\\
17.85	77927800\\
17.8666666666667	76066500\\
17.8833333333333	75183400\\
17.9	75483300\\
17.9166666666667	75470800\\
17.9333333333333	75851000\\
17.95	76080100\\
17.9666666666667	75049900\\
17.9833333333333	74211700\\
18	73837100\\
18.0166666666667	73931700\\
18.0333333333333	73746200\\
18.05	73448000\\
18.0666666666667	72709900\\
18.0833333333333	71637600\\
18.1	70585900\\
18.1166666666667	70675500\\
18.1333333333333	70553900\\
18.15	71561800\\
18.1666666666667	73456400\\
18.1833333333333	74745100\\
18.2	75489800\\
18.2166666666667	74527700\\
18.2333333333333	74601100\\
18.25	73460500\\
18.2666666666667	71504300\\
18.2833333333333	70226600\\
18.3	71280100\\
18.3166666666667	72363000\\
18.3333333333333	73357800\\
18.35	74014200\\
18.3666666666667	75064100\\
18.3833333333333	77276200\\
18.4	79917800\\
18.4166666666667	81093800\\
18.4333333333333	82811800\\
18.45	85763400\\
18.4666666666667	85482900\\
18.4833333333333	85383700\\
18.5	85721600\\
18.5166666666667	86549800\\
18.5333333333333	86549500\\
18.55	83379400\\
18.5666666666667	83083600\\
18.5833333333333	80177500\\
18.6	81017900\\
18.6166666666667	81837400\\
18.6333333333333	81541000\\
18.65	82375900\\
18.6666666666667	82004400\\
18.6833333333333	83187900\\
18.7	83122400\\
18.7166666666667	83186200\\
18.7333333333333	82970900\\
18.75	83849600\\
18.7666666666667	84808100\\
18.7833333333333	81853300\\
18.8	74342600\\
18.8166666666667	71248500\\
18.8333333333333	69129300\\
18.85	69741100\\
18.8666666666667	70483100\\
18.8833333333333	71058900\\
18.9	72068900\\
18.9166666666667	72623400\\
18.9333333333333	72067900\\
18.95	69284500\\
18.9666666666667	65618400\\
18.9833333333333	62295600\\
19	61555900\\
19.0166666666667	61491600\\
19.0333333333333	62705400\\
19.05	63996800\\
19.0666666666667	65054000\\
19.0833333333333	65620000\\
19.1	64595800\\
19.1166666666667	64319000\\
19.1333333333333	63775600\\
19.15	63623300\\
19.1666666666667	63732300\\
19.1833333333333	63541200\\
19.2	63348300\\
19.2166666666667	63504900\\
19.2333333333333	64228500\\
19.25	65664500\\
19.2666666666667	67655700\\
19.2833333333333	68296400\\
19.3	68617200\\
19.3166666666667	67818500\\
19.3333333333333	67178600\\
19.35	66858800\\
19.3666666666667	67508400\\
19.3833333333333	68491400\\
19.4	65069400\\
19.4166666666667	63357000\\
19.4333333333333	61978400\\
19.45	60181200\\
19.4666666666667	60171000\\
19.4833333333333	61304300\\
19.5	61838900\\
19.5166666666667	62737400\\
19.5333333333333	62848600\\
19.55	62767000\\
19.5666666666667	63475800\\
19.5833333333333	63831400\\
19.6	63365500\\
19.6166666666667	62393600\\
19.6333333333333	63133400\\
19.65	64596600\\
19.6666666666667	65519600\\
19.6833333333333	64582400\\
19.7	64381600\\
19.7166666666667	65127900\\
19.7333333333333	66619600\\
19.75	68457800\\
19.7666666666667	69715700\\
19.7833333333333	71810900\\
19.8	72548900\\
19.8166666666667	72529400\\
19.8333333333333	72406900\\
19.85	71864900\\
19.8666666666667	66069500\\
19.8833333333333	62328000\\
19.9	61803000\\
19.9166666666667	59713200\\
19.9333333333333	60163100\\
19.95	59293400\\
19.9666666666667	57226000\\
19.9833333333333	57124300\\
20	58798600\\
20.0166666666667	59129000\\
20.0333333333333	59574900\\
20.05	60398200\\
20.0666666666667	61329000\\
20.0833333333333	62210000\\
20.1	62894800\\
20.1166666666667	63638400\\
20.1333333333333	63803300\\
20.15	62646000\\
20.1666666666667	61822000\\
20.1833333333333	62552300\\
20.2	62952600\\
20.2166666666667	62842500\\
20.2333333333333	63428500\\
20.25	63095800\\
20.2666666666667	61464900\\
20.2833333333333	58203900\\
20.3	56736500\\
20.3166666666667	56569400\\
20.3333333333333	56314400\\
20.35	56857800\\
20.3666666666667	57253900\\
20.3833333333333	57814200\\
20.4	59258800\\
20.4166666666667	60702500\\
20.4333333333333	61907300\\
20.45	63101900\\
20.4666666666667	63924500\\
20.4833333333333	64991700\\
20.5	65532500\\
20.5166666666667	65780600\\
20.5333333333333	67158100\\
20.55	68797900\\
20.5666666666667	67271200\\
20.5833333333333	67623000\\
20.6	68174500\\
20.6166666666667	69450000\\
20.6333333333333	70456200\\
20.65	72353700\\
20.6666666666667	73717800\\
20.6833333333333	74392500\\
20.7	75411400\\
20.7166666666667	75283400\\
20.7333333333333	73521100\\
20.75	71845500\\
20.7666666666667	70486900\\
20.7833333333333	68835700\\
20.8	67485100\\
20.8166666666667	66276600\\
20.8333333333333	65254600\\
20.85	65495100\\
20.8666666666667	66108500\\
20.8833333333333	66469400\\
20.9	67457300\\
20.9166666666667	67650300\\
20.9333333333333	67722200\\
20.95	68549000\\
20.9666666666667	69175100\\
20.9833333333333	67197100\\
21	66673800\\
21.0166666666667	66052400\\
21.0333333333333	64685100\\
21.05	62469700\\
21.0666666666667	60357500\\
21.0833333333333	59877500\\
21.1	59072400\\
21.1166666666667	58031100\\
21.1333333333333	57360400\\
21.15	56355500\\
21.1666666666667	55776800\\
21.1833333333333	56546400\\
21.2	56833900\\
21.2166666666667	56591300\\
21.2333333333333	56097300\\
21.25	55256300\\
21.2666666666667	54440400\\
21.2833333333333	54681000\\
21.3	55166600\\
21.3166666666667	55744200\\
21.3333333333333	56195300\\
21.35	56867400\\
21.3666666666667	56745100\\
21.3833333333333	56882600\\
21.4	57441500\\
21.4166666666667	57490600\\
21.4333333333333	57473200\\
21.45	58054100\\
21.4666666666667	58082300\\
21.4833333333333	58474500\\
21.5	58871200\\
21.5166666666667	58972500\\
21.5333333333333	59079900\\
21.55	59033700\\
21.5666666666667	59607400\\
21.5833333333333	59827600\\
21.6	60126600\\
21.6166666666667	60167000\\
21.6333333333333	59902000\\
21.65	59888200\\
21.6666666666667	60140900\\
21.6833333333333	60536700\\
21.7	61427800\\
21.7166666666667	62470900\\
21.7333333333333	59971800\\
21.75	57367900\\
21.7666666666667	55727400\\
21.7833333333333	54517200\\
21.8	53984800\\
21.8166666666667	53211100\\
21.8333333333333	53957100\\
21.85	53800000\\
21.8666666666667	53011300\\
21.8833333333333	52289700\\
21.9	52476300\\
21.9166666666667	52895200\\
21.9333333333333	53158300\\
21.95	53876400\\
21.9666666666667	54787500\\
21.9833333333333	54694700\\
22	56059200\\
22.0166666666667	58142700\\
22.0333333333333	57996100\\
22.05	58029800\\
22.0666666666667	57071400\\
22.0833333333333	56500000\\
22.1	53266600\\
22.1166666666667	52900600\\
22.1333333333333	53139700\\
22.15	53834600\\
22.1666666666667	54508600\\
22.1833333333333	50527700\\
22.2	49086100\\
22.2166666666667	48486800\\
22.2333333333333	46148100\\
22.25	45265800\\
22.2666666666667	44436800\\
22.2833333333333	44199700\\
22.3	43833400\\
22.3166666666667	44469000\\
22.3333333333333	45752400\\
22.35	45834300\\
22.3666666666667	45481000\\
22.3833333333333	45124300\\
22.4	45157900\\
22.4166666666667	45319800\\
22.4333333333333	45662200\\
22.45	46328900\\
22.4666666666667	45519100\\
22.4833333333333	45702900\\
22.5	45536600\\
22.5166666666667	45935200\\
22.5333333333333	46257900\\
22.55	47757500\\
22.5666666666667	49795200\\
22.5833333333333	51378400\\
22.6	50253200\\
22.6166666666667	49129000\\
22.6333333333333	48414700\\
22.65	46780700\\
22.6666666666667	45498800\\
22.6833333333333	45286600\\
22.7	45975200\\
22.7166666666667	45565700\\
22.7333333333333	45893300\\
22.75	45334800\\
22.7666666666667	43639300\\
22.7833333333333	42474200\\
22.8	42482400\\
22.8166666666667	43259900\\
22.8333333333333	42573400\\
22.85	42470300\\
22.8666666666667	41976900\\
22.8833333333333	41958500\\
22.9	42182900\\
22.9166666666667	42333500\\
22.9333333333333	42033700\\
22.95	42105900\\
22.9666666666667	41742700\\
22.9833333333333	41493500\\
23	41596500\\
23.0166666666667	41615200\\
23.0333333333333	42260000\\
23.05	42717500\\
23.0666666666667	43159900\\
23.0833333333333	43949200\\
23.1	44458300\\
23.1166666666667	44091700\\
23.1333333333333	44237700\\
23.15	44231200\\
23.1666666666667	44400400\\
23.1833333333333	44765300\\
23.2	45541900\\
23.2166666666667	45729600\\
23.2333333333333	46798300\\
23.25	47442900\\
23.2666666666667	48099700\\
23.2833333333333	48174800\\
23.3	48013300\\
23.3166666666667	48522500\\
23.3333333333333	48827600\\
23.35	49185600\\
23.3666666666667	49851500\\
23.3833333333333	49581500\\
23.4	49668200\\
23.4166666666667	49476600\\
23.4333333333333	48762300\\
23.45	47490100\\
23.4666666666667	47480700\\
23.4833333333333	47583100\\
23.5	47650100\\
23.5166666666667	47135100\\
23.5333333333333	47066500\\
23.55	46732100\\
23.5666666666667	45011600\\
23.5833333333333	44141300\\
23.6	43701900\\
23.6166666666667	43697600\\
23.6333333333333	44214500\\
23.65	44534900\\
23.6666666666667	45022100\\
23.6833333333333	45602300\\
23.7	44705200\\
23.7166666666667	44556000\\
23.7333333333333	44976600\\
23.75	44983400\\
23.7666666666667	45896200\\
23.7833333333333	45832400\\
23.8	46978100\\
23.8166666666667	47413500\\
23.8333333333333	48147200\\
23.85	47995800\\
23.8666666666667	48350400\\
23.8833333333333	48168200\\
23.9	48056200\\
23.9166666666667	48331000\\
23.9333333333333	49726300\\
23.95	50808600\\
23.9666666666667	48931000\\
23.9833333333333	48431300\\
24	48709600\\
24.0166666666667	49557200\\
24.0333333333333	49609600\\
24.05	48346700\\
24.0666666666667	48084400\\
24.0833333333333	49281400\\
24.1	49617900\\
24.1166666666667	49119000\\
24.1333333333333	50073500\\
24.15	50377400\\
24.1666666666667	51409500\\
24.1833333333333	50607500\\
24.2	48657200\\
24.2166666666667	45919300\\
24.2333333333333	44358200\\
24.25	44108100\\
24.2666666666667	44342500\\
24.2833333333333	44857000\\
24.3	45251300\\
24.3166666666667	45280400\\
24.3333333333333	45795800\\
24.35	45994700\\
24.3666666666667	45356500\\
24.3833333333333	46293500\\
24.4	48401200\\
24.4166666666667	48602700\\
24.4333333333333	48629600\\
24.45	46565400\\
24.4666666666667	46761200\\
24.4833333333333	46952500\\
24.5	47600600\\
24.5166666666667	46249900\\
24.5333333333333	46106700\\
24.55	46867800\\
24.5666666666667	47068200\\
24.5833333333333	46713400\\
24.6	46373500\\
24.6166666666667	45191800\\
24.6333333333333	44919700\\
24.65	45261100\\
24.6666666666667	45994700\\
24.6833333333333	46013500\\
24.7	46385800\\
24.7166666666667	46654400\\
24.7333333333333	47547600\\
24.75	47428500\\
24.7666666666667	47710400\\
24.7833333333333	47922300\\
24.8	49594700\\
24.8166666666667	49898400\\
24.8333333333333	50545300\\
24.85	50978700\\
24.8666666666667	52158100\\
24.8833333333333	54230500\\
24.9	54980700\\
24.9166666666667	55095900\\
24.9333333333333	56658800\\
24.95	57263700\\
24.9666666666667	58073000\\
24.9833333333333	59563400\\
25	61617400\\
25.0166666666667	61593200\\
25.0333333333333	61591300\\
25.05	61392500\\
25.0666666666667	62420300\\
25.0833333333333	64990600\\
25.1	65197900\\
25.1166666666667	66240300\\
25.1333333333333	65729500\\
25.15	66568300\\
25.1666666666667	64668800\\
25.1833333333333	64209800\\
25.2	61897200\\
25.2166666666667	60428300\\
25.2333333333333	60596600\\
25.25	60720000\\
25.2666666666667	61349600\\
25.2833333333333	61407100\\
25.3	60753100\\
25.3166666666667	59568000\\
25.3333333333333	59584800\\
25.35	59580200\\
25.3666666666667	61255400\\
25.3833333333333	61648300\\
25.4	57926900\\
25.4166666666667	59819200\\
25.4333333333333	61707300\\
25.45	62495800\\
25.4666666666667	62186000\\
25.4833333333333	63234100\\
25.5	64437400\\
25.5166666666667	65175100\\
25.5333333333333	65941000\\
25.55	67039600\\
25.5666666666667	66749000\\
25.5833333333333	65780500\\
25.6	56168600\\
25.6166666666667	56007700\\
25.6333333333333	57170000\\
25.65	56016100\\
25.6666666666667	58218100\\
25.6833333333333	56081000\\
25.7	56164500\\
25.7166666666667	50944100\\
25.7333333333333	48844500\\
25.75	48709400\\
25.7666666666667	49459800\\
25.7833333333333	50589600\\
25.8	52537300\\
25.8166666666667	54419900\\
25.8333333333333	55283000\\
25.85	54284200\\
25.8666666666667	54727700\\
25.8833333333333	55153700\\
25.9	54133300\\
25.9166666666667	52502600\\
25.9333333333333	51209000\\
25.95	48360800\\
25.9666666666667	47614300\\
25.9833333333333	45349700\\
26	43075900\\
26.0166666666667	43677700\\
26.0333333333333	45354500\\
26.05	46943700\\
26.0666666666667	49836900\\
26.0833333333333	50012800\\
26.1	52054100\\
26.1166666666667	51734300\\
26.1333333333333	53395200\\
26.15	50725300\\
26.1666666666667	52098100\\
26.1833333333333	54352200\\
26.2	53453500\\
26.2166666666667	52290400\\
26.2333333333333	49306100\\
26.25	47012600\\
26.2666666666667	45535500\\
26.2833333333333	45740700\\
26.3	46427000\\
26.3166666666667	46918900\\
26.3333333333333	47117900\\
26.35	46067600\\
26.3666666666667	45607000\\
26.3833333333333	45472200\\
26.4	44243900\\
26.4166666666667	43878400\\
26.4333333333333	44266500\\
26.45	44834900\\
26.4666666666667	45656600\\
26.4833333333333	46553800\\
26.5	47355700\\
26.5166666666667	47545500\\
26.5333333333333	48158900\\
26.55	48190600\\
26.5666666666667	47315400\\
26.5833333333333	45773700\\
26.6	45806800\\
26.6166666666667	46612700\\
26.6333333333333	46826000\\
26.65	46440100\\
26.6666666666667	45765800\\
26.6833333333333	44700100\\
26.7	42932600\\
26.7166666666667	42776400\\
26.7333333333333	42550700\\
26.75	42695600\\
26.7666666666667	42349500\\
26.7833333333333	42712700\\
26.8	44125200\\
26.8166666666667	45900900\\
26.8333333333333	47253800\\
26.85	47904100\\
26.8666666666667	48674500\\
26.8833333333333	49858700\\
26.9	50162000\\
26.9166666666667	49166900\\
26.9333333333333	48848300\\
26.95	48331900\\
26.9666666666667	47921400\\
26.9833333333333	47404300\\
27	47479100\\
27.0166666666667	47350100\\
27.0333333333333	47627900\\
27.05	46977300\\
27.0666666666667	46806200\\
27.0833333333333	46963100\\
27.1	46894800\\
27.1166666666667	47736300\\
27.1333333333333	48303400\\
27.15	49608200\\
27.1666666666667	50667600\\
27.1833333333333	51632000\\
27.2	52598500\\
27.2166666666667	53165800\\
27.2333333333333	54019700\\
27.25	55536100\\
27.2666666666667	56949600\\
27.2833333333333	56983800\\
27.3	56673800\\
27.3166666666667	55204100\\
27.3333333333333	55476800\\
27.35	55815000\\
27.3666666666667	56057400\\
27.3833333333333	55845300\\
27.4	56898400\\
27.4166666666667	58968300\\
27.4333333333333	59467200\\
27.45	59457700\\
27.4666666666667	60658900\\
27.4833333333333	61501900\\
27.5	61968700\\
27.5166666666667	62197000\\
27.5333333333333	62356600\\
27.55	62552900\\
27.5666666666667	63637200\\
27.5833333333333	63720800\\
27.6	64305800\\
27.6166666666667	65416900\\
27.6333333333333	65446700\\
27.65	65757600\\
27.6666666666667	63032100\\
27.6833333333333	60988800\\
27.7	61151400\\
27.7166666666667	62230400\\
27.7333333333333	59222800\\
27.75	58778200\\
27.7666666666667	58008000\\
27.7833333333333	57695800\\
27.8	57116900\\
27.8166666666667	56752500\\
27.8333333333333	57114800\\
27.85	57154200\\
27.8666666666667	57710300\\
27.8833333333333	58492200\\
27.9	59725000\\
27.9166666666667	61845800\\
27.9333333333333	62668500\\
27.95	63481000\\
27.9666666666667	63756700\\
27.9833333333333	64033000\\
28	65213700\\
28.0166666666667	65527200\\
28.0333333333333	66853600\\
28.05	67240500\\
28.0666666666667	67585500\\
28.0833333333333	68599000\\
28.1	69867300\\
28.1166666666667	69667800\\
28.1333333333333	67323700\\
28.15	65808100\\
28.1666666666667	65154100\\
28.1833333333333	64720100\\
28.2	64788400\\
28.2166666666667	65646000\\
28.2333333333333	65620000\\
28.25	67193800\\
28.2666666666667	68166800\\
28.2833333333333	68656800\\
28.3	69411200\\
28.3166666666667	70250700\\
28.3333333333333	70556100\\
28.35	72314000\\
28.3666666666667	70466800\\
28.3833333333333	69822200\\
28.4	69394800\\
28.4166666666667	70066200\\
28.4333333333333	68457400\\
28.45	62638600\\
28.4666666666667	60418000\\
28.4833333333333	57687600\\
28.5	60628400\\
28.5166666666667	59188400\\
28.5333333333333	59730600\\
28.55	60126600\\
28.5666666666667	58781100\\
28.5833333333333	58699000\\
28.6	60520500\\
28.6166666666667	59462300\\
28.6333333333333	60072800\\
28.65	58985500\\
28.6666666666667	58912200\\
28.6833333333333	58482500\\
28.7	58093300\\
28.7166666666667	57620300\\
28.7333333333333	54599700\\
28.75	54014700\\
28.7666666666667	53867000\\
28.7833333333333	53935700\\
28.8	54247100\\
28.8166666666667	54043300\\
28.8333333333333	53698800\\
28.85	53258600\\
28.8666666666667	53095700\\
28.8833333333333	52989400\\
28.9	50820400\\
28.9166666666667	53139700\\
28.9333333333333	54008100\\
28.95	49715500\\
28.9666666666667	46460900\\
28.9833333333333	45740000\\
29	44907700\\
29.0166666666667	43465000\\
29.0333333333333	43526100\\
29.05	43730200\\
29.0666666666667	44485700\\
29.0833333333333	45140300\\
29.1	46902600\\
29.1166666666667	45861400\\
29.1333333333333	48454500\\
29.15	49091600\\
29.1666666666667	50054800\\
29.1833333333333	49618000\\
29.2	50602500\\
29.2166666666667	51945300\\
29.2333333333333	52255800\\
29.25	51918300\\
29.2666666666667	52136600\\
29.2833333333333	52759400\\
29.3	52494700\\
29.3166666666667	52869200\\
29.3333333333333	52942400\\
29.35	53589900\\
29.3666666666667	54601500\\
29.3833333333333	54201400\\
29.4	51794600\\
29.4166666666667	51576500\\
29.4333333333333	50354500\\
29.45	49050400\\
29.4666666666667	48071400\\
29.4833333333333	47529900\\
29.5	46471400\\
29.5166666666667	47191400\\
29.5333333333333	47651800\\
29.55	46880200\\
29.5666666666667	47954500\\
29.5833333333333	49142900\\
29.6	51218300\\
29.6166666666667	51141100\\
29.6333333333333	51454500\\
29.65	52519700\\
29.6666666666667	53188400\\
29.6833333333333	54098400\\
29.7	54377700\\
29.7166666666667	56162000\\
29.7333333333333	56390100\\
29.75	56467700\\
29.7666666666667	57180500\\
29.7833333333333	57869600\\
29.8	59578800\\
29.8166666666667	59956500\\
29.8333333333333	60091800\\
29.85	60380200\\
29.8666666666667	59388400\\
29.8833333333333	58220900\\
29.9	57768100\\
29.9166666666667	57692400\\
29.9333333333333	57898700\\
29.95	57676500\\
29.9666666666667	57100600\\
29.9833333333333	56779000\\
};

\end{axis}

\begin{axis}[%
width=5.833in,
height=4.375in,
at={(0in,0in)},
scale only axis,
xmin=0,
xmax=1,
ymin=0,
ymax=1,
axis line style={draw=none},
ticks=none,
axis x line*=bottom,
axis y line*=left,
legend style={legend cell align=left, align=left, draw=white!15!black}
]
\end{axis}
\end{tikzpicture}%
}
	\end{center}
	\caption{Isolation wetted only inside with cable, isolation surface not wiped and needles stuck vertically on the bottom of the isolation}
	\label{fig:inside}
\end{figure}

The last experiment was performed with 1cm needles which did not pierced fully the isolation to the inner hole and fully wet. Isolation surface was wiped carefully from the water, so potential smaller resistance than dry material is caused by the water inside the \verb|Armaflex|. Result is presented on the \figurename{} \ref{fig:inside}. Figure consists of 4 curves from which all of them are between range 10M$\Omega$ and 40M$\Omega$.

\begin{figure}[H]
	\begin{center}
		\scalebox{.7}{\input{./plots/inside_short.tex}}
	\end{center}
	\caption{Isolation fully wet with cable, isolation surface wiped and needles 1cm needless stuck in the middle}
	\label{fig:inside_short}
\end{figure}

\section{Conclusion}
The best results are achieved during experiments showed on \figurename{} \ref{fig:full_3} and \figurename{} \ref{fig:inside_short}. First of them, which is needles stuck to the down part of isolation tube gave relatively stable measurements in time, with values of hundreds kilo ohms, which is much lower than dry tube. However it is quite problematic to stick needles exactly in the same place with similar depth and distance to the inner hollow part. Experiment on \figurename{} \ref{fig:inside_short} gave also similar results for 4 tests, but the values behave similar to noise, so potential hint that there is a problem with isolation can be discovered by averaging resistance readouts and comparing them with the average resistance of dry isolation. Additionally resistance is pretty high -- 8M$\Omega$, what us really close to the dry material. Talking above into consideration there is possibility to utilise needles as a primitive humidity sensor when they will be stuck to the bottom of \verb|Armaflex| mainly because low resistance, however they should be stuck relatively in the same depth and place.

\section{Technology and future work}
After verifying the principle, the next step will be to establish an on-line measuring system using resistance-measuring equipment that can be directly integrated in the detector \verb|DCS| \verb|DSS|

\listoffigures

\end{document}